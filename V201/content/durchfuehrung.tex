\section{Durchführung}
\label{sec:Durchführung}
\subsection{Messung zur Bestimmung der Wärmekapazität des Kaloriemeters}
Zur Bestimmung der Wärmekapazität des Kaloriemeters werden kaltes und erhitztes
Wasser im Kaloriemeter zusammengemischt, deren jeweilgen Temperaturen
$T_{\mathrm{X}}$ und $T_{\mathrm{Y}}$ werden mit einem
Thermoelement vermessen und deren Gewicht $M_{\mathrm{X}}$ und $M_{\mathrm{Y}}$
bestimmt. Nach einiger Zeit stellt sich eine Mischtemperatur $T_{\mathrm{M}}$
ein, diese wird ebenfalls aufgenommen.

\subsection{Messung zur Bestimmung der spezifischen Wärmekapazität unterschiedlicher Stoffe}
Das Kaloriemeter wird mit Wasser der zuvor gemessenen Masse $m_{\mathrm{w}}$ befüllt.
Die Masse $M_\mathrm{k}$ des Körpers wird gewogen, dieser wird anschließend in einem
Wasserbad erhitzt. Der Körper, mit der Temperatur $T_{\mathrm{k}}$, wird zum Wasser
im Kaloriemeter gegeben. Im Körper und im Wasser stellt sich eine Mischtemperatur
$T_{\mathrm{M}}$ ein, diese wird auch an beiden gemessen.
Diese Messung vier mal durchgeführt, davon drei mal für einen Stoff und ein mal für einen
anderen Stoff.
