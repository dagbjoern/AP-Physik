\section{Durchführung}
\label{sec:Durchführung}
\subsection{Messung zur Bestimmung der Wärmekapazität des Kalorimeters}
Zur Bestimmung der Wärmekapazität des Kalorimeters werden kaltes und erhitztes
Wasser im Kaloriemeter zusammengemischt, deren jeweiligen Temperaturen
$T_{\mathrm{x}}$ und $T_{\mathrm{y}}$ werden zuvor mit einem
Thermoelement vermessen und deren Gewicht $M_{\mathrm{x}}$ und $M_{\mathrm{y}}$
bestimmt. Nach einiger Zeit stellt sich eine Mischtemperatur $T'_{\mathrm{m}}$
ein, diese wird ebenfalls aufgenommen.

\subsection{Messung zur Bestimmung der spezifischen Wärmekapazität unterschiedlicher Stoffe}
Das Kalorimeter wird mit Wasser der zuvor gemessenen Masse $m_{\mathrm{w}}$ befüllt.
Die Masse $M_\mathrm{k}$ des Körpers wird gewogen, dieser wird anschließend in einem
Wasserbad erhitzt. Der Körper, mit der Temperatur $T_{\mathrm{K}}$, wird zum Wasser
im Kaloriemeter gegeben. Im Körper und im Wasser stellt sich eine Mischtemperatur
$T_{\mathrm{M}}$ ein, diese wird auch an beiden gemessen.
Diese Messung vier mal durchgeführt, davon drei mal für Zinn und ein mal für Graphit.
