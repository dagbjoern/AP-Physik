\newpage
\section{Fehlerrechnung}
\label{fehlerrechnung}
In der Auswertung müssen Mittelwerte und Standartabweichungen berechnet werden.
Die Formel für die Mittelwerte lautet
\begin{align}
  \bar{x}=\frac{1}{n} \sum_{i=1}^n x_i\\
\intertext{und für die Standardabweichung ergibt sich:}
s_i=\sqrt{\frac{1}{n-1}\sum_{j=1}^n (v_j-\bar{v_i})^2}
\end{align}
mit $v_j$ mit $j=1,..,n$ als Wert mit zufällig behafteten Fehlern.\\
\\
Diese werden mit Hilfe von
Numpy 1.9.2, einer Erweiterung von Python 3.2.0, berechnet.
\\
Die Fehlerfortpflanzung wird mit der Gauß´schen Fehlerfortpflanzung berechnet
 \eqref{eqn:gaus}.
\begin{equation}
\Delta f= \sqrt{\left(\frac{\partial f}{\partial x}\Delta x \right)^{2} + \left( \frac{\partial f}{\partial y}\Delta y\right)^2...}\label{eqn:gaus}.
\end{equation}
Diese wird von der Erweiterung Uncertainties 2.4.6.1 von Python 3.2.0 übernommen.\\
\\
Für die Abweichung von den Theoriewerten wird die Formel:
\begin{align}
  \biggl|  \frac{a_\mathrm{Messung}-a_\mathrm{Theorie}}{a_\mathrm{Theorie}}  \biggr|
\end{align}
