\section{Auswertung}
\label{sec:Auswertung}
\subsection{Bestimmung der Wärmekapazität vom Kaloriemeter}
Um die Wärmekapazität des Kaloriemeters zu bestimmen muss folgende Beziehung betrachtet werden:
Die vom warmen Wasser abgegebene Wärmemenge
\begin{align}
Q_\mathrm{1}=c_\mathrm{k} m_\mathrm{k}(T_\mathrm{k}-T_\mathrm{m})
\end{align}
entspricht der vom kalten Wasser und von den Wänden des kaloriemeters aufgenommenen Wärmemenge:
\begin{align}
Q_\mathrm{2}=(c_\mathrm{w} m_\mathrm{w} + c_\mathrm{g} m_\mathrm{g}) (T_\mathrm{m}-T_\mathrm{w}).
\end{align}
$c_\mathrm{w}$ ist hier die spezifische Wärmekapazität des Wassers.
Da $Q_\mathrm {1}=Q_\mathrm{2}$ gilt, ergibt sich nach weiteren Umformungen:
\begin{align}
c_\mathrm{g} m_\mathrm{g}=\frac{c_\mathrm{w} m_\mathrm{w}(T_\mathrm{w}-T_\mathrm{m})-c_\mathrm{w} m_\mathrm{k}(T_\mathrm{m}-T_\mathrm{k})}{(T_\mathrm{m}-T_\mathrm{k})}
\end{align}\\
%\begin{align*}
%c_\mathrm{g}=
%m_\mathrm{g}=\frac{c_\mathrm{w} m_\mathrm{w}(T_\mathrm{w}-T_\mathrm{m})-c_\mathrm{w} m_\mathrm{k}(T_\mathrm{m}-T_\mathrm{k})}{(T_\mathrm{m}-T_\mathrm{k})}
%\end{align*}\\
Mit den folgenden Werten,
\begin{align}
  c_\mathrm{g}m_\mathrm{g}=256,73 \si{\joule\per\kelvin}.
\intertext{Wärmekapazität Wasser:}
  c_\mathrm{w}&=4,18\,\si{\joule\per\gram\per\kelvin}\\
\intertext{Masse warmes Wasser:}
  m_\mathrm{w}&=327,11 \,\si{\gram}\\
\intertext{Masse kaltes Wasser:}
  m_\mathrm{k}&=240,06 \, \si{\gram}\\
\intertext{Temperatur warmes Wasser:}
  T_\mathrm{w}&=358,4 \, \si{\kelvin}\\
\intertext{Temperatur kaltes Wasser:}
  T_\mathrm{k}&=294,64\, \si{\kelvin}\\
\intertext{Temperatur gemischtes Wasser:}
  T_\mathrm{m}&=327,82\, \si{\kelvin}\\
\intertext{ergibt sich für die Wärmekapazität des Kalorimeters:}
  c_\mathrm{g}m_\mathrm{g}&=256,73 \si{\joule\per\kelvin}
\end{align}


\subsection{Bestimmung der spezifische Wärmekapazität von Zinn und Graphit}
Für die spezifische Wärmekapazitat von verschiedenen Stoffen ergibt sich:
\begin{align}
c_\mathrm{k}=\frac{(c_\mathrm{w} m_\mathrm{w}+c_\mathrm{g} m_\mathrm{g})(T_\mathrm{m}-T_\mathrm{w})}{m_\mathrm{k}(T_\mathrm{k}-T_\mathrm{m})}\label{eqn:ck}.
\end{align}
Werden nun die gemessenen Werte für Zinn aus der Tabelle \ref{tab:zinn} in die Formel \eqref{eqn:ck} eingesetzt und die erhaltenen $c_{\mathrm{Zinn}}$ gemittelt.
Daraus ergibt sich ein $c_{\mathrm{Zinn}}$ von
\begin{align*}
c_\mathrm{Zinn} &=  (0,43\pm0,13 \,\si{\joule\per\kelvin\gram}.\\
\intertext{Dieser weicht um $85\si{\percent}$ von dem Literaturwert}
 c_\mathrm{zTheorie} &= 0,23 \,\si{\joule\per\kelvin\gram}
\intertext{ab.\cite{ck}}
\end{align*}
\FloatBarrier
\begin{table}
  \centering
  \caption{Messwerte aus dem Versuch mit Zinn.}
  \label{tab:zinn}
   \begin{tabular}{c c c c c}
\toprule
Masse Zinn & Masse Wasser & Temperatur Zinn & Temperatur Wasser  & Mischtemperatur \\
$m_\mathrm{Zinn}/\si{\gram}$ & $m_\mathrm{Wasser}/\si{\gram}$ & $T_\mathrm{Zinn}/\si{\kelvin}$ & $T_\mathrm{Wasser}/\si{\kelvin}$ & $T_\mathrm{Misch}/\si{\kelvin}$ \\
\midrule
     232,41 &   450,01 &  352,19 &   294,15  &   295,64 \\
     232,41 &   470,56 &  351,71 &   293,88  &   296,88 \\
     232,41 &   491,59 &  349,31 &   293,65  &   296,38 \\
\bottomrule
\end{tabular}
\end{table}
\FloatBarrier
Die spezifische Wärmekapazität von Graphit ergibt sich aus den Werten von Tabelle \ref{tab:graphit},
die ebenfalls in die Formel \eqref{eqn:ck} eigesetzt werden:

\begin{align*}
c_\mathrm{Graphit} &= 1,19 \,\si{\joule\per\kelvin\gram}.
\intertext{Dieser Werte besitzt eine Abweichung von $66\si{\percent}$ vom Literaturwert\cite{ck}:}
c_\mathrm{gTheorie} &= 0.715\,\si{\joule\per\kelvin\gram}.
\end{align*}

\begin{table}
  \centering
  \caption{Messwerte aus dem Versuch mit Graphit.}
  \label{tab:graphit}
   \begin{tabular}{c c c c c}
\toprule
Masse Zinn & Masse Wasser & Temperatur Zinn & Temperatur Wasser  & Mischtemperatur \\
$m_\mathrm{Graphit}/\si{\gram}$ & $m_\mathrm{Wasser}/\si{\gram}$ & $T_\mathrm{Graphit}/\si{\kelvin}$ & $T_\mathrm{Wasser}/\si{\kelvin}$ & $T_\mathrm{Misch}/\si{\kelvin}$ \\
\midrule
107,67 & 465,18 & 352,19 & 293,40 & 296,63 \\
\bottomrule
\end{tabular}
\end{table}




\subsection{Bestimmung der Molwärme von Zinn und Graphit}
Die Molwärme für konstantes Volumen berechnet sich nach:
\begin{align}
  C_\mathrm{V} & =C_\mathrm{P}-9\alpha^2\kappa V_\mathrm{0}T_\mathrm{m}\\
  C_\mathrm{V} & =c_\mathrm{k}\cdot M_\mathrm{k}-9\alpha^2\kappa \frac{M_\mathrm{k}}{\rho}T_\mathrm{m}.
\end{align}
Die Werte für $\rho$, $M$, $\alpha$ und $\kappa$ finden sich in der Versuchsanleitung.
Für die Molwärme von Zinn und Graphit ergibt sich durch Mittelwertbildung:
\begin{align*}
  C_\mathrm{VZ}&=48,86 \ \si{\joule\per\mol\kelvin}\\
  C_\mathrm{VG}&=9,64  \ \si{\joule\per\mol\kelvin}.
\end{align*}
Laut Theorie soll sich für alle Molwärmen der Wert $3R$ gelten, somit ergibt sich
eine Abweichung von
$96\si{\percent}$
für Zinn und $43\si{\percent}$ für Graphit.
