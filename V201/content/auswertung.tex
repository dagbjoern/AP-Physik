\section{Auswertung}
\label{sec:Auswertung}
\subsection{Bestimmung der spezifische Wärmekapazität vom Kaloriemeter}
Um die Wärmekapazität des Kaloriemeters zu bestimmen muss folgende Beziehung betrachtet werden:
Die vom warmen Wasser abgegebene Wärmemenge
\begin{align}
Q_\mathrm{1}=c_\mathrm{k} m_\mathrm{k}(T_\mathrm{k}-T_\mathrm{m})
\end{align}
entspricht der vom kalten Wasser und von den Wänden des kaloriemeters aufgenommenen Wärmemenge:
Q_\mathrm{2}=(c_\mathrm{w} m_\mathrm{w}+c_\mathrm{g} m_\mathrm{g})(T_\mathrm{m}-T_\mathrm{w}).
Da $Q_\mathrm{1}=Q_\mathrm{2}$ ergibt sich nach Umformungen:
\begin{align}
c_\mathrm{g} m_\mathrm{g}=\frac{c_\mathrm{w} m_\mathrm{y}\lvert(T_\mathrm{y}-T_\mathrm{m}\rvert)-c_\mathrm{w} m_\mathrm{x}\lvert(T_\mathrm{m}-T_\mathrm{x}\rvert)}{(T_\mathrm{m}-T_\mathrm{x}\rvert)}
\end{align}
Mit den Werten aus der Tabelle \ref{} ergibt sich für die spezifische Wärmekapazität des Kalorimeters:
\begin{align}

\end{align}

\subsection{Bestimmung der spezifische Wärmekapazität von Zinn und Graphit}
Für die spezifische Wärmekapazitat von verschiedenen Stoffen ergibt sich:
\begin{align}
c_\mathrm{k}=\frac{(c_\mathrm{w} m_\mathrm{w}+c_\mathrm{g} m_\mathrm{g})(T_\mathrm{m}-T_\mathrm{w})}{m_\mathrm{k}(T_\mathrm{k}-T_\mathrm{m})}
\end{align}
