\section{Auswertung}
\label{sec:Auswertung}
\subsection{Bestimmung der spezifische Wärmekapazität vom Kaloriemeter}
Um die Wärmekapazität des Kaloriemeters zu bestimmen muss folgende Beziehung betrachtet werden:
Die vom warmen Wasser abgegebene Wärmemenge
\begin{align}
Q_\mathrm{1}=c_\mathrm{k} m_\mathrm{k}(T_\mathrm{k}-T_\mathrm{m})
\intertext{entspricht der vom kalten Wasser und von den Wänden des kaloriemeters aufgenommenen Wärmemenge:}
Q_\mathrm{2}= (c_\mathrm{w} m_\mathrm{w}+c_\mathrm{g} m_\mathrm{g})(T_\mathrm{m}-T_\mathrm{w}).\\
\end{align}
Da $Q_\mathrm{1}=Q_\mathrm{2}$ ergibt sich nach Umformungen:
\begin{align}
c_\mathrm{g} \cdot m_\mathrm{g}=\frac{c_\mathrm{w} m_\mathrm{y}\lvert(T_\mathrm{y}-T_\mathrm{m}\rvert)-c_\mathrm{w} m_\mathrm{x}\lvert(T_\mathrm{m}-T_\mathrm{x}\rvert)}{(T_\mathrm{m}-T_\mathrm{x}\rvert)}
\end{align}
Mit den Werten aus der Tabelle \ref{} ergibt sich für die spezifische Wärmekapazität des Kalorimeters:
\begin{align*}
c_\mathrm{g} \cdot m_\mathrm{g}=256,73 \, \si{\joule\per\kelvin}
\end{align*}

\subsection{Bestimmung der spezifische Wärmekapazität von Zinn und Graphit}
Für die spezifische Wärmekapazitat von verschiedenen Stoffen ergibt sich:
\begin{align}
c_\mathrm{k}=\frac{(c_\mathrm{w} m_\mathrm{w}+c_\mathrm{g} m_\mathrm{g})(T_\mathrm{m}-T_\mathrm{w})}{m_\mathrm{k}(T_\mathrm{k}-T_\mathrm{m})}\label{eqn:ck}
\end{align}
Werden nun die gemessenen Werte für Zinn aus der Tabelle \ref{tab.zinn} in die Formel \eqref{eqn:ck} eingesetzt und die erhaltenen $c_{\mathrm{Zinn}}$ gemittelt.
Daraus ergibt sich ein $c_{\mathrm{Zinn}}$ von
\begin{align*}
c_\mathrm{Zinn} =
\end{align*}
\begin{table}
  \centering
  \caption{Messwerte aus dem Versuch mit Zinn.}
  \label{tab:zinn}
   \begin{tabular}{c c c c c}
\toprule
Masse Zinn & Masse Wasser & Temperatur Zinn & Temperatur Wasser  & Mischtemperatur \\
$m_\mathrm{Zinn}/\si{\gram}$ & $m_\mathrm{Wasser}/\si{\gram}$ & $T_\mathrm{Zinn}/\si{\kelvin}$ & $T_\mathrm{Wasser}/\si{\kelvin}$ & $T_\mathrm{Misch}/\si{\kelvin}$ \\
\midrule
     232.41 &   450.01 &  352.19 &   294.15  &   295.64 \\
     232.41 &   470.56 &  351.71 &   293.88  &   296.88 \\
     232.41 &   491.59 &  349.31 &   293.65  &   296.38 \\
\bottomrule
\end{tabular}
\end{table}
