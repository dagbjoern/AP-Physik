\section{Diskussion}
\label{sec:Diskussion}
In allen Mesungen ergeben sich eine große Abweichungen von den theoretischen Werten der spezifischen Wärmekapazität.
Dies lässt sich einmal dadurch erklären ,dass beim Aufstellen der Formel für die spezifische Wärmekapazität der wärme Verlust an die Umgebung vernächlässigt wird.
Da die Geräte sind nicht ausreichend isoliert um die Wärmeabgabe nach außen hin zu vernachlässigen, lassen sich somit die
Abweichungen der spezifischen Wärmekapazität begründ.
Ebenfalls kann angenommen werden, dass die Mischtemperatur nicht richtig gemessen wurde, da
die Wärmeverteilung im Kaloriemeter während der Messung inhomogen ist\footnote{Durch längere Wartezeiten würde die Temperatur homogen werden jedoch
auch mehr Wärme entweichen was ebenfalls zu einem ungenauen Messergebnis führt} , wird somit die Temperatur des Wassers nahe am
Körper gemessen, ergibt sich eine ähnliche Temperaturen für Wasser und Körper.
Jedoch kann das Wasser unten im Gefäß kälter sein als das Wasser in der nähe des Körpers, somit erschwert sich die Messung der Temperatur.

Durch die fehlerbehaftete Wärmekapazität ergibt sich somit auch eine ungenaue Molwärme.
Diese sollte nach dem Dulong-Petitsche Gesetz genau 3R betragen, aber da die aus Messung
berecheneten Molwärmen davon abweichen, kann das Gesetz mit diesem Versuch weder
bestätigt werden noch  wiederlegt werden, da die Messungenaugkeiten bei dem Versuch zu hoch sind.    
