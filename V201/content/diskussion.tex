\section{Diskussion}
\label{sec:Diskussion}
In allen Mesungen ergeben sich abweichungen von den theoretischen Werten.
Beim Aufstellen der Formel für die spezifische Wärmekapazität, wird angenommen, dass keine Wärme an die Umgebung abgeht.
Die Geräte sind nicht ausreichend isoliert um die Wärmeabgabe nach außen hin zu vernachlässigen, damit lassen sich die Abweichungen der spezifischen Wärmekapazität begründ.
Ebenfalls kann angenommen werden, dass die Mischtemperatur nicht richtig gemessen wurde. Die Wärmeverteilung im Kaloriemeter ist inhomogen, wird die Temperatur des Wassers nahe am
Körper gemessen, so ergeben sich ähnliche Temperaturen für Wasser und Körper. Das Wasser unten im Gefäß ist aber kälter als das Wasser in der nähe des Körpers.


Bei der Berechnung der Wärmekapazität des Kaloriemeters wird angenommen, dass die vom warmen Wasser abgegebene Wärmemenge
der vom kalten Wasser und den von den Wänden des Kaloriemeters aufgenommenen Wärmemenge entspricht.
