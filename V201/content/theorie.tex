\section{Theorie}
\label{sec:Theorie}
\subsection{Klassische Betrachtung}
Die Molwärme $C$ beschreibt die Wärmemenge $dQ$
die benötigt wird um ein Mol eines Stoffes um $dT$ zu erwärmen.
An dieser Stelle wird die spezifische Wärmekapazität bei konstantem Volumen und nicht bei
konstantem Druck unterucht:
\begin{align}
  C_{\mathrm{V}}=\frac{dQ}{dT}.
\end{align}
Das Dulong-Petitsche Gesetz besagt, dass die Molwärme bei konstantem Volumen
$3R$ beträgt, mit $R$ als allgemeinen Gaskonstante, unabhängig von den Stoffeigenschaften.
Dieser Wert lässt sich durch Energiebetrachtungen am Oszillator herleiten. Da Atome in einem
festen Körper durch Gitterkräfte gebunden sind, führen diese harmonische Schwingungen
aus. Es ergibt sich schließlich der folgende Zusammenhang für die mittlere Gesammtenergie
eines Atoms:
\begin{align}
  \bigl<u\bigr>=\bigl<E_{\mathrm{kin}}\bigr>+\bigl<E_{\mathrm{pot}}\bigr>=2\bigl<E_{\mathrm{kin}}\bigr>.
\end{align}
Unter Berücksichtigung des Äquipartitionstheorems, welches besagt, dass ein Atom
pro Freiheitsgrad die mittlere kinetische Energie $\frac{1}{2}kT$ besitzt,
wenn kein Temperaturunterschied zur Umgebung herrscht, ergibt sich für
die gesamte Energie, eines auf einem Gitterplatz schwingenden Atoms, die Bezieung:
\begin{align}
  \bigl<u\bigr>=2\bigl<E_{\mathrm{kin}}\bigr>=kT.
\end{align}
Für ein Mol eines Stoffes ergibt sich mit $N_\mathrm{L}$ Atomen eine mittlere Energie von:
\begin{align}
  \bigl<U\bigr>=N_{\mathrm{L}}\bigl<u\bigr>=N_{\mathrm{L}}kT=RT
\end{align}
pro Bewegungsfreiheitsgrad.
Die Molwärme $C_{\mathrm{V}}$ hat den Dulong-Petitschen Wert:
\begin{align}
  C_{\mathrm{V}}=3R.
\end{align}\\
\subsection{Quantenmechanische Betrachtung}
Eine Problematik ergibt sich bei niedrigen Temperaturen, da wird der Molwärme Wert
von $3R$ nicht erreicht. Eine quantenmeschanische Betrachtung wird herangezogen.
Die Quantentheorie besagt, dass ein Oszillator nur Energien von bestimmter Größe
aufnehmen oder abgeben kann.
Es gilt:
\begin{align}
  \Delta u=n\cdot\hbar\cdot\omega
\end{align}
Mit weiteren Betrachtungen ergibt sich für die mittlere Energie:
\begin{align}
\bigl<U_{\mathrm{qu}}\bigr>=\frac{3N_{\mathrm{L}}\hbar\omega}{e^{\frac{\hbar\omega}{kT}}-1}.
\end{align}
Gilt $kT \gg \hbar\omega$, so strebt die Energie wieder gegen $3R$.
