\section{Theorie}
\label{sec:Theorie}
Die magnetische Flussdichte $\vec{B}$ in Materie setzt sich zusammen aus der magnetischen Feldstärke $\vec{H}$,
der Induktionskonstante $\mu_\mathrm{0}$ und der Magnetisierung $\vec{M}$ zur
folgenden Beziehung:
\begin{align}
  \vec{B}=\mu_\mathrm{0}\vec{H}+\vec{M}.
\end{align}
Die Magnetisierung wird verursacht durch magnetische Momente und hängt ab
von $\vec{H}$ ab. Es gilt die Beziehiung:
\begin{align}
  \vec{M}=\mu_\mathrm{0}\chi\vec{H}.
\end{align}
Die Suszeptibilität $\chi$ ist von $\vec{H}$ und der Temperatur $T$ abhängig.
Die Suszeptibilität ist negativ im Falle des Diamagnetismus,
dieser stammt von einer Induktion magnetischer Momente mittels äußerem
Magnetfeld und ist eine allgemeine Eigenschaft der Materie.
Dem gegenüber steht der Paramagnetismus, dieser tritt nur auf bei
Atomen mit einem Drehimpuls ungleich null. Der Paramagnetismus wird
verursacht von der Orientierung magnetischer Momente relativ
zu einem äußeren Feld, diese sind mit dem Drehimpuls gekoppelt.
Durch thermische Bewegung wird die Ausrichtung der magnetischen Momente
gestört, damit ist der Paramagnetismus eine Größe, die von der Temperatur
abhängt.
Magnetische Momente und der Dehimpuls sind gekoppelt.
Der Gesamtdrehimpuls $\vec{J}$ eines Atoms ergibt sich aus
$\vec{L}$ dem Bahndrehimpuls der Elektronenhülle, $\vec{S}$ dem Spin
der Elektronen und dem Kerndrehimpuls, welcher bei Paramagnetismus
vernachlässigbar ist.
