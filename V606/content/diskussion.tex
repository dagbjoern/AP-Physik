\section{Diskussion}
\label{sec:Diskussion}
Bei dem Ergebnissen für die Suzeptibilität $\chi$
der Stoffe wird ein Unterschied
zwischen den zwei Messverfahren deutlich.
Das Messverfahren bei dem die Brückenspannung $U_sp$ gemessen besitzt
eine größere Abweichung zu den Theoriewerten
im Vergleich zu der Methode bei dem dem $\delta R$ bestimmt wird.
Dies könnte daran liegen, dass in die Formel \eqref{eqn:chi1}
mehr Parameter mit eingehen die Fehlerbehaftet sein können als im
Vergleich zur Formel \eqref{eqn:chi2}.
Die Abweichungen zu den Theoriewerten könnten
durch Verwendung von genauere Messgeräte verringert werden,.
Folglich sind beide Methode   geeignet um die Suzeptibilität
von paramagnetischen Stoffen zu bestimmen, wobei die
zweite Methode genauere Werte bei ungenauen Messgräten liefert.
