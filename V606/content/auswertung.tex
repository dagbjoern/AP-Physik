\section{Auswertung}
\label{sec:Auswertung}
\subsection{Selektivverstärker}
Um die Suszeptibilität $\chi$ zu bestimmen, muss die
Durchlassfrequenz $\nu_0$ des, der Brückenschaltung
vorgeschalteten, Selektivverstärkers
bestimmt werden.  Hierfür wird der gemessene Quotient von Ausgangspannung $U_A$
und Eingangspannung $U_E$ in Abhängigkeit von der Frequenz $\nu_0$
aufgetragen wie in der Abbilung \ref{fig:plot1} zu sehen.
Die zugehörigen Messwerte sind in der Tabelle \ref{tab:tab1} zu finden.
\begin{figure}
  \centering
  \includegraphics[width=0.7\textwidth]{a).pdf}
  \caption{Quotient von $U_A$ und $U_E$ in Abhängigkeit von $\nu$. }
  \label{fig:plot1}
\end{figure}

\begin{table}
  \centering
  \caption{Messwerte zur Bestimmung der Durchlassfrequenz $\nu_0$.}
  \label{tab:tab1}
  \begin{tabular}{c c}
    \toprule
    $\nu/\si{\kilo\hertz}$ & $U_A$/$U_E$ \\
    \midrule
    30    & 0,03\\
    31    & 0,04\\
    32    & 0,05\\
    33    & 0,075\\
    34    & 0,14\\
    34,5  & 0,23\\
    %34,7  & 0,31\\
    34,8  & 0,36\\
    34,9  & 0,455\\
    35    & 0,58\\
    35,05 & 0,655\\
    %35,06 & 0,68\\
    35,07 & 0,705\\
    35,1  & 0,77\\
    35,12 & 0,79\\
    35,14 & 0,85\\
    %35,15 & 0,865\\
    35,16 & 0,9\\
    35,18 & 0,93\\
    35,2  & 0,99\\
    35,22 & 1,04\\
    35,24 & 1,07\\
    35,26 & 1,11\\
    35,28 & 1,13\\
    35,30 & 1,14\\
    35,31 & 1,14\\
    35,32 & 1,13\\
    35,34 & 1,11\\
    35,36 & 1,08\\
    35,38 & 1,04\\
    35,40 & 1\\
    35,42 & 0,95\\
    35,44 & 0,89\\
    35,46 & 0,84\\
    35,48 & 0,79\\
    35,52 & 0,72\\
    35,57 & 0,62\\
    35,6  & 0,56\\
    35,65 & 0,5\\
    35,7  & 0,455\\
    35,8  & 0,37\\
    35,9  & 0,31\\
    36,5  & 0,155\\
    37,5  & 0,085\\
    38,5  & 0,06\\
    40    & 0,04\\
  \bottomrule
  \end{tabular}
\end{table}
\FloatBarrier
Für die Durchlassfrequenz $\nu_0$  des Selektivverstärkers
ergibt sich aus den Messwerten ein $\nu_0$ von ungefähr
\begin{align*}
  \nu_0=35,30\si{\kilo\hertz}.
\end{align*}
Um die Güte G des Selektivverstäkers zu bestimmen, wird noch $\nu_-$ und $\nu_+$ benötigt.
Diese befinden sich dort, wo der Quotient zwischen $U_A$ und $U_B$ nur noch  das $\frac{1}{\sqrt{2}}$-Fache des Maximum beträgt.
Somit ergibt aus den Messwerten für
\begin{align*}
  \nu_+\approx 35,48\si{\kilo\hertz}
\intertext{und}
 \nu_-\approx 35,12\si{\kilo\hertz}.
\end{align*}
Werden nun die Ergbenisse in die Formel \eqref{eqn:G} für die Güte eines Selektivverstärkers eingesetzt,
ergibt sich ein Güte von:
\begin{align*}
G\approx98,05.
\end{align*}
\begin{align}
G=\frac{\nu_0}{\nu_+-\nu_-} \label{eqn:G}
\end{align}
\subsection{Bestimmung der Suszeptibilität}
\subsubsection{Berechnung des theoretischen Wertes}
Um die theoretischen Werte für die Suszeptibilität $\chi$ der Materiale berechnen zu können,
wird der Landé-Faktor $g_\mathrm{J}$ mit Hilfe der Formel \eqref{eqn:lande}
berechent. Dabei werden Gesamtspin S, Gesamtbahndrehimpuls L und Gesamtdrehimpuls J
über die Hundschen Regeln bestimmt.
In der Tabelle \ref{tab:spin} sind die errechenten Gesamtspin,-bahndrehimpuls,-drehimpuls sowie der entsprechende Landé-Faktor
aufgelistet.
\begin{table}
  \centering
  \caption{ Gesamtspin,-bahndrehimpuls,-drehimpuls sowie der entsprechende Landé-Faktor.}
  \label{tab:spin}
  \begin{tabular}{c c c c c}
  Stoff & S & L & J & $g_J$\\
     \midrule
     $\mathrm{Nd_2O_3}$ & $\frac{3}{2}$  & 6 & $\frac{9}{2}$ & $\frac{8}{11}$\\
     \midrule
     $\mathrm{Gd_2O_3}$ & $\frac{7}{2}$   & 0 &$\frac{7}{2}$& $2$\\
     \midrule
     $\mathrm{Dy_2O_3}$ & $\frac{5}{2}$ & 5 &$\frac{15}{2}$& $\frac{4}{3}$\\
     \bottomrule
  \end{tabular}
\end{table}


Ebenfalls wird die Anzahl N der Momente pro Volumeneinheit der Stoffe benötigt.
Diese berechnet sich nach der Formel\eqref{eqn:N}, wobei $\rho$ die Dichte und M die molare Masse des Stoffes ist.
\begin{align}
  N=2\cdot N_\mathrm{A}\cdot\frac{\rho}{M}\label{eqn:N}
\end{align}
In der Tabelle \ref{tab:N} sind die berechenten Momente pro Volumeneinheiten
aufgelistet.

\begin{table}
  \centering
  \caption{Berechung der Momente pro Volumeneiheit.}
  \label{tab:N}
  \begin{tabular}{c c c c}
  Stoff &  $M/\si{\kilo\gram\per\mol}$ & $\rho/\si{\kilo\gram\per\meter\tothe{3}}$ & $N/\si{\mol\per\meter\tothe{3}}$\\
     \midrule
     $\mathrm{Nd_2O_3}$ & 0,33684  & 7240 &21494\\
     $\mathrm{Gd_2O_3}$ & 0,3625   & 7400 &20414\\
     $\mathrm{Dy_2O_3}$ & 0,372998 & 7800 &20911\\
     \bottomrule
  \end{tabular}
\end{table}

Mit Hilfe der Formel \eqref{eqn:XT} und den oben gelisteten Werten lässt sich nun die theoretische magnetische Suszeptibilität
der Stoffe bei Raumtemperatur $T\approx293\si{\kelvin}$ berechnen.
Somit ergibt sich für die Suszeptibilität\\
von $\mathrm{Nd_2O_3}$
\begin{align*}
\chi_\mathrm{Nd}=0,003 \  \ ,
\intertext{für $\mathrm{Gd_2O_3}$}
\chi_\mathrm{Gd}=0,014 \  \ ,
\intertext{und für $\mathrm{Dy_2O_3}$}
\chi_\mathrm{Dy}=0,025\  \ .
\end{align*}
\subsubsection{Experimentelle Bestimmung der Suszebtiblilität.}
Die Suszebtibilität lässt sich wie in dem Kapitel\ref{sec:test} beschrieben über zwei Methoden bestimmen. Bei beiden Methoden ist
erforderlich, dass der reale Querschnitt $Q_\mathrm{real}$ der Proben bestimmt wird.
Dieser berechnet sich aus der Formel \eqref{eqn:Qreal}:
\begin{align}
  Q_\mathrm{real}=\frac{M_\mathrm{p}}{L\rho_\mathrm{w}}\label{eqn:Qreal}.
\end{align}
Dabei ist L die Länge und $M_\mathrm{p}$ die Masse der Probe und
$\rho_\mathrm{w}$ die Dichte des Probenmaterials.
In der Tabelle \ref{tab:Q} sind die bestimmten Werte des realen Querschnittes aufgelistet.
\begin{table}
  \centering
  \caption{Berechung der realen Querschnittsfläche $Q_\mathrm{real}$ der Proben.}
  \label{tab:Q}
  \begin{tabular}{c c c c c}
  Stoff &  $M_\mathrm{p}/\si{\kilo\gram}$ & $\rho/\si{\kilo\gram\per\meter\tothe{3}}$ & Länge L/\si{\meter} & $Q_\mathrm{real}/\si{\meter\tothe{2}}\cdot 10^{-6}$\\
     \midrule
     $\mathrm{Nd_2O_3}$ & 0,0095  & 7240 & 0,16 & 8,2\\
     $\mathrm{Gd_2O_3}$ & 0,01408 & 7400 & 0,16 & 11,9\\
     $\mathrm{Dy_2O_3}$ & 0,0185  & 7800 & 0,16 & 14,8\\
     \bottomrule
  \end{tabular}
\end{table}
\FloatBarrier
Desweitern werden noch die Daten der Messspule und der Brückenschaltung benötigt sowie
die Spannung $U_\mathrm{sp}$ und die Frequnez $\nu$ der Spannungsquelle:

\begin{align*}
&\text{Daten der Messspule:}  \ \ n=250; \  \ F=86,6\si{\meter\tothe{2}}\cdot10^{-6}; \ \ l=0,135\si{\meter}; \ \ R= 0,7\si{\ohm}, \\
&\text{Daten der Brückenschaltung:} \ \  R_3=998\si{\ohm}.\
\end{align*}
Die Spannungsquelle besitzt eine Spannung von
\begin{align*}
U_\mathrm{sp}=1\si{\volt}
\end{align*}
diese wird durch Verstärker um den Faktor $100$ verstärkt
und eine Frequenz $\nu$ die ungefähr der Durchlassfequenz $\nu_0$
des Selektivverstärkers entspricht:
\begin{align*}
\nu\approx35,30\si{\kilo\hertz}.
\end{align*}
Nun lässt sich mit Hilfe der Messdaten, die in der Tabelle \ref{tab:mess}
aufgelistet sind, und den vorherigen Daten
die Suszebtilbilität \chi sowohl durch die Formel \eqref{eqn:chi1}
als auch  durch die Formel \eqref{eqn:chi2} berechnen.
Die berechneten Werte sind in der Tabelle \ref{tab:X1X2}
zu finden.

%\begin{align}
%\chi&=\frac{U_\mathrm{Br}}{U_\mathrm{Sp}}\frac{4l}{\omega\mu_0n^2Q}\sqrt{R^2 + \omega^2\left(\mu_0\frac{n^2}{l}F\right)^2}\label{eqn:chi1}\\
%\chi&=2\frac{\Delta R}{R_3}\frac{F}{Q}\label{eqn:chi2}
%\end{align}
\begin{table}
  \centering
  \caption{Messwerte der Messung.}
  \label{tab:mess}
  \begin{tabular}{c c c c c c c}
    &   vorher & nachher & vorher & nachher & \\
  Stoff &  $U_\mathrm{br}/\si{\volt}\cdot 10^{-3}$ & $U_\mathrm{br}/\si{\volt}\cdot 10^{-3}$  & $R/\si{\ohm}\cdot10^{-3}$  & $R/\si{\ohm}\cdot10^{-3}$  & $\Delta R/\si{\ohm}\cdot10^{-3}$ & $\Delta U/\si{\volt}\cdot10^{-3}$ \\
   \midrule
  $\mathrm{Nd_2O_3}$ & 1,4  &  3,05 & 2,885 & 2,750 & 0,135 & 1,65  \\
                     & 1,4  &  4,6  & 3,000 & 2,770 & 0,230 & 3,20 \\
                     & 1,4  &  3,6  & 2,900 & 2,695 & 0,205 & 2,20 \\
   \midrule
   $\mathrm{Gd_2O_3}$ & 1,3  & 15,5  & 3,08 & 2,100 & 0,980 & 14,20 \\
                      & 1,3  & 13,75 & 2,79 & 1,955 & 0,835 & 12,45 \\
                      & 1,25 & 13,5  & 2,79 & 1,955 & 0,795 & 12,25 \\
   \midrule
  $\mathrm{Dy_2O_3}$ & 1,4 & 31,0 & 2,895 & 1,115 & 1,780 & 29,6 \\
                     & 1,3 & 29,0 & 2,955 & 1,215 & 1,740 & 27,7 \\
                     & 1,3 & 32,5 & 2,520 & 1,115 & 1,405 & 31,2 \\
   \bottomrule
  \end{tabular}
\end{table}
\FloatBarrier
\begin{table}
  \centering
  \caption{Die berechneten Suszebtibilitäten aus den Messwerten durch die Formeln \ref{eqn:chi1} und \ref{eqn:chi2}.}
  \label{tab:X1X2}
  \begin{tabular}{c c c}
          &     $\chi_\mathrm{1}$  & $\chi_\mathrm{2}$\\
    Stoff &   mit Formel \eqref{eqn:chi1} &mit Formel \eqref{eqn:chi2}\\
       \midrule
       $\mathrm{Nd_2O_3}$ &0,0007 & 0,003 \\
                          &0,0013 & 0,005 \\
                          &0,0009 & 0,004 \\
       \midrule
       $\mathrm{Gd_2O_3}$ &0,0041 & 0,014  \\
                          &0,0036 & 0,012 \\
                          &0,0036 & 0,012 \\
        \midrule
        $\mathrm{Dy_2O_3}$&0,0069 & 0,021 \\
                          &0,0064 & 0,020 \\
                          &0,0073 & 0,016 \\
      \bottomrule
    \end{tabular}
  \end{table}
\FloatBarrier
In der Tabelle \ref{tab:alle} sind
jeweils alle Mittelwerte
der vorher berechneten Suszebtibilitäten
und deren Abweichungen
von den Theoriewerten aufgelistet.
\FloatBarrier
\begin{table}
  \centering
  \caption{Alle berechneten Suszebtibilitäten auf einen Blick.}
  \label{tab:alle}
  \begin{tabular}{c c c c c c}
          &  Theoretisches &   $\overline{\chi}_\mathrm{1}$  & $\overline{\chi}_\mathrm{2}$ & Abweichung & Abweichung\\
    Stoff &  $\chi_\mathrm{theo}$ &mit Formel \eqref{eqn:chi1} &mit Formel \eqref{eqn:chi2}& $a_1$ & $a_2$\\
       \midrule
       $\mathrm{Nd_2O_3}$ &0,003 & 0,0099\pm0,0003&0,0040\pm0,0008 &0,67 &0,33 \\
       $\mathrm{Gd_2O_3}$ &0,014 & 0,0038\pm0,0003&0,013\pm0,001   &0,39 &0,04 \\
       $\mathrm{Dy_2O_3}$ &0,025 & 0,0069\pm0,0003&0,019\pm0,002   &0,72 &0,24 \\
      \bottomrule
    \end{tabular}
  \end{table}
