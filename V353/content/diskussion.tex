\section{Diskussion}
\label{sec:Diskussion}
Nun werden die aus der Auswertung erhaltenen
Ergebnisse in Hinblick auf die Theorie untersucht.
Zu einem muss gesagt werden, dass die Messwerte aus der Abbildung \ref{abb:c} stark
um die Ausgleichsfunktion streuen, dies liegt daran das die Zeitabstände a und b mit
Hilfe der Cursers des Oszilloskopes gemesssen wurde und diese nicht genau eingestellt werden konnten.
Dies ließe sich mit einem Oszilloskop, das die Daten speichert und ausgibt verhmeiden.
Ebenfalls würde mit Erhöhung der Messpaare im niedrigerem Frequenzbereich die Zeitkonstant $RC$ genauer werden .
Außerdem wurde eine Frequenzabhängigkeit bei der Spannung $U_0$ gemessen, diese sank bei höheren Frequenzen ab.
Diese ist aber für den Frequenzbreich, indem die Messungen durchgeführt
worden sind, sehr gering und kann somit vernachlässigt werden.
Bei dem Vergleich der Zeitkonnstanten,
\begin{align*}
  RC_\mathrm{1}=(1,24\pm0,02)\,\si{\milli\second},\\
  RC_\mathrm{2}=9,36 \pm 0,0007\si{\milli\second},\\
  RC_\mathrm{3}=8,685\pm0,0002\,\si{\milli\second},
\end{align*}
die mit unterschiedlichen Verfahren gemessenen wurden, fällt auf, dass
 $RC_\mathrm{1}$ deutlich von $RC_\mathrm{2}$ und $RC_\mathrm{3}$ abweicht. Durch diesen großen Unterschied
 kann vermutet werden, dass entweder bei der Messung 1  oder bei den Messung 2 und 3 ein Fehler gemacht wurde, da 2 und 3 die selben Einstellung
 benötigen kann nicht sofort davon ausgegangen werden, dass die Messung 1 Fehlerbehaftet ist.
 Folglich ist es nicht möglich die Formel  aus dem Kapitel \ref{sec:Theorie} mit diesen Messergebnissen zu verifizieren,
