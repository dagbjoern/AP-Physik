\section{Diskussion}
\label{sec:Diskussion}
Nun werden die aus der Auswertung erhaltenen
Ergebnisse in Hinblick auf die Theorie untersucht.
Zu einem muss gesagt werden das die Messwerte aus der Abbildung \ref{abb:c} stark
um die Ausgleichsfunktion streuen, dies liegt daran das die Zeitabstände a und b mit
Hilfe der Cursers des Oszilloskopes gemesssen wurde und diese nicht genau eingestellt werden konnten.
Dies ließe sich mit einem genaueren Oszilloskop mindern.
Ebenfalls würde mit Erhöhung der Messpaare im niedrigerem Frequenzbereich die Zeitkonstant $RC$ genauer werden .
Außerdem wurde eine Frequenzabhängigkeit bei der Spannung $U_0$ gemessen, diese sank bei höheren Frequenzen ab.
Diese ist aber für den Frequenzbreich, indem die Messungen durchgeführt
worden sind, sehr gering und kann somit vernachlässigt werden.
Die bei den anderen Messverfahren erhaltenen Ergebnisse bestätigen
die aus dem Kapitel \ref{sec:Theorie} aufgeführten Formeln. 
