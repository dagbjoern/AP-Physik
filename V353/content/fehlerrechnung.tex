\section{Fehlerrechnung}
 \label{fehlerrechnung}
Bei dem Versuch entstehen Messfehler,
diese gilt es in Folgenden zu erörtern.
Bei der Messung der Frequenz und der Spannung bei jeder Messreihe
wurde diese mit Hilfe des Ozilloskopes oder des Generators gemessen,
Dadurch kommt es zu eine Fehler, der in der Auswertung abgeschätzt wird.
\\
Der Fehler einzelnen gemessener Größen, die aus der Theorie berechnet werden, wird mit der Gauß´schen Fehlerfortpflanzung
berechnenet \eqref{eqn:gaus}.
\begin{equation}
\Delta f= \sqrt{\left(\frac{\partial f}{\partial x}\Delta x \right)^{2} + \left( \frac{\partial f}{\partial y}\Delta y\right)^2...}\label{eqn:gaus}
\end{equation}
\\
Dies übernimmt das Programm Python mit Hilfe uncertainties.
