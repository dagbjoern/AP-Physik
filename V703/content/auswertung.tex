\section{Auswertung}
\label{sec:Auswertung}
In dem Versuch wurde Thallium 204 als radioaktive Quelle verwendent.
In der Tabelle \ref{tab:mess} sind die Messwerte für
die Zählrohr-Charakteristik
aufgelistet. Der gewählte Fehler für die Zählrate $N$
beträgt:
\begin{align*}
  \Delta=\frac{\sqrt{Impulszahl}}{\Delta t}.
\end{align*}

\begin{table}
  \centering
  \caption{Messwerte für die Zählrohr-Charakteristik.}
  \label{tab:mess}
  \begin{tabular}{c c}
    \toprule
Spannung & Zählrate\\
$U$ in $\si{\volt}$ & $N$ in $\si{\per\second}$ \\
    \midrule
    330  & 274,1\pm5,2    \\
    350  & 284,3\pm5,3    \\
    325  & 278,7\pm5,3    \\
    375  & 291,3\pm5,4    \\
    395  & 295,3\pm5,4    \\
    400  & 291,6\pm5,4    \\
    430  & 297,6\pm5,5    \\
    450  & 295,4\pm5,4    \\
    470  & 294,9\pm5,4    \\
    500  & 299,3\pm5,5    \\
    530  & 298,9\pm5,5    \\
    550  & 302,7\pm5,5    \\
    570  & 310,8\pm5,6    \\
    590  & 304,4\pm5,5    \\
    600  & 303,2\pm5,5    \\
    630  & 305,0\pm5,5    \\
    650  & 315,5\pm5,6    \\
    670  & 308,7\pm5,6    \\
    690  & 317,6\pm5,6    \\
    695  & 319,7\pm5,7    \\
    \bottomrule
  \end{tabular}
\end{table}
\FloatBarrier
Die Zählrate $N$ wird, wie in Abbildung \ref{fig:plot1} zu sehen,
in Abhängigkeit von der Spannung $U$ aufgetragen.
\begin{figure}
  \centering
  \includegraphics[width=0.7\textwidth]{a).pdf}
  \caption{Die Zählrate $N$ pro sekunde in Abhängigkeit von der Spannung $U$.}
  \label{fig:plot1}
\end{figure}
\FloatBarrier
Aus der Abbildung lässt sich nun die Lage des Plateau-Bereiches
abschätzen. Dieser beginnt ungefähr bei  $375\si{\volt}$ und
endet ungefähr bei $670\si{\volt}$ somit beträgt die Länge
des Plateaus $295\si{\volt}$.
Aus den Messwerten innerhalb der Plateaus wird nun mit Hilfe
einer lineraren Regression eine Gerade bestimmt.
Es ergeben sich folgende Werte für die Parameter der Geraden.
\begin{align*}
A=&(267\pm5)\si{\per\second}\\
B=&(0,065\pm0,009)\si{\per\volt\second}
\intertext{Die Plateau-Steigung $\epsilon$ ergibt sich über die Formel \eqref{eqn:relf}}:
\end{align*}
\begin{align}
 \epsilon=\frac{\mathrm{Steigung}\cdot100}{\mathrm{Zählrate \ bei} \ 500\si{\volt}} \label{eqn:relf}.
\end{align}
\begin{align*}
\intertext{Sie beträgt:}
\epsilon=(2,2\pm0,3)\%.
\end{align*}

Der durch das Ozilloskop gemessene Abstand, der mit einer
Genauigkeit von $\pm20$ vom Ozilloskop abgelesen wird, zwischen Primär- und
Nachentladungsimpulsen beträgt
\begin{align*}
\text{bei einer Spannung von $350\si{\volt}$} \ &(120\pm20)\si{\micro\second}\\
\text{und bei einer Spannung von $700\si{\volt}$} \ &(105\pm20)\si{\micro\second}.
\end{align*}
Die ebenfalls mit Hilfe des Oszilloskop bestimmte Totzeit beträgt
bei einer Betriebspannng von $700\si{\volt}$:
\begin{align*}
  T=(80\pm20)\si{\micro\second}.
\end{align*}
Bei der Zwei-Quelllen-Methode ergibt sich aus den gemessenen
Zählraten,
\begin{align*}
  N_1&= (195,2\pm4,4)\si{\per\second},\\
  N_{1+2}&=(417,9\pm6,4)\si{\per\second},\\
  N_2&=(236,0\pm4,9)\si{\per\second}
\end{align*}
und der folgenden Formel  \eqref{eqn:totzeit}
\begin{align}
T\approx\frac{N_1+N_2-N_{1+2}}{2N_1N_2}\label{eqn:totzeit}.\\
\end{align}
\begin{align*}
\intertext{eine Totzeit von:}\\
T=(140\pm100)\si{\micro\second}.
\end{align*}
Desweiteren sollen die pro einfallendem Teilchen im Zählrohr freigesetzten
Ladungen in Abbhängigkeit von der Zählrohrspannung untersucht werden.
Dabei berechnen sich die freigesetzten Ladungen nach der Formel \eqref{eqn:Q}
\begin{align}
  \Delta Q=\frac{\overline{I}}{N}.\label{eqn:Q}
\end{align}
In der Tabelle \ref{tab:I} sind die berechneten Ladungen $\Delta Q$ in Abhängigkeit
von der Zählrohrspannung $U$ und des gemessenen Stroms aufgelistet.
Als Fehler für den gemessenen Strom $I$ werden
$0,2\si{\micro\ampere}$ abgeschätzt.
\begin{table}
  \centering
  \caption{Messwerte zur Bestimmung der freigesetzten Ladungen.}
  \label{tab:I}
  \begin{tabular}{c c c c}
    \toprule
Spannung & Zählrate  & Strom & Ladungen \\
$U$ in $\si{\volt}$ & $N$ in $\si{\per\second}$ & $I$ in $\si{\micro\ampere}$ & $Q$ in $10^{9}\si{\elementarycharge}$\\
    \midrule
    330  & 274,1\pm16,6  &  0,1\pm0,2 &  2,2\pm0,5 \\
    350  & 284,3\pm16,9  &  0,2\pm0,2 &  4,4\pm0,4 \\
    325  & 278,7\pm16,7  &  0,1\pm0,2 &  2,2\pm0,4 \\
    375  & 291,3\pm17,1  &  0,2\pm0,2 &  4,2\pm0,4 \\
    395  & 295,3\pm17,2  &  0,4\pm0,2 &  8,5\pm0,4 \\
    400  & 291,6\pm17,1  &  0,4\pm0,2 &  8,6\pm0,4 \\
    430  & 297,6\pm17,3  &  0,5\pm0,2 & 10,5\pm0,4 \\
    450  & 295,4\pm17,2  &  0,6\pm0,2 & 12,7\pm0,4 \\
    470  & 294,9\pm17,2  &  0,7\pm0,2 & 14,8\pm0,4 \\
    500  & 299,3\pm17,3  &  0,8\pm0,2 & 16,7\pm0,4 \\
    530  & 298,9\pm17,3  &  0,9\pm0,2 & 18,8\pm0,4 \\
    550  & 302,7\pm17,4  &  1,0\pm0,2 & 20,6\pm0,4 \\
    570  & 310,8\pm17,6  &  1,1\pm0,2 & 22,1\pm0,4 \\
    590  & 304,4\pm17,4  &  1,1\pm0,2 & 22,6\pm0,4 \\
    600  & 303,2\pm17,4  &  1,2\pm0,2 & 24,7\pm0,4 \\
    630  & 305,0\pm17,5  &  1,3\pm0,2 & 26,6\pm0,4 \\
    650  & 315,5\pm17,8  &  1,4\pm0,2 & 27,7\pm0,4 \\
    670  & 308,7\pm17,6  &  1,4\pm0,2 & 28,3\pm0,4 \\
    690  & 317,6\pm17,8  &  1,6\pm0,2 & 31,4\pm0,4 \\
    695  & 319,7\pm17,9  &  1,7\pm0,2 & 33,2\pm0,4 \\
    \bottomrule
  \end{tabular}
\end{table}
\FloatBarrier
In der Abbildung \ref{fig:IU} ist die gemessen Ladung $Q$ aus der Tabelle \ref{tab:I} in Ahbängigkeit von der eingestellten Zählrohrspannung  $U$
abgebildet.


\begin{figure}
  \centering
  \includegraphics[width=0.7\textwidth]{plotIU.pdf}
  \caption{Die Ladung $Q$ in Abhängigkeit von der Zählrohrspannung $U$ von den Messwerten aus der Tabelle \ref{tab:I}.}
  \label{fig:IU}
\end{figure}
