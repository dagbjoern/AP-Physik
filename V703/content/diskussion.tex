\section{Diskussion}
\label{sec:Diskussion}
Die bestimmte Gerade, die durch die Messwerte für die Charakterisik
des Zählrohres gefittet wurde, verläuft durch alle Fehlerbalken somit
kann gesagt werde, dass die Gerade im Rahmen der Messungenauhigkeit liegt.
Desweiteren kann über die Formel \eqref{eqn:relf}
\begin{align}
 \epsilon=\frac{\mathrm{Steigung}\cdot100}{\mathrm{Zählrate \ bei} \ 500\si{\volt}} \label{eqn:relf}
\end{align}
der relative statistische Fehler jedes Messpunktes bestimmt werden.
Bei dieser Messung sollte dieser unter $1\%$  liegen, jedoch
ergibt sich aus den Messwerten
ein $\epsilon$ von:
\begin{align*}
\epsilon=2,2\%.
\end{align*}
Dies könnte der maximale Impulsrate geschuldet
sein, da diese nicht wie vorgeschrieben
nicht wesentlich über $100/\si{\per\second}$ steigen
sollte, jedoch im Veruch teilweise sogar über $300/\si{\per\second}$
lag.
Bei der Messung zur Bestimmung des Abstandes zwischen Primär- und Nachentladungsimpulsen
wird deutlich, dass dieser sich bei höherer Zählrohrspannung verringert.
Die Bestimmung der Totzeit liefert bei den unterschiedlichen
Messwerfahren Ergebnisse die nicht stark von einander abweichen, somit
liefern beide Varianten Ergebnisse die im Rahmen liegen.
Die Messung, der im Zählrohres freigesetzten Ladungen,
zeigt, dass diese mit Erhöhung der Zählrohrspannung $U$
ebenfalls steigt. Es lässt sich ein linearer Zusammmenhang
vermuten.
