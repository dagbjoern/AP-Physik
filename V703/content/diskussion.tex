\section{Diskussion}
\label{sec:Diskussion}
Die bestimmte Gerade, die durch die Messwerte für die Charakterisik
des Zählrohres gefittet wurde, verläuft durch alle Fehlerbalken somit
kann gesagt werde, dass die Gerade im Rahmen der Messungenauhigkeit liegt.
Kleine Fehler könnten der maximale Impulsrate geschuldet
sein, da diese nicht wie vorgeschrieben
nicht wesentlich über $100\si{\per\second}$ steigen
sollte, jedoch im Veruch teilweise sogar über $300\si{\per\second}$
lag.
Bei der Messung zur Bestimmung des Abstandes zwischen Primär- und Nachentladungsimpulsen
wird deutlich, dass dieser sich bei höherer Zählrohrspannung verringert.
Die Bestimmung der Totzeit $T$ liefert bei den unterschiedlichen
Messverfahren folgende Ergebnisse:
\begin{align*}
  \intertext{über das Oszilloskop:}
  T=(80\pm20)\si{\micro\second},
  \intertext{über die Zwei-Quellen-Methode:}
  T=(140\pm100)\si{\micro\second}.
\end{align*}
Dabei fällt auf, dass der Fehler bei der Ozilloskopmessung deutlich kleiner ist
als bei der Zwei-Quellen-Methode, somit
ist die Oszllposkopmessung geeigneter zur Bestimmung der Totzeit $T$.
Die Messung, der im Zählrohres freigesetzten Ladungen,
zeigt, dass diese mit Erhöhung der Zählrohrspannung $U$
ebenfalls steigt. Aus der Abbildung \ref{fig:IU} lässt sich ein linearer Zusammmenhang
zwischen der Zählrohrspannung $U$ und den freigesetzten Ladungen vermuten.
