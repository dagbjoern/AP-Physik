\section{Theorie}
\label{sec:theorie}

\cite{sample}
Bewegt sich ein Körper in einer Flüssigkeit, so wirken auf
ihn die stokesche Reibungskraft,
\begin{align}
  \vec{F}_{\mathrm{R}}=6\pi\eta v r,
\end{align}
mit $\eta$ als dynamische Viskosität und $v$ der Kugelgeschwindigkeit,
die Schwerkraft $\vec{F}_{\mathrm{G}}$ und die Auftriebskraft $\vec{F}_{\mathrm{A}}$.
Die Reibungskraft wirkt entgegen der Bewegungsrichtung, somit stellt sich ein
Kräftegleichgewicht ein und der Körper, in diesem Fall eine Kugel,
fällt dann mit konstanter Geschwindigkeit.
Die Fallvorrichtung ist so ausgelegt, dass beim Eintauchen
keine Verwirbelungen entstehen. Eine wirbelfreie Strömung
wird als laminar bezeichnet, ein Maß hierfür ist die Reynoldsche Zahl, berechnet nach
\begin{align}
Re=\frac{\rho \cdot v \cdot d}{\eta}\label{eqn:rey}\text{\cite{rey}}.
  %Re=\frac{\rho v d}{\eta}\label{eqn:rey}\cite{rey}.
\end{align}
Die Formel für die Reynoldsche Zahl bezieht sich auf Rohrströmungen.
$\rho$ ist die Dichte der Flüssigkeit, $\eta$ die dynamische Viskosität der
Flüssigkeit, $v$ die Strömungsgeschwindigkeit der Flüssigkeit
gegenüber des Körpers und $d$ der Durchmesser des Rohrs.
Die kritische Reynoldzahl $Re_{\mathrm{krit}}=2040$ gibt
den Übergang zu turbulenten Strömungen an.\cite{reykrit}
Die dynamische Viskosität $\eta$ lässt sich
aus der Fallzeit $t$, Flüssigkeitsdichte $\rho_{FL}$ und Kugeldichte $\rho_{PK}$
bestimmen. $K$ ist dabei eine  Apparaturkonstante, die Fallhöhe
und und Kugelgeometrie enthält, für $\eta$ gilt:
\begin{align}
  \eta=K(\rho{K}-\rho{FL})\cdot t \label{eqn:vis}
\end{align}
Die dynamische Viskösitat ist ebenfalls Temperaturabhängig somit gilt ebenfalls die Beziehung:
\begin{align}
  \eta(T)= A e^{\frac{B}{T}}\label{eqn:andra},
\end{align}
$A$ und $B$ sind hierbei Konstanten.
