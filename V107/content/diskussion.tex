\section{Diskussion}
\label{sec:Diskussion}
Das Experiment zeigt, dass die Viskosität von Wasser sich bei
Erhöhung der Temperatur verringert und sich sowohl durch
die Formel \eqref{eqn:vis}, als auch durch die Formel
\eqref{eqn:andra} berechnen lässt.
Außerdem liegt die berechnete Ausgleichsgerade im Bereich der angenommenen Messfehler, dies bestätigen ebenfalls die Geleichungen.
Jedoch weicht die durch den Versuch bestimmte Viskosität bei ungefähr Raumtempertaur (20°)
\begin{align*}
\eta_{\mathrm{exp}}&=1,26\,\si{\pascal\second}
\intertext{von dem Literaturwert}
\eta_{\mathrm{lit}}&= 1 \,\si{\pascal\second}
\end{align*}
um 26\% ab. Diese relative Abweichung lässt sich durch systematrische Fehler bei
der Messung begründen. Zum Beispiel werden die Zeiten bei dem Versuch nur mit einer Stoppuhr gemessen, die per Hand gestoppt wird.
Ebenfalls wird die Apparatekonstante nur aus einer Anderen, im Skript gegebenen, berechnet und nicht expliziet für die große Kugel bestimmt.
Desweiteren wird die Temperatur des destillierten Wassers nicht explizit in der Röhre gemessen, sondern nur an einem Thermostat abgelesen. Dies führt zur einer ungenauen Temperaturbestimmung.
Trotzdem kann gesagt werden, dass der Versuch, wenn die systematrischen Fehler minimiert werden, sich zur Viskositätbestimmung eignet.
\newpage
