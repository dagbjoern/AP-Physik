\section{Diskussion}
\label{sec:Diskussion}
Durch die Auswertung der ersten Messung lässt sich die
Formel \eqref{eqn:theo} nicht klar verifizieren. Zwar ist ein lineare
Zusammenhang zwischen den Messwerten zu erkennen,
dieser unterscheidet sich
jedoch von der Formel \eqref{eqn:theo}. Diese Abweichung
könnte durch die Frequenz der Ultraschalllquelle entstehen, da diese
nur angegeben war und
nicht zu Kontrolle gemessen wurde.
Die Ergebnisse der zweiten Messung zeigen, dass sich zur Zeit der
Messung in dem Rohr keine laminare Strömung gebildet hat, da
bei solch einer Strömung die Strömungsgeschwindigkeit erst bei
steigender Eindringtiefe bis zu einem Maximum, das in der Mitte des Rohres liegt,
zu nimmt und ab da wieder anfängt zu sinken.
Da Dies bei der Abbilung \ref{fig:v2} nicht beobachtet werden kann, liegt bei
der Messung ein nicht wirbelfreie Strömung vor und somit kann ebenfalls
keine Aussage über die gemessene Streuintensität gemacht werden.
