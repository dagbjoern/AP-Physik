\section{Theorie}
\label{sec:Theorie}
Metalle haben die Eigenschaft leitfähig zu sein, dies liegt an den
fast frei beweglichen Elektronen, den Leitelektronen, die aus der äußeren Schale abgespalten werden.
In Festkörpern bilden die Valenzelektronen der Atome eine gemeinsames System und damit spalten sich
die zuvor klar definierten Energienevaus in sogenannte Energiebänder auf.
Bei den Energiebändern kommt es zu Überlappungen, aber auch zu endlich breiten Lücken zwischen zwei Bändern.
Die Elektronen des Äußeren nicht vollständig besetzten Bandes können beim anliegen eines elektrischen Feldes Energie aufnehmen
und sich damit in Richtung des Feldes bewegen, somit entsteht ein Strom. Die Leitfähigkeit von Metallen wird von vielen verschiedenen Parametern bestimmt.

\subsection{Parameter in Abhängigkeit vom Widerstand eines stromdurchflossenen Leiters}
Die Leiterelektronen führen in einem realen Kristall auf Grund von Fehlstellen oder Defekten in der Struktur ständig
Zusammenstöße aus. $\tau$ ist dabei das gemittelte Zeitintervall zwischen zwei Zusammenstößen. Liegt ein elektrisches Feld an
einer Probe an, ergibt sich eine Geschwindigkeitsängerung in Richtung des elektrischen Feldes in der Zeit $\tau$:
\begin{align}
\Delta v=-\frac{e_0}{m_0}\vec{E}\tau
\end{align}
Bei einem Zusammenstoß werden die Leiterelektronen in zufällige Richtungen geleitet, eine mittlere Driftgeschwindigkeit wird daher eingeführt.
Für ein Elektron gilt dann:
\begin{align}
\vec{v_\mathrm{d}}=\frac{1}{2}\Delta v.
\end{align}
Die Stromdichte eines Leiters mit n Elektronen pro Volumeneinheit ergibt sich zu:
\begin{align}
j=\frac{1}{2}\frac{e_0^2}{\m_0 n} \tau E.
\end{align}
Durch das Ersetzten von $j$ zu $\frac{I}{Q}$ und $E$ durch $\frac{U}{L}$ egibt sich das Ohmsche Gesetzt in folgender Form:
\begin{align}
I=\frac{1}{2}\frac{e_0^2}{m_0}n\tau \frac{Q}{L}U.
\end{align}
$Q$ ist der Queerschnitt des Leiters und $L$ die Länge.
Die reziproke Leitfähigkeit ist der Widerstand $R$, dieser ergibt sich zu:
\begin{align}
R=2\frac{\m_o}{e_0^2}\frac{1}{n\tau}\frac{L}{Q}.
\end{align}
Mit dem leicht messbaren Widerstand kann auf alle größen bis auf $n$ und $\tau$ geschlossen werden.

\subsection{Parameter in Abhängigkeit vom Hall-Effekt}
Der Hall-Effekt beschreibt im wesentlichen eine auftretende Spannung in einem stromdurchflossenen Leiter, welcher sich in
einem homogenen Magnetfeld befindet. Die Spannung entsteht durch die Lorentz-Kraft.
Mit weiteren Überlegungen ergibt sich die Hall-Spannung:
\begin{align}
 U_\mathrm{H}=-\frac{1}{ne_0}\frac{B\cdot I_\mathrm{q}}{d}.
\end{align}
Die Größe $B$ bezeichnet die magnetische Flussdichte, $d$ die Dicke des Leiters und $I_\mathrm{q}$ den Queerstrom.
Aus de Hall-Spannung lassen sich nun auch $n$ und $\tau$ berechen.

\subsection{Weitere Parameter}
Von Interesse für die Leitfähigkeit ist ebenfalls die mittlere freie Weglänge $\ell$, dies ist die Entfernung eines Elektrons die
es beim Zusammenstoß zurücklegt im Mittel. Definiert ist diese durch $\ell=\tau\cdot|v|$ mit $|v|$ der Totalgeachwindigkeit.
Unter Beachtung des Pauli-Verbotes und der Fermi-Energie gilt für $|v|$:
\begin{align}
|v|\approx \sqrt{\frac{2E_\mathrm{F}}{m_0}}
\end{align}
und damit für die mittlere freie Weglänge:
\begin{align}
\ell \approx \tau\sqrt{\frac{2 E_\mathrm{F}}{m_0}}.
\end{align}
Der letzte Parameter ist die Beweglichkeit $\mu$ der Ladungsträger, diese stellt den Proportionalitätsfaktor
zwischen der Driftgeschwindigkeit und der elektrischen Feldstärke:
\begin{align}
\vec{v_\mathrm{d}}=\mu \vec{E}.
\end{align}
