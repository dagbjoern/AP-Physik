\newpage
\section{Diskussion}
\label{sec:Diskussion}
Die Tablle \ref{tab:vergleich} enthält nochmal alle bestimmten Brennweiten $f$
die in der Auswertung über die unterschiedlichen Methoden bestimmt worden sind.
\begin{table}
  \centering
  \caption{Theoriewerte und Messergebnisse der Brennweiten $f$ für die unterschiedlichen Messverfahren und deren Abweichungen von der Theorie.}
   \label{tab:vergleich}
  \begin{tabular}{c c c c}
  \toprule
  Messung  & Theoretischer Wert  & Gemessener Wert &   Abweichung von der Theorie    \\%\multicolumn{2}{c}{Verhältniss}\\
   & $f/\si{\milli\meter}$ & $f/\si{\milli\meter}$ & $a/\si{\percent}$ \\
   \midrule
  Linsengleichung 1 &100 &  95,8\pm3,4 & 4,2\\
  Linsengleichung 2 &150 &  163,2\pm0,6& 8,8\\
  Bessel            &150 &  160\pm17   & 6,7\\
  Abbe              & ?? &  187\pm5 & \\
  \bottomrule
 \end{tabular}
\end{table}
Es fällt auf das alle Abweichungen zu den Theoretischengrößen für eine Linse im Rahmen liegen.
Somit eingnet sich sowohl zur Bestimmung der Brennweite $f$  die Methode
über die Linsengleichung als auch die Methode von Bessel.
Für ein Linsensystem mit zwei ??.

Desweiteren lässt sich aus der Tabelle \ref{tab:1} sagen, dass sich die Verhältniss
$G/B$ und $g/b$ ungefähr gleich verhalten. Somit lässt sich der Zusammenhang
\begin{align*}
  \frac{G}{B}=\frac{g}{b}
\end{align*}
vermuten.
Aus den Abbildungen \ref{fig:plot1} und \ref{fig:plot2} kann die Genauigkeit der Messungen aus \ref{sec:1} Berurteilt werden.
In der Abblidung \ref{fig:plot1} treffen nicht alle Linien sich in einem Punkt
somit ist die Messung nicht so genau wie die aus \ref{fig:plot2}
wo sich alle Linien in einem Punkt kreuzen. ??

Bei den Brennweiten der Linse bei einem roten und blauen Filter
\begin{align*}
  f_\mathrm{rot}=(161,81\pm1,84)\si{\milli\meter}
  f_\mathrm{blau}=(161,83\pm1,45)\si{\milli\meter}
\end{align*}
kann keine genau Aussage getätigt werden welches Licht stärker gebochen wird, da
die Wesswerte zu einander bei einander liegen.
??
