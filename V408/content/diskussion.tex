\newpage
\section{Diskussion}
\label{sec:Diskussion}
Die Tablle \ref{tab:vergleich} enthält alle bestimmten Brennweiten $f$
die in der Auswertung über die unterschiedlichen Methoden bestimmt worden sind.
\begin{table}
  \centering
  \caption{Theoriewerte und Messergebnisse der Brennweiten $f$ für die unterschiedlichen Messverfahren und deren Abweichungen von der Theorie.}
   \label{tab:vergleich}
  \begin{tabular}{c c c c}
  \toprule
  Messung  & Theoretischer Wert  & Gemessener Wert &   Abweichung von der Theorie    \\%\multicolumn{2}{c}{Verhältniss}\\
   & $f/\si{\milli\meter}$ & $f/\si{\milli\meter}$ & $a/\si{\percent}$ \\
   \midrule
  Linsengleichung 1 &100 &  95,8\pm3,4 & 4,2\\
  Linsengleichung 2 &150 &  163,2\pm0,6& 8,8\\
  Bessel            &150 &  160\pm17   & 6,7\\
  Abbe              &164 &  187\pm5    & 14,0 \\
  \bottomrule
 \end{tabular}
\end{table}
Es fällt auf, dass alle Abweichungen zu den theoretischen Größen für eine Linse im Rahmen liegen.
Somit eingnen sich zur Bestimmung der Brennweite $f$ sowohl die Methode
über die Linsengleichung, als auch die Methode von Bessel.
Bei der Methode von Abbe, für eine Linsensystem, weicht der gemssene Wert am stärksten von dem Theoriewert
ab. Die Ursache dieser Ungenaugikeit liegt darin, dass es nicht immer möglich war
klar zu bestimmen, wann das Bild scharf gestellt war.\\
\\
Des Weiteren lässt sich aus Tabelle \ref{tab:1} folgern, dass sich die Verhältnisse
$G/B$ und $g/b$ ungefähr gleich verhalten. Somit lässt sich der Zusammenhang
\begin{align*}
  \frac{G}{B}=\frac{g}{b}
\end{align*}
verifiziere.
Aus den Abbildungen \ref{fig:1} und \ref{fig:2} kann die
Genauigkeit der Messungen aus \ref{sec:1} berurteilt werden.
In der Abblidung \ref{fig:1} treffen sich nicht alle Linien in einem Punkt
somit ist die Messung nicht so genau wie die aus \ref{fig:2},
in der sich alle Linien in einem Punkt kreuzen.

Bei den Brennweiten der Linse mit einem roten und blauen Filter
\begin{align*}
  f_\mathrm{rot}=(161,8\pm0,6)\si{\milli\meter}\\
  f_\mathrm{blau}=(161,8\pm0,5)\si{\milli\meter}
\end{align*}
kann keine genau Aussage getroffen werden welches Licht stärker gebochen wird, da
die signifikanten Stellen der Messwerte übereinstimmen.
