\newpage
\section{Fehlerrechnung}
\label{sec:Fehlerrechnung}
Die Mittelwerte der Messgrößen $x_i$ bestimmen sich in der Auswertung nach:
\begin{align}
  \bar{x}=\frac{1}{n} \sum_{i=1}^n x_i\,.
\end{align}
Für die Standardabweichung ergibt sich:
\begin{align}
 s_i=\sqrt{\frac{1}{n-1}\sum_{j=1}^n (v_j-\bar{v_i})^2}
\end{align}
mit $v_j$ mit $j=1,..,n$ als Wert mit zufällig behafteten Fehlern.
Der Fehler des Mittelwertes ergibt sich aus der Formel \eqref{eqn:mitfehl}
\begin{align*}
  s_{\overline{x}}=\frac{s_i}{\sqrt{n}}.\label{eqn:mitfehl}
\end{align*}

Diese Größen werden mit Hilfe von
Numpy 1.9.2, einer Erweiterung von Python 3.2.0, berechnet.
Der Fehler $\Delta f$ für einen Wert $f$ der von fehlerbehafteten Größen $x_j$ mit Fehler $\Delta x_j$  abhängt wird mit der Gauß´schen Fehlerfortpflanzung berechnet
 \eqref{eqn:gaus}.
\begin{equation}
\Delta f=\sqrt{\sum_{j=1}^n \left(\frac{\partial f}{\partial x_j}\Delta x_j \right)^{2} }\label{eqn:gaus}.
\end{equation}
Diese wird von der Erweiterung Uncertainties 2.4.6.1 von Python 3.2.0 übernommen.
Desweitern wird in der Auswertung eine Lineare Regression benutzt,
um die Konstanten A und B aus einen Gleichung der Form
\begin{align}
  y(x)=&A+B\cdot x\label{eqn:lin}
\intertext{zu berechnen. B errechnet sich hierbei aus der Formel}
B&=\frac{\overline{xy}-\overline{x}\cdot \overline{y}}{\overline{x^2}-\overline{x}^2}.
\intertext{und A durch die Geleichung }\\
A&=\overline{y}-B\cdot \overline{x}\,.\\
\intertext{Die Ungenauigkeit von A und B ergibt sich aus der
mittleren Streuung:}\\
s_{\mathrm{y}}&=\sqrt{\frac{1}{N-2}\cdot \sum_{i=1}^{N}\left(y_i-A-B\cdot x_i\right)^2}\,.\\
\intertext{Für die Ungenauigkeit von B gilt:}\\
s_{\mathrm{B}}&=s_{\mathrm{y}} \cdot \sqrt{\frac{1}{N \cdot \left(\overline{x^2}-\left(\overline{x}\right)^2\right)}}\,.\\
\intertext{Für die Ungenauigkeit von A gilt:}\\
s_{\mathrm{A}}&=s_{\mathrm{B}} \cdot\sqrt{\overline{x^2}}\,.
\end{align}
Für die Lineare Regression wird die Erweiterung Scipy 0.15.1 für Python 3.2.0
benutzt.
Für andere Fits wird die Funktion curve$\_$fit, die ebenfalls in der
Erweiterung Scipy enthalten, ist genutzt.

Abweichungen von den Theoriewerten werden mit der Formel
\begin{align}
  a=\frac{|a_\mathrm{gemessen}-a_\mathrm{theorie}|}{a_\mathrm{theorie}} \label{eqn:abweich}
\end{align}
berechnet.

\newpage
