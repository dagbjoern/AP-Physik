\section{Diskussion}
\label{sec:Diskussion}
Das Experiment zur Bestimmung
der Wärmeleitfähigkeit liefert Ergebnisse,
die im Rahmen des Erwarteten liegen.
Zwar besitzen die Ergebnisse von Messing und Aluminium eine recht hohe
Ungenauigkeit. Diese lassen sich durch die Mittelung der Amplituden und
Phasendifferenzen bergrüngen. Am Anfang der Messung sind die Wärmewellen
nicht so konstant aufgetreten, was die Ungenauigkeiten bei der Mittelung
verursacht. Eine weiteres Problem ist der Vergleich mit Literaturwerten
vom Edelstahl, da sich die Wärmeleitfähigkeit schon bei unterschiedlichen
Legierungen deutlich ändert.

Die Berechnung vom Wärmestrom zeigt deutlich, dass dieser im Betrag
am Anfang höher ist und mit der Zeit auf einen konstanten Wert abfällt.
Alles in allem kann gesagt werden, dass mit diesem Experiment
die Wärmeleitfähigkeiten von unterschiedlichen Materialien bestimmt werden kann,
aber für einen höhere Genauigkeit noch mehr Perioden gemessen werden sollten.
