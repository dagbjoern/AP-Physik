\section{Theorie}
\label{sec:Theorie}
Temperaturunterschiede in einem Körper können durch Konvektion, Wärmestrahlung oder Wärmeleitung ausgeglichen werden.
Mit dem letzteren Fall, den Wärmeleitungen, wird sich in diesem Versuch beschäftigt.
Herrscht ein Temperaturunterschied in einem langen Stab, so fließt eine Wärmemenge $dQ$ vom wärmeren um kälteren Ende.
Für die Wärmemenge gilt:
\begin{align}
  dQ=-\kappa A\frac{\partial T}{\partial x}dt
\end{align}
mit A als Querschnitsfläche und $\kappa$ als materialspezifische Wärmekapazität.
Mit Hilfe der Kontinuitätsgleichung und der Stromdichte, für die gilt:
\begin{align}
  j_\omega=-\kappa\frac{\partial T}{\partial x},
\end{align}
ergibt sich für die eindimensionale Wärmeleitungsgleichung:
\begin{align}
  \frac{\partial T}{\partial t}=\frac{\kappa}{\rho c} \frac{\partial^2 T}{\partial^2 x^2}\label{eqn:wärmeleitungsgleichung}.
\end{align}
Diese beschreibt den räumlichen und zeitlichen Verlauf der Temperaturverteilung mit $\sigma_T=\frac{\kappa}{\rho c}$ als
Temperaturleitfähigkeit. Diese ist ein Maß für die Schnelligkeit des Temperaturausgleiches.
Wird der Stab periodisch erwärmt und abgekühlt so breitet sich in diesem eine Temperaturwelle der Form:
\begin{align}
  T(x,t)=T_\mathrm{max} e^{\left[-\sqrt{\frac{\omega\rho c}{2\kappa}} \ \right]\cos\left(\omega t-\sqrt{\frac{\omega\rho c}{2\kappa}} x\right)}
\end{align}
aus.Die Phasengeschwindigkeit der Welle lautet:
\begin{align}
  v=\sqrt{\frac{2\omega\kappa}{\rho c}}.
\end{align}
Mit dem Amplitudenverhältnis $A_\mathrm{nah}$ und $A_\mathrm{fern}$ an den Stellen $x_\mathrm{nah}$ und $x_\mathrm{fern}$
lässt sich die Dampfung bestimmen. Weiterhin lassen sich die Beziehungen:
\begin{align}
  \omega&=\frac{2\pi}{T^{\ast}} \intertext{T^{\ast} bezeichnet die Periodendauer} \\
  \phi&=\frac{2\pi\Delta t}{T^{\ast}} \intertext{\phi bezeichnet die Phase}
\end{align}
ausnutzen, um die Wärmeleitfähigkeit zu bestimmen:
\begin{align}
  \kappa=
\end{align}
