\newpage
\section{Auswertung}
\label{sec:Auswertung}
\subsection{Eigenschaften der Bauteile}
Für die Messungen werden Bauteile mit folgenden für die Auswertung relevanten Eigenschaften eingesetzt:
Induktivität der Spule: $L=(23,954\pm )mH$\\
Kapazität vom Kondensator: $(C=0,7932\pm )nF$\\
Kapazität der Spule: $C_{Sp}=(0,028\pm )nF$\\
\subsection{Abstimmung der Resonanzfrequenz}
Der Theoriewert für die Resonanzfrequenz ist gegeben durch Gleichung \eqref{eqn:Frequenz1},
diese muss angepasst werden, da die Spule ebenfalls eine geringe Kapazität $C_{Sp}$ besitzt.\\
Es gilt somit:\\
\begin{equation}
  \nu^{+}=\frac{1}{2\pi\sqrt{LC+C_{Sp}}}=(35,9\pm 1,2)kHz
\end{equation}\\
Der Fehler errechnet sich mit der Gauß'schen Fehlerfortpflanzung \eqref{eqn:gaus}.
Bei der Messung ist es nicht möglich beide Schwingkreise auf die gleiche Resonanzfrequenz abzustimmen, deshalb folgen zwei Messwerte:
\begin{align}
f_1&=(32,89 \pm )kHz\\
f_2&=(33,11 \pm )kHz
\end{align}
\subsection{Verhältnis von Schwingung und Schwebung}
Der theoretische Wert für das Verhältnis ergibt sich nach Gleichung \eqref{eqn:verhaeltnis}.
Mit der Anpassung, bedingt durch die Kapazität $C_{Sp}$ der Spule, gilt für $\nu^{+}$ und $\nu^{-}$:
\begin{align}
  \nu^{+}&=\frac{1}{2\pi\sqrt{LC+C_{Sp}}}\label{eqn:angepasst1} \\
  \nu^{-}&=\frac{1}{2\pi\sqrt{L\left(\frac{1}{C}+\frac{2}{C_K}\right)^{-1}+C_{Sp}}}\label{eqn:angepasst2}
\end{align}\\
Die Abweichung von dem Verhältnis nach der Theorie $n$ und dem gemessenen Verhältnis $n_{M}$ errechnet sich nach:\\
\begin{equation}
  a=\frac{\lvert n_{M}-n\rvert}{n}
\end{equation}\\
Für $n_M$ wird, durch die geringe Ausprägung der Wellen, ein Ablesefehler von $\pm 1$ angenommen.\\

\begin{table}
 \centering
 \caption{Verhältnis von Schwingungs- und Schwebungfrequenz von Theorie und Messwerten }
 \label{tab:a)}
 \begin{tabular}{c c c}
   \toprule
{$ n_M $} & {$ n $} & {$ a $} \\
\midrule
n.v.    &2.2 \pm  0.2  & n.v. \\
n.v.    &3.8 \pm  0.6  & n.v.\\
2 \pm 2 &4.5  \pm 0.7 & 0.56\\
6 \pm 2 &7.1  \pm 1.2 & 0.16\\
6 \pm 2 &9.9  \pm 1.8 & 0.39\\
8 \pm 2 &11.7 \pm 2.2 & 0.32\\
10\pm 2 &14.1 \pm 2.7 & 0.29\\
12\pm 2 &16.7 \pm 3.2 & 0.28\\


\bottomrule
\end{tabular}
\end{table}

\subsection{Bestimmung der Fundamentalfrequenzen}
In der Folgenden Tabelle werden die gemessenen Fundamentalfrequenzen $f_n$ mit den theoretischen Frequenzen $\nu^{+/-}$ verglichen.
Dies geschieht mit zwei Unterschiedlichen Verfahren
Die theoretischen Frequenzen berechnen sich nach Gleichung \eqref{eqn:angepasst1} und \eqref{eqn:angepasst2}.
Fehler ergeben sich wieder nach der Formel \eqref{eqn:gaus}. Ein Ablesefehler von $\pm 5 kHz$ wird abgeschätzt.
\subsubsection{Ohne Sweep}

\begin{table}
 \centering
 \caption{Theorie und Messwerte der Fundamentalfrequenzen mit Aufbau \ref{abb:fundamentalaufbau} ohne Sweep}
 \label{tab:b)}
 \begin{tabular}{c c c c c }
   \toprule
{$C_k/ nF  $} & {$ f1/kHz $} & {$ \nu^+/kHz $} & {$f2/kHz $} & {$\nu^-/kHz$}\\
   \midrule
1.0\pm0.2    & 33.1\pm5   & 35.9 \pm 1.2 &  79.3\pm5  &   56.2\pm 3.5\\
2.2\pm0.4    & 33.1\pm5   & 35.9 \pm 1.2 &  59.5\pm5  &   46.5\pm 2.3\\
2.7\pm0.5    & 32.6\pm5   & 35.9 \pm 1.2 &  56.8\pm5  &   44.8\pm 2.0\\
4.7\pm0.9    & 33.7\pm5   & 35.9 \pm 1.2 &  48.0\pm5  &   41.3\pm 1.6\\
6.8\pm1.4    & 33.7\pm5   & 35.9 \pm 1.2 &  43.8\pm5  &   39.7\pm 1.4\\
8.2\pm1.6    & 32.8\pm5   & 35.9 \pm 1.2 &  41.8\pm5  &   39.1\pm 1.4\\
10.0\pm0.20  & 32.8\pm5   & 35.9 \pm 1.2 &  39.3\pm5  &   38.5\pm 1.4\\
12.0\pm0.24  & 33.1\pm5   & 35.9 \pm 1.2 &  40.0\pm5  &   38.1\pm 1.3\\
\bottomrule
\end{tabular}

\end{table}

\subsubsection{Mit Sweep}

Nun wird die Messung mit dem Sweep durchgeführt.
Um aus den gemessenen Zeiten die Fundamentalfrequenz
zu errechnen werden diese in die Gleichung \eqref{eqn:Sweep}
eingesetzt.
\begin{equation}
  f_n=f_{start}+(f_{end}-f_{start})\frac{t_n}{T}\label{eqn:Sweep}
\end{equation}
Hierbei ist $f_{start}$ die Startfrequenz
,$f_{end}$ die Endfrequenz und
$T$ die Dauer des Sweeps.
$f_{start}$ beträgt an dieser Stelle $19.23 \pm 5 kHz$.
$f_{end}=97.66 \pm 5kHz$ und $T=1s$
Die Ergebnisse sind in der Tabelle 3 dargestellt.


\begin{table}
 \centering
 \caption{Theorie und Messwerte der Fundamentalfrequenzen mit Aufbau \ref{abb:fundamentalaufbau} mit Hilfe von Sweep}\label{tab:c)}
 \begin{tabular}{c c c c c c c }
   \toprule
   & \multicolumn{2}{c}{Zeitlicher Abstand $t_n$}\\
   & \multicolumn{2}{c}{vom Sweepanfang}\\
{$C_k/ nF $} & {$t_1/ms$} & {$t_2/ms$} & {$ f_1/kHz $} & {$ \nu^+/kHz $} & {$f_2/kHz $} & {$\nu^-/kHz$}\\
   \midrule
   1.0 \pm 0.2 & 180\pm5 & 784\pm5 & 33.3\pm 4.2 & 35.9\pm 1.2 &  80.7 \pm 4.1 & 56.2\pm 3.5 \\
   2.2 \pm 0.4 & 176\pm5 & 524\pm5 & 33.0\pm 4.2 & 35.9\pm 1.2 &  60.3 \pm 3.6 &  46.5\pm 2.3 \\
   2.7 \pm 0.5 & 180\pm5 & 476\pm5 & 33.3\pm 4.2 & 35.9\pm 1.2 &  56.6 \pm 3.6 &  44.8\pm 2.0 \\
   4.7 \pm 0.9 & 176\pm5 & 368\pm5 & 33.0\pm 4.2 & 35.9\pm 1.2 &  48.1 \pm 3.7 &  41.3\pm 1.6 \\
   6.8 \pm 1.4 & 188\pm5 & 324\pm5 & 33.9\pm 4.2 & 35.9\pm 1.2 &  44.6 \pm 3.8 &  39.7\pm 1.4 \\
   8.2 \pm 1.6 & 180\pm5 & 300\pm5 & 33.3\pm 4.2 & 35.9\pm 1.2 &  42.8 \pm 3.8 &  39.1\pm 1.4 \\
   10.0\pm 2.0 & 180\pm5 & 276\pm5 & 33.3\pm 4.2 & 35.9\pm 1.2 &  40.9 \pm 3.9 &  38.5\pm 1.4 \\
   12.0\pm 2.4 & 176\pm5 & 260\pm5 & 33.0\pm 4.2 & 35.9\pm 1.2 &  39.6 \pm 3.9 &  38.1\pm 1.3 \\
\bottomrule
\end{tabular}
\end{table}
