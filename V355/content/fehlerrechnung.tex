\section{Fehlerrechnung}
 \label{fehlerrechnung}
Bei dem Versuch entstehen Messfehler,
diese gilt es in Folgenden zu erörtern.
Bei der Messung weicht die Kapazität des Koppelkondensators $C_k$ laut Gerät um $\pm 20 \%$ ab. Für die Werte $C,L $ und $C_Sp$ gilt
ein geschätzte Fehler von$ \pm 5 \% $.
Bei der Messung der Frequenz bei jeder Messreihe
wurde diese mit Hilfe des Ozilloskopes gemessen,
da diese nicht expliziet am Generator abgelesenwerden kann.
Dadurch entsteht eine Fehler, da die Frequenzen
nur durch positionieren von Cursorn bestimmt werden
die nicht genau einzustellen sind. Dieser Fehler beträgt $\pm 5 kHz$
\\
Der Fehler einzelnen Frequenzen, die aus der Theorie berechnet werden, wird mit der Gauß´schen Fehlerfortpflanzung
berechnenet \eqref{eqn:gaus}.
\begin{equation}
\Delta f= \sqrt{\left(\frac{\partial f}{\partial x}\Delta x \right)^{2} + \left( \frac{\partial f}{\partial y}\Delta y\right)^2...}\label{eqn:gaus}
\end{equation}
\\
Dies übernimmt das Programm Python mit Hilfe uncertainties.
