\section{Fehlerrechnung}
 \label{fehlerrechnung}
Bei dem Versuch entstehen Messfehler;
diese gilt es im Folgenden zu erörtern.
Bei der Messung weicht die Kapazität des Koppelkondensators $C_k$ laut Gerät um $\pm 20 \%$ ab. Für die Werte $C,L $ und $C_{sp}$ gilt
ein abgeschätzter Fehler von  $\pm 5\si{\percent}$.
Bei der Messung der Frequenz bei jeder Messreihe
wird diese mit Hilfe des Ozilloskopes gemessen,
da diese nicht expliziet am Generator abgelese nwerden kann.
Dadurch entsteht ein Fehler, da die Frequenzen
nur durch Positionieren von Cursorn bestimmt werden,
die nicht genau einzustellen sind. Dieser Fehler beträgt $\pm 5 \si{\kilo\hertz}$.
\\
Die Fehlerfortpflanzung für die aus der Theorie berechneten einzelnen Frequenzen

 wird mit% der Gauß´schen Fehlerfortpflanzung
%berechnet:
%\begin{equation}
%\Delta f= \sqrt{\left(\frac{\partial f}{\partial x}\Delta x \right)^{2} + \left( \frac{\partial f}{\partial y}\Delta y\right)^2...}\label{eqn:gaus}.
%\end{equation}

 dem Programm Python mit Hilfe von der Bibliothek uncertainties ausgeführt.
