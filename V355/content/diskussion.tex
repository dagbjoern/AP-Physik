\section{Diskussion}
\label{sec:Diskussion}
Bei der Anpassung der Resonanzfrequenzen lassen sich
die beiden Schwingkreise nicht  genau auf die gleiche Resonanzfrequenz
eichen. Es besteht eine Abweichung zwischen den Werten
\begin{align}
f_1&=(32,89\pm5)\,\si{\kilo\hertz}\\
f_2&=(33,11\pm5)\,\si{\kilo\hertz}
\intertext{von}
Ab&=0,22\,\si{\kilo\hertz}.
\end{align}
Diese Abweichung wirkt sich auf den gesamten Versuch aus und entsteht
dadurch, dass sich zum einen die Kapazität nicht auf den
benötigten Wert einstellen lässt, dies ist dem verwendeten Gerät geschuldet,
und zum anderen führen die Ableseungenauigkeiten zu einer
noch größeren Abweichung der beiden Resonanzfrequenzen.
\\
\\
Die Fehler bei den Verhältnissen von Schwingung und Schwebung ist ebenfalls durch Ablesefehler und Justagefehler
erklärbar, da bei manchen Kapazitäten keine oder nur eine ungenaue
Periode erkennbar war. Dies lässt sich durch die Dämpfung erklären.
Trotzdem kann gesagt werden, dass die Formel \eqref{eqn:verhaeltnis}
zutrifft.
\\
\\
Bei den unterschiedlichen Messungen der
Fundamentalfrequenzen $\nu^+ $ und $\nu^-$
stimmen die beiden gemessenen Werte der beiden
Messvarianten fast überein. Werden nun die
Theoriewerte hinzugezogen, liegen nur für $\nu^+$ die Theoriewerte noch
im Fehler der gemessenen Werte.
Bei den Werten
von $\nu^-$ weichen
die gemessenen Werte bei geringeren Kapazitäten
deutlich von der Theorie ab. Aber bei steigender
Kapazität liegen wieder die Theoriewerte im Feher der gemessenen Werte. Diese Abweichung lässt
sich durch einen nicht genau nachvollziehbaren
systematischen Fehler erklären und variert ungefähr zwischen $2\,\si{\percent}$ und $44\,\si{\percent}$. Aus dem Versuch lässt sich
die Formel für $ \nu^+$ \eqref{eqn:angepasst1}
und für $\nu^-$ \eqref{eqn:angepasst2} bestätigen.
Für $\nu^-$ gilt dies nur ab höheren Kapazitäten.
Die Messung mit Hilfe von Sweep lieferte genauere Messwerte
als die ohne Sweep. Folglich ist die Messung mit Sweep
geeigeneter für die Messung der Fundamentalfrequenzen
eines gekoppelten Schwingkreises.
