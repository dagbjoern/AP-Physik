\section{Diskussion}
Bei der Anpassung der Resonanzfrequenzen lassen sich
die beiden Schwingkreise nicht  genau auf die gleiche Resonanzfrequenz
eichen, da sich zum einen die Kapazität nicht auf den
benötigten Wert einstellen lässt, dies ist dem verwendeten Gerät geschuldet,
und zum anderen führen die Ableseungenauigkeiten zu einer
noch größeren Abweichung der beiden Resonanzfrequenzen.
Dieser anfängliche Fehler wirkt sich auf den gesamten Versuch aus.
\\
\\
Die Fehler bei den Verhältnissen ist ebenfalls durch Ablesefehler
erklärbar, da bei manchen Kapazitäten keine oder nur eine ungenaue
Periode erkennbar war. Dies lässt sich durch die Dämpfung erklären.
Trotzdem kann gesagt werden, dass die Formel \eqref{eqn:verhaeltnis}
zutrifft.
\\
\\
Bei den unterschiedlichen Messungen der
Fundamentalfrequenzen $\nu^+ $ und $\nu^-$
stimmen die beiden gemessenen Werte der beiden
Messvarianten fast überein. Werden nun die
Theoriewerte hinzugezogen, stimmen nur die gemessenen Werte
für $\nu^+$ überein. Bei den Werten
von $\nu^-$ weichen
die gemessenen Werte bei geringeren Kapazitäten
deutlich von der Theorie ab. Aber bei steigender
Kapazität wird der Fehler geringer. Diese Abweichung lässt
sich durch einen nicht genau nachvollziehbaren
systematischen Fehler erklären. Aus dem Versuch lässt sich
die Formel für $ \nu^+$ \eqref{eqn:angepasst1}
und für $\nu^-$ \eqref{ean:angepasst2} bestätigen.
Für $\nu^-$ gilt dies nur ab höheren kapazitäten.
Die Messung mit Hilfe von Trigger lieferte genauere Messwerte
als die ohne Trigger. Folglich ist die Messung mit Trigger
geeigeneter für die Messung der Fundamentalfrequenzen
eines Gekoppeltenschwingkreises.


\label{sec:Diskussion}
