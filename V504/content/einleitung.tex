Werden Metalloberflächen erhitzt, treten Elektronen aus. Dies wird als glühelektrischer Effekt bezeichnet
und ist Gegenstaand dieses Versuchs. Untersucht werden dabei die Kennlinien und ihre drei Bereiche
einer Hochvakuumdiode.
