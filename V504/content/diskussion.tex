\newpage
\section{Diskussion}
Das Ergebnis für die experimentelle
Bestimmung des Exponeneten der Formel \eqref{eqn:LRS} beträgt
\begin{align*}
b=1,48\pm0,01
\end{align*}
und weicht somit nur um $1,3\si{\percent}$ von dem
aus der Theorie ab.
\begin{align*}
b_\mathrm{theo}=1,5.
\end{align*}
Folglich lässt sich mit diesem Experiment der Zusammenhang aus Formel
\eqref{eqn:LRS} für das Raumladungsgebietes verifizieren.
Desweiteren kann die theoretische Austrittsarbeit von Wolfram
mit dem experimentell ermittelten verglichen werden.
In der Tabelle \ref{tab:ver} sind die entsprechendne Werte und
deren Abweichungen aufgelistet.
\begin{table}
  \centering
  \caption{Die theoretische Austrittsarbeit \cite{web} und der aus dem Experiment
  ermittelte Wert und dessen Abweichung.}
  \label{tab:ver}
  \begin{tabular}{c c c}
\toprule
$\Phi_\mathrm{theo}/\si{\electronvolt} $ & $\overline{\Phi}/\si{\electronvolt}$ & Abweichung $a/\si{\percent}$\\
\midrule
 4,54 & 4,85\pm0,07 & 6,8 \\
\bottomrule
\end{tabular}
\end{table}
Die Abweichung der experimentellen bestimmten Austrittsarbeit von der
der Theorie liegt im Rahmen der Messung.
Desweiteren lässt sich die bestimmte Kathodentemperatur bei maximaler Heizleistung
über den Anlaufstrom
\begin{align*}
  T_\mathrm{Anlauf}=(2573\pm35)\si{\kelvin}
\intertext{und über die Leistungsbilanz}
T_\mathrm{Leistung}=(2340\pm2)\si{\kelvin}
\end{align*}
vergleichen.
Es wird deutlich, dass die Temperaturen in der größen Ordnung zwar übereinstimmen
aber trotzem leicht von einander abweichen.
Die Ursache dafür liegt in dem nA-Meter, da
es nicht immer möglich war den Strom von diesem genau abzulesen.
Desweiteren veränderte sich die stärke des Strom bei einem Skalenwechsel
ohne das die Gegenspannung verändert wurde.
Folglich eignet sich die Variante über die Leistungsbilanz
eher um die Temperatur zu bestimmen, da dort die Fehlerquellen geringer sind.
