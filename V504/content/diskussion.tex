\newpage
\section{Diskussion}
Das Ergebnis für die experimentelle
Bestimmung des Exponeneten der Formel \eqref{eqn:LRS} beträgt
\begin{align*}
b=1,48\pm0,01
\end{align*}
und weicht somit nur um $1,3\si{\percent}$ von
\begin{align*}
b_\mathrm{theo}=1,5.
\end{align*}
ab.
Folglich lässt sich mit diesem Experiment der Zusammenhang aus Formel
\eqref{eqn:LRS} für das Raumladungsgebiets verifizieren.
Desweiteren kann die theoretische Austrittsarbeit von Wolfram
mit dem experimentell Ermittelten verglichen werden.
In der Tabelle \ref{tab:ver} sind die entsprechenden Werte und
deren Abweichungen aufgelistet.
\begin{table}
  \centering
  \caption{Die theoretische Austrittsarbeit \cite{web} und der aus dem Experiment
  ermittelte Wert und dessen Abweichung.}
  \label{tab:ver}
  \begin{tabular}{c c c}
\toprule
$\Phi_\mathrm{theo}/\si{\electronvolt} $ & $\overline{\Phi}/\si{\electronvolt}$ & Abweichung $a/\si{\percent}$\\
\midrule
 4,54 & 4,85\pm0,07 & 6,8 \\
\bottomrule
\end{tabular}
\end{table}
\FloatBarrier
Die Abweichung der experimentellen bestimmten Austrittsarbeit von der
Theorie liegt unterhalb von $10\si{\percent}$ somit kann von einer erfolgreichen Messung
gesprochen werden.
Desweiteren lässt sich die bestimmte Kathodentemperatur bei maximaler Heizleistung
über den Anlaufstrom
\begin{align*}
  T_\mathrm{Anlauf}=(2570\pm40)\si{\kelvin}
\intertext{und über die Leistungsbilanz}
T_\mathrm{Leistung}=(2340\pm2)\si{\kelvin}
\end{align*}
vergleichen.
Es wird deutlich, dass die Temperaturen ungefähr übereinstimmen
aber trotzem leicht von einander abweichen.
Die Ursachen dafür liegen zum Beispiel in dem nA-Meter, da
es nicht immer möglich war den Strom von diesem genau abzulesen sowie
nicht berücksichtigen Leistungsverluste in der Apparatur.
Desweiteren veränderte sich die stärke des Strom bei einem Skalenwechsel
ohne das die Gegenspannung verändert wurde.

Folglich eignet sich die Variante über die Leistungsbilanz
eher um die Temperatur zu bestimmen, da dort die Fehlerquellen geringer sind.
