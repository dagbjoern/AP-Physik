\newpage
\section{Diskussion}
Das Ergebnis für die experimentelle
Bestimmung des Exponeneten der Formel \eqref{eqn:LRS} beträgt
\begin{align*}
b=1,48\pm0,01
\end{align*}
und weicht somit nur um $1,3\si{\percent}$ von dem
aus der Theorie ab.
\begin{align*}
b_\mathrm{theo}=1,5.
\end{align*}
Folglich lässt sich mit diesem Experiment der Zusammenhang aus Formel
\eqref{eqn:LRS} für das Raumladungsgebietes verifizieren.
Desweiteren kann die theoretische Austrittsarbeit von Wolfram
mit den experimentell ermittelten verglichen werden.
In der Tabelle \ref{tab:ver} sind die entsprechendne Werte und
deren Abweichungen aufgelistet.
\begin{table}
  \centering
  \caption{Die theoretische Austrittsarbeit \cite{web} und die aus dem Experiment
  ermittelten Werte und deren Abweichungen.}
  \label{tab:ver}
  \begin{tabular}{c c c c c}
\toprule
$\Phi_\mathrm{theo}/\si{\electronvolt} $ & $\Phi_\mathrm{anlauf}/\si{\electronvolt}$ & Abweichung $a_1/\si{\percent}$ & $\Phi_\mathrm{leist}/\si{\electronvolt}$ & Abweichung $a_2/\si{\percent}$\\
\midrule
 4,54 & 5,52\pm0,08 & 21,6 & 4,737\pm0,005  & 4,3  \\
\midrule
  \end{tabular}
\end{table}
Es wird deutlich, dass die Austrittsarbeit die über den
Anlaufstrom bestimmt würde stärker
von dem Theoriewert Abweicht.
Die Ursache dafür liegt in dem nA-Meter, da
es nicht immer möglich war den Strom von diesem genau abzulesen.
Desweiteren veränderte sich die stärke des Strom bei einem Skalenwechsel
ohne das die Gegenspannung verändert wurde.
Folglich eignet sich die Variante über die Leistungsbilanz
eher um die Austrittsarbeit zu bestimmen.
