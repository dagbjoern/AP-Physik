\section{Durchführung}
\label{sec:Durchführung}
In der ersten Messreihe werden fünf Kennlinien aufgenommen. Dazu wird die Heizstrom von $2,5\si{\ampere}$ bis $2,1\si{\ampere}$
in $0,1\si{\volt}$ Schritten eingestellt. Für jede Heizstromseinstellung wird die Absaugspannung von $0-250\si{\volt}$ hochgeregelt,
zuerst in $5\si{\volt}$-Schritten und ab $60\,\si{\volt}$ in $10\,\si{\volt}$-Schritten.
Der entsprechende Strom wird notiert. Der Aufbau ist in Abbildung \ref{fig:aufbau1} zu finden.
\begin{figure}
 \centering
 \includegraphics[width=0.7\textwidth]{Aufbaukenn.png}
 \caption{Aufbau zur Aufnahme der Kennlinien.\cite{sample}}
 \label{fig:aufbau1}
 \end{figure}
Der Heizstrom und Absaugspannung werden jemweils von einem regelbaren Spannungsgerät erzeugt.
An eingenauten Messgeräten können diese Werte abgelesen werden.\\
Bei der anschließenden Messung wird der Anlaufstrom vermessen. Bei maximalem Heizstrom wird eine Gegenspannung angelegt und
in $0,05\si{\volt}$-Schritten bis $1\si{\volt}$ hochgeregelt, der entsprechende Anlaufstrom wird von einem Nanoamperemeter registriert.
Die Abbildung \ref{fig:aufbau2} enthält den benötigten Aufbau.
\begin{figure}
 \centering
 \includegraphics[width=0.7\textwidth]{Aufbauan.png}
 \caption{Aufbau zur Aufnahme der Anlaufstromkurve.\cite{sample}}
 \label{fig:aufbau2}
 \end{figure}
