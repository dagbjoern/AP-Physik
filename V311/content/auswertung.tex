\section{Auswertung}
\label{sec:Auswertung}
Zu Beginn wird die Dicke $d$ der Platinen von Zink und Wolfram
berechnet, dazu wird die Hall-Spannung genutzt:
\begin{align}
U_\mathrm{H}=A_\mathrm{H}\frac{IB}{d}.
\end{align}
Diese Formel wird nun nach $d$ umgestellt und die Messwerte eingesetzt.
Die Hallkonstanten $A_\mathrm{H}$ betragen dabei:
\begin{align*}
  \text{Zink:}\ \ 6,4\cdot10^{-11}\si{\meter\tothe{3}\per\coulomb}, \text{\cite{zink}}\\
  \text{Wolfram:}\ \ 11,8\cdot10^{-11}\si{\meter\tothe{3}\per\coulomb}\text{\cite{wolfram}}.
  \end{align*}
Die Ergebnisse für $d$ werden gemittelt und es ergeben sich die zwei Dicken von:
\begin{align*}
d_\mathrm{Zink}&=(0,67\pm0,16)\si{\micro\meter},\\
d_\mathrm{Wolfram}&=(1,6\pm0,5)\si{\micro\meter}.
\end{align*}
Ebenfalls lässt sich
der Widerstand nach
\begin{align}
  R=\frac{U}{I}
\end{align}
errechnen.
Die Werte für $U$ und $I$ für Zink und Wolfram werden aus der Tabelle \ref{tab:R} entnommen und anschließend gemittelt.
Es ergeben sich folgende gemittelte Widerstände:
\begin{align*}
\text{Zink:} \ R&=(11,2\pm0.4)\si{\milli\ohm},\\
\text{Wolfram:} \ R&=(12,4\pm0,6)\si{\milli\ohm}.
\end{align*}
\begin{table}
  \centering
  \caption{Messwerte zur Bestimmung des Wiederstandes }
  \label{tab:R}
  \begin{tabular}{c c c}
    \toprule
                       &     Zink                        & Wolfram\\
Strom $I/\si{\ampere}$ & Spannung $U/\si{\milli\volt}$  & Spannung $U/\si{\milli\volt}$ \\
    \midrule
    0,5   &  6,0  &  6,8\\
    1,0   &  11,7 &  12,9\\
    1,5   &  17,4 &  19,5\\
    2,0   &  22,7  &  25,8\\
    2,5   &  28,5  &  32,0\\
    3,0   &  33,8  &  37,3\\
    3,5   &  39,2  &  42,5\\
    4,0   &  44,8  &  48,3\\
    4,5   &  50,4  &  53,8\\
    5,0   &  56,0  &  59,7\\
    5,5   &  60,6  &  63,0\\
    6,0   &  66,1  &  68,3\\
    6,5   &  73,5  &\\
    7,0   &  78,3  &\\
    7,5   &  79,4  &\\
    8,0   &  83,3  &\\
   \bottomrule
  \end{tabular}
\end{table}
\FloatBarrier

In Tabelle \ref{tab:hys} sind die Werte für die Hysteresekurve des Magnetfeldes aufgelistet, daraus lässt sich die maximale Flussdichte des Magnetfeldes
für die Messung der Hall-Spannung entnehmen:
\begin{align*}
 B_\mathrm{max}=1,229\si{\tesla}.
\end{align*}
\begin{table}
  \centering
  \caption{Messwerte für Hysterese Kurve.}
  \label{tab:hys}
  \begin{tabular}{c c}
    \toprule
    Strom $I/\si{\ampere}$ & Flussdichte $B/\si{\tesla}$\\
    \midrule
    5,0   &  1,226 \\
    4,5   &  1,179 \\
    4,0   &  1,087 \\
    3,5   &  0,997 \\
    3,0   &  0,870 \\
    2,5   &  0,733 \\
    2,0   &  0,570 \\
    1,5   &  0,428 \\
    1,0   &  0,296 \\
    0,5   &  0,155 \\
    0,0   &  0,007 \\
    0,5   &  0,137 \\
    1,0   &  0,279 \\
    1,5   &  0,414 \\
    2,0   &  0,555 \\
    2,5   &  0,691 \\
    3,0   &  0,825 \\
    3,5   &  0,980 \\
    4,0   &  1,095 \\
    4,5   &  1,170 \\
    5,0   &  1,229 \\
    \bottomrule
  \end{tabular}
\end{table}
\FloatBarrier
Die Abbildung \ref{fig:hys} enthält die gemessenen Werte aus der Tabelle\ref{tab:hys}.
\begin{figure}
  \centering
  \includegraphics[width=0.7\textwidth]{hysterese.pdf}
  \caption{Die gemessene Hysteresekurve.}
  \label{fig:hys}
\end{figure}
\FloatBarrier
Mit Hilfe der zuvor bestimmten Parametern werden nun die mikroskopischen Leitfähigkeitsparameter
bestimmt.
Dazu gehören unter anderem die Ladungsträger pro Volumen $n$, die durch Umstellen der Formel \eqref{eqn:u_h} in die Form
\begin{align}
  n=\frac{B \cdot I}{U_\mathrm{H}\cdot e_0 \cdot d}
\end{align}
berechnet werden können.
Für Zink und Wolfram ergibt sich Folgendes:
\begin{align*}
n_\mathrm{Zink}&=(9,8\pm2,3)\cdot10^{28}\si{\coulomb\per\meter\tothe{3}},\\
n_\mathrm{Wolfram}&=(5,3\pm2,1)\cdot10^{28}\si{\coulomb\per\meter\tothe{3}}.
\end{align*}

Aus den Ladungsträgern pro Volumen kann nun die Zahl der Ladungsträger pro Atom $z$
bestimmt werden. Dies geschieht über die folgende Formel:
\begin{align}
  z=\frac{n \cdot V_m}{N_\mathrm{A}}.
\end{align}
Hierbei steht $V_m$ für das Molare Volumen des Stoffes.
\begin{align*}
\text{Dieses beträgt für Zink:}& \ V_m=9,16\cdot10^{-6}\si{\meter\tothe{3}\per\mol}\\
\text{ und für Wolfram}:& \ V_m=9,47\cdot10^{-6}\si{\meter\tothe{3}\per\mol}.
\end{align*}
Es ergibt sich:
\begin{align}
z_\mathrm{Zinn}&=(1,45\pm0,35),\\ %Einheiten???
z_\mathrm{Wolfram}&=(0,83\pm0,33)
\end{align}
Des Weiteren kann nun durch die Formel \eqref{eqn:wiederstand} die mittlere
Flugzeit $\overline\tau$ bestimmt werden. Durch Umstellen ergibt sich:
\begin{align}
  \overline{\tau}=\frac{2m_0\cdot L}{e_0^2 \cdot n \cdot R \cdot Q}.
\end{align}
Dabei ist Q der errechnete Querschnitt und L die gemessene Länge der Leiterplatte.
\begin{align*}
  Q_\mathrm{Zinn}&=(1,7\pm0,4)\cdot10^{-8}\si{\square\meter},\\
  Q_\mathrm{Wolfram}&=(3,1\pm1,2)\cdot10^{-8}\si{\square\meter},\\
l_\mathrm{Zinn}&=0,034\si{\meter},\\
l_\mathrm{Wolfram}&=0,04\si{\meter}.\\
\end{align*}

%Abmessungen!! war zu faul um das Messheft zu holen :D
Für die mittlere Flugzeit ergeben sich Werte von:
\begin{align*}
\overline{\tau}_\mathrm{Zinn}&=(1,33\pm0,2)\cdot10^{-18}\si{\second},\\
\overline{\tau}_\mathrm{Wolfram}&=(1,4\pm0,07)\cdot10^{-19}\si{\second}.
\end{align*}
Die mittlere Driftgeschwindigkeit $\overline{v}_\mathrm{d}$ ergibt sich durch die Formel \eqref{eqn:vd}:
\begin{equation}
  \overline{v}_\mathrm{d}=-\frac{j}{n \cdot e_0}\label{eqn:vd}.
\end{equation}
Für eine Stromdichte von $1\si{\ampere\per\milli\meter\tothe{3}}$ ergeben sich folgende Driftgeschwindigkeiten:
\begin{align}
\overline{v}_\mathrm{d_\mathrm{Zinn}}&=(-6,4\pm1,5)\cdot10^{-5}\si{\meter\per\second},\\
\overline{v}_\mathrm{d_\mathrm{Wolfram}}&=(1,2\pm0,5)\cdot10^{-4}\si{\meter\per\second}.
\end{align}
Eine weiterer mikroskopischer Leitfäigkeitsparameter ist die Beweglichkeit $\mu$.
Diese berechnet sich aus der Formel \eqref{eqn:mu}:
\begin{equation}
\mu=-\frac{e_0\cdot\overline{\tau}}{2m_0}\label{eqn:mu}.
\end{equation}
Die beiden Werte betragen:
\begin{align*}
\mu_\mathrm{Zinn}&=(-0,0117\pm0,0004)\si{\square\meter\per\volt\second},\\
\mu_\mathrm{Wolfram}&=(-0,0123\pm0,0006)\si{\square\meter\per\volt\second}.
\end{align*}
Für die Totalgeschwindigkeit $|v|$ ist es notwendig zuvor die entsprechende Fermi-Energie
$E_\mathrm{F}$ aus der Formel \eqref{eqn:EF} zu berechnen:
\begin{equation}
E_\mathrm{F}=\frac{\mathrm{h}^2}{2m_0}\sqrt[3]{\left(\frac{3}{8\pi}n\right)^2}\label{eqn:EF}.
\end{equation}
Die Fermi-Energie ergibt sich zu:
\begin{align*}
E_\mathrm{F_\mathrm{Zinn}}&=(1,24\pm0,20)\cdot10^{-18}\si{\joule},\\
E_\mathrm{F_\mathrm{Wolfram}}&=(0,82\pm0,22)\cdot10^{-18}\si{\joule}.
\end{align*}
Nun  kann sowohl über die Formel \ref{eqn:v} die Totalgeschwindigkeit $|v|$ also auch
über die Formel \ref{eqn:l} die mittlere frei Wellenlänge  $\overline{\ell}$ bestimmt werden.
Die Werte sind:
\begin{align*}
  |\overline{v}|_\mathrm{Zinn}&=(1,65\pm0,13)\cdot10^{6}\si{\meter\per\second},\\
  |\overline{v}|_\mathrm{Wolfram}&=(1,34\pm0,18)\cdot10^{6}\si{\meter\per\second},\\
  \overline{\ell}_\mathrm{Zinn}&=(0,22\pm0,02)\si{\micro\meter},\\
  \overline{\ell}_\mathrm{Wolfram}&=(0,19\pm0,03)\si{\micro\meter}.
\end{align*}
