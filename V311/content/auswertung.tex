\section{Auswertung}
\label{sec:Auswertung}
Zu Beginn wird die Dicke $d$ der Platinen von Zink und Wolfram berechnet, dazu wird die Hall-Spannung genutzt:
\begin{align}
U_\mathrm{H}=A_\mathrm{H}\frac{IB}{d}.
\end{align}
Die Hallkonstanten $A_\mathrm{H}$ betragen:
\begin{align*}
  \text{Zink:}\ \6,4\cdot 10^{-11} \si{\meter\tothe{3}\per\coulomb}\\
  \text{Wolfram:}\ \
\end{align*}
Der Widerstand errechnet sich nach:
\begin{align}
  R=\frac{U}{I}.
\end{align}
Die Werte für $U$ und $I$ für Zink und Wolfram werden aus der Tabelle \ref{tab:R} entnommen und anschließend gemittelt.
Es ergeben sich gemittelte Widerstände von:
%\begin{align*}
%\end{align*}
\begin{table}
  \centering
  \caption{Messwerte zur Bestimmung des Wiederstandes }
  \label{tab:R}
  \begin{tabular}{c c c}
    \toprule
                       &     Zink                        & Wolfram\\
Strom $I/\si{\ampere}$ & Spannung $U/\si{\milli\volt}$  & Spannung $U/\si{\milli\volt}$ \\
    \midrule
    0,5   &  6,02  &  6,8\\
    1     &  11,74 &  12,9\\
    1,5   &  17,43 &  19,5\\
    2,0   &  22,7  &  25,8\\
    2,5   &  28,5  &  32,0\\
    3     &  33,8  &  37,3\\
    3,5   &  39,2  &  42,5\\
    4     &  44,8  &  48,3\\
    4,5   &  50,4  &  53,8\\
    5     &  56,0  &  59,7\\
    5,5   &  60,6  &  63,0\\
    6     &  66,1  &  68,3\\
    6,5   &  73,5  &\\
    7     &  78,3  &\\
    7,5   &  79,4  &\\
    8     &  83,3  &\\
   \bottomrule
  \end{tabular}
\end{table}

In Tabelle \ref{eqn:hys} sind die Werte für die Hysteresekurve des Magnetfeldes aufgelistet, daran lässt sich die Flussdichte des Magnetfeld
für die Messung der Hall-Spannung entnehmen:
\begin{align*}
 B_\mathrm{max}=1,229\si{\tesla}.
\end{align*}
\begin{table}
  \centering
  \caption{Messwerte für Hysterese Kurve.}
  \label{tab:hys}
  \begin{tabular}{c c}
    \toprule
    Strom $I/\si{\ampere}$ & Flussdichte $B/\si{\tesla}$\\
    \midrule
    5     &  1,226 \\
    4,5   &  1,179 \\
    4     &  1,087 \\
    3,5   &  0,997 \\
    3     &  0,870 \\
    2,5   &  0,733 \\
    2     &  0,570 \\
    1,5   &  0,428 \\
    1     &  0,296 \\
    0,5   &  0,155 \\
    0     &  0,007 \\
    0,5   &  0,137 \\
    1     &  0,279 \\
    1,5   &  0,414 \\
    2     &  0,555 \\
    2,5   &  0,691 \\
    3     &  0,825 \\
    3,5   &  0,980 \\
    4     &  1,095 \\
    4,5   &  1,170 \\
    5     &  1,229 \\
    \bottomrule
  \end{tabular}
\end{table}
