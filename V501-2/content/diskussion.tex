\newpage
\section{Diskussion}
Die aus den Messwerten bestimmte Apperaturkonstante
\begin{align*}
a_\mathrm{experiment}=(233\pm3)\,\si{\milli\meter}
\end{align*}
weicht von der theoretische Apperaturkonstante
\begin{align*}
  a_\mathrm{theo}=357,5\,\si{\milli\meter}.
\end{align*}
um $9,6\,\si{\percent}$. Diese Abweichung
liegt im Rahmen der Messung
und somit können die Formeln, die die
Ablenkung eines Elektronenstahls im
E-Feld beschreibt, über das Experiment
verifiziert werden.
Des weiteren kann
die Synchronisationsbedingung
bestätigt werden, auch wenn
leichte Abweichungen bei den
gemessenen Frequenzen existieren.
Diese entstehen durch die eher
grob einstellbare Sägezahnfrequenz.

Die über die Ablenkung
eines Elektronenstrahls im transversalen
B-Feld bestimmte spezifische Ladung
\begin{align*}
\overline{\frac{e_0}{m_0}}=(1,84\pm0,02)\cdot 10^{11}\,\si{\coulomb\per\kilo\gram},.
\end{align*}
weicht um $4,6\si{\percent} $
von dem Theoriewert
\begin{align*}
\frac{e_0}{m_0}_\mathrm{theo}=1,759\cdot 10^{11}\,\si{\coulomb\per\kilo\gram} \text{\cite{em}}
\end{align*}
ab. Folglich kann durch dieses Experiment
die spezifische Ladung recht genau bestimmt werden.
Die bestimmte Erdmagnetfeldstärke
\begin{align*}
  B_\mathrm{total}=39,4\,\si{\micro\tesla}
\end{align*}
liegt im Rahmen der Stärke des
Erdmagnetfeldes, das laut \cite{erde}
zwischen $25\,\si{\micro\tesla}-70\,\si{\micro\tesla}$
liegt.
