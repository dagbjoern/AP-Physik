\section{Auswertung}
\label{sec:Auswertung}
\subsection{E-Feld}
Die Tabellen \ref{tab:1}-\ref{tab:2} enthalten
die Messwerte für die Ablenkspannungen $U_d$ und Verschiebungen $D$
bei unterschiedlichen Beschleunigungsspannugen $U_B$.

\begin{table}
  \centering
  \caption{Die Messwerten von Ablenkspannung $U_d$ und Verschiebung $D$ bei
   unterschiedlichen Beschleunigungsspannugen $U_B$.}
  \label{tab:1}
  \begin{tabular}{c |c c c c c}
  \toprule  %\multicolumn{2}{c}{}
          & $U_B=180\,\si{\volt}$ & $U_B=250\,\si{\volt}$ &  $U_B=300\,\si{\volt}$ \\
$D/\si{\milli\meter}$ & $U_d/\si{\volt}$ & $U_d/\si{\volt}$ & $U_d/\si{\volt}$ \\
  \midrule
0,0\pm0,2    &  -19,5 &  -27,3  &   -32,4\\
6,3\pm0,2    &  -16,3 &  -22,4  &   -26,9\\
12,7\pm0,2   &  -12,9 &  -18,0  &   -21,1\\
19,0\pm0,2   &  -9,6  &  -13,3  &   -15,9\\
25,4\pm0,2   &  -6,2  &  -8,8   &   -10,5\\
31,8\pm0,2   &  -2,9  &  -4,4   &   -4,5\\
38,1\pm0,2   &  0,6   &  0,7    &   1,4\\
44,4\pm0,2   &  4,4   &  5,9    &   7,3\\
50,8\pm0,2   &  8,2   &  10,6   &   13,2\\
%\multicolumn{2}{c}{}
\bottomrule
\end{tabular}
\end{table}
\FloatBarrier
\begin{table}
  \centering
  \caption{Die Messwerten von Ablenkspannung $U_d$ und Verschiebung $D$ bei
   unterschiedlichen Beschleunigungsspannugen $U_B$.}
  \label{tab:2}
  \begin{tabular}{c | c c c c c}
  \toprule  %\multicolumn{2}{c}{}
          & $U_B=400\,\si{\volt}$ & $U_B=500\,\si{\volt}$ \\
\midrule
$D/\si{\milli\meter}$ & $U_d/\si{\volt}$ & $U_d/\si{\volt}$ \\
  \midrule
0,0\pm0,2  & -27,9  & -34,2  \\
3,2\pm0,2  & -25,1  & -30,6  \\
6,3\pm0,2  & -21,6  & -26,9  \\
9,5\pm0,2  & -17,8  & -22,6  \\
12,7\pm0,2  & -14,1  & -17,7  \\
15,9\pm0,2  & -10,1  & -12,0  \\
19,0\pm0,2  & -6,8   & -8,0  \\
22,2\pm0,2  & -2,4   & -3,5  \\
25,4\pm0,2  &  0,9   &  1,9 \\
%\multicolumn{2}{c}{}
\bottomrule
\end{tabular}
\end{table}
\FloatBarrier

Die Messwerte aus den Tabellen \ref{tab:1}-\ref{tab:2} werden
nun in den Abbildugnen \ref{fig:1} \ref{fig:2} in der Form
$D$ gegen $U_d$ aufgetragen. Mit Hilfe einer linearen Regression
wird die Empfindichkeit $D/U_d$ für die
unterschiedlichen Beschleunigungsspannugen $U_d$ berechnet.
Dafür wird die Formel \eqref{eqn:D}
\begin{align}
  D=m*U_B+b \label{eqn:D}
\end{align}
genutzt.
Der Parameter $m$ ist die Gesuchte Empfindlichkeit $D/U_d$.



\begin{figure}
 \centering
 \includegraphics[width=0.7\textwidth]{plotV5011.pdf}
 \caption{Die Messwerte der Tabellen \ref{tab:1}-\ref{tab:2} in der Form
 $D$ gegen $U_d$ aufgetragen.  }
 \label{fig:1}
\end{figure}

\begin{figure}
 \centering
 \includegraphics[width=0.7\textwidth]{plotV5011.pdf}
 \caption{Die Messwerte aus der Tabellen \ref{tab:1}-\ref{tab:2} in der Form
 $D$ gegen $U_d$ aufgetragen.}
 \label{fig:2}
\end{figure}

\begin{figure}
 \centering
 \includegraphics[width=0.7\textwidth]{plotV5012.pdf}
 \caption{Die Messwerte aus der Tabellen \ref{tab:1}-\ref{tab:2} in der Form
 $D$ gegen $U_d$ aufgetragen.}
 \label{fig:2}
\end{figure}



Die unterschiedlichen Empfindlichkeiten aus den linearen Regressionen sind in
der Tabelle \ref{tab:m} aufgelistet.

\begin{table}
  \centering
  \caption{Die Werten der Beschleunigungsspannugen $U_B$ und die dazugehörige Empfindlichkeit $D/U_d$.}
  \label{tab:m}
  \begin{tabular}{c c}
  \toprule  %\multicolumn{2}{c}{}
Beschleunigungsspannung  &  Empfindlichkeit\\
       $U_B/\si{\volt}$ &  $D\, {U_d}^{-1}/\si{\milli\meter\per\volt}$\\
  \midrule
     180 & 1,84\pm0,02 \\
     250 & 1,35\pm0,01 \\
     300 & 1,11\pm0,01 \\
     400 & 0,86\pm0,01 \\
     500 & 0,69\pm0,02 \\
%\multicolumn{2}{c}{}
\bottomrule
\end{tabular}
\end{table}
\FloatBarrier

Die Werte aus der Tabelle \ref{tab:m} werden nun
ebenfalls in ein Diagramm in der Form $D/U_d$ in Abhängigkeit von $1/U_B$
aufgetragen. Dieses ist in der Abbildung \ref{fig:A} zusehen.

\begin{figure}
 \centering
 \includegraphics[width=0.7\textwidth]{plotV501a.pdf}
 \caption{Die Werte der Tabellen \ref{tab:m}
  in der Form $D/U_d$ in Abhängigkeit von $1/U_B$.}
 \label{fig:A}
\end{figure}



Druch eine lineare Regession wird wieder eine Ausgleichsgerade bestimmt.
Als Parameter ergeben sich:
\begin{align*}
  m&=(0,233\pm0,003)\,\si{\meter},\\
  b&=(0,5\pm0,1)\cdot10^{-6}\,\si{\meter\per\volt}.
\end{align*}
Aus der Formel
\eqref{eqn:??} ergibt sich, dass $m$ der Apperaturkonstante
\begin{align*}
  a=\frac{L\,p}{2d}
\end{align*}
entspricht.
Dabei ist $L$ der Abstand zwischen Ablenkplatten und Schirm,
$p$ die Länge der Ablenkplatten und $d$ der Abstand zwischen den
beiden Ablenkplatten.
Bei diesem Experiment beträt
\begin{align*}
  L&=143\si{\milli\meter},\\
  p&=19\si{\milli\meter},\\
\shortintertext{und}
  d&=3,8\si{\milli\meter}.
\end{align*}
Aus diesen Werten ergibt sich eine theoretische Apperaturkonstante
von
\begin{align*}
  a_\mathrm{theo}=357.5\,\si{\milli\meter}.
\end{align*}

Für die gemessenen Sychronisationsfrequenzen siehe Tabelle \ref{tab:syn}
sollen die ,
die Faktoren $m$ und $n$
aus der Synchronisationsbedingung
\begin{align*}
  n\,\nu_\mathrm{Sä}=m\,\nu_\mathrm{sin}
\end{align*}
bestimmt werden.
Da der Sinusgenerator eine Frequenz
ungefähr zwischen $80\si{\kilo\hertz}$ und $90\si{\kilo\hertz}$
ausgibt,kann angenommen werden, dass für $\nu_\mahtrm{Sä_1}$ die Faktoren $m$ und $n$
gleich $1$ sind. Somit lässt sich auch für die anderen Frequenezen $\nu_\mathrm{Sä}$ die Faktoren über
die Synchronisationsbedingungen abschätzen.

\begin{table}
  \centering
  \caption{Die Sychronisationsfrequenzen der Sinusspannung.}
  \label{tab:syn}
  \begin{tabular}{c c c c}
\toprule  %\multicolumn{2}{c}{}
& $\nu_\mathrm{Sä}/\si{\kilo\hertz}$ & n & m \\
\midrule
\nu_\mathrm{Sä_1} & 79,39  & 1 & 1 \\
\nu_\mathrm{Sä_2} & 26,47  & 3 & 1 \\
\nu_\mathrm{Sä_3} & 19,85  & 4 & 1 \\
\nu_\mathrm{Sä_4} & 129,57 & $\frac{1}{2}$ & 1\\
\bottomrule
\end{tabular}
\end{table}
\FloatBarrier

Desweiteren kann die Amplitude der Sinusspannung bestimmt werden.
Dafür wird die gemessenen maximale Strahlauslenkung
\begin{align*}
D_\mathrm{max}=(10\pm2)\,\si{\milli\meter},
\end{align*}
die durch die Sinusspannung erzeugt wird,
und die verwendete Beschleunigungsspannung
\begin{align*}
U_B=300\,\si{\volt}
\end{align*}
in die nach $U_d$ umgestellte Gleichung \eqref{eqn:umge}
\begin{align}
U_d=\frac{U_B\,D_max}{a_\mathrm{theo}}\label{eqn:umge}
\end{align}
eingesetzt.
Es ergbit sich eine Amplitude von
\begin{align*}
U_\mathrm{sin}=(8,0\pm1,7)\,\si{\volt}
\end{align*}





\subsection{B-Feld}
Die Messwerte zur Bestimmung von $e_0/m_0$ sind in den Tabelle
\ref{tab:11}
aufgelistet.

\begin{table}
  \centering
  \caption{Die Messwerten von .}
  \label{tab:11}
  \begin{tabular}{c |c c c c |c| c}
  \toprule  %\multicolumn{2}{c}{}
          & $U_B=250\,\si{\volt}$ & $U_B=300\,\si{\volt}$ &  $U_B=350\,\si{\volt}$ &  $U_B=400\,\si{\volt}$  & &  $U_B=500\,\si{\volt}$ \\
$D/\si{\milli\meter}$ & $I/\si{\ampere}$ & $I/\si{\ampere}$ & $I/\si{\ampere}$  & $I/\si{\ampere}$ & $D/\si{\milli\meter}$ &$I/\si{\ampere}$\\
  \midrule
0,0\pm0,2  & 0,0    &  0,0   & 0,0   &  0,0   & 0,0\pm0,2  & 0,0\\
6,3\pm0,2  & 0,3    &  0,31  & 0,40  &  0,42  & 3,2\pm0,2  & 0,21\\
12,7\pm0,2 & 0,6    &  0,62  & 0,77  &  0,82  & 6,3\pm0,2  & 0,46\\
19,0\pm0,2 & 0,91   &  1,01  & 1,12  &  1,19  & 9,5\pm0,2  & 0,67\\
25,4\pm0,2 & 1,25   &  1,35  & 1,51  &  1,60  & 12,7\pm0,2 & 0,89\\
31,8\pm0,2 & 1,56   &  1,71  & 1,90  &  2,01  & 15,9\pm0,2 & 1,12\\
38,1\pm0,2 & 1,88   &  2,06  & 2,28  &  2,40  & 19,0\pm0,2 & 1,35\\
44,4\pm0,2 & 2,22   &  2,45  & 2,68  &  2,81  & 22,2\pm0,2 & 1,56\\
50,8\pm0,2 & 2,51   &  2,78  & 3,04  &   -    & 25,4\pm0,2 & 1,80\\
%\multicolumn{2}{c}{}
\bottomrule
\end{tabular}
\end{table}
\FloatBarrier

Diese werden nun in der Form $D/(L^2+D2)$ in Abhängigkeit von der magnetischen Flussdichte $B$ aufgetragen
siehe Abbildungen \ref{fig:11}-\ref{fig:22}.
Die Flussdichte berechnet sich über den Zusammenhang \eqref{eqn:b}:
\begin{align}
  B=\mu_0\frac{8}{\sqrt{125}}\frac{N\,I}{R}.\label{eqn:b}
\end{align}
Die Windungen der Spule $N$ beträgt
\begin{align*}
  N&=20.\\
  \intertext{Der Radius $R$ beträgt:}
R&=0,282\,\si{\meter}
\end{align*}

\begin{figure}
 \centering
 \includegraphics[width=0.7\textwidth]{plotV5021.pdf}
 \caption{Die Werte der Tabellen \ref{tab:11}
in der Form $D/(L^2+D2)$ in Abhängigkeit von $B$.}
 \label{fig:11}
\end{figure}


\begin{figure}
 \centering
 \includegraphics[width=0.7\textwidth]{plotV5022.pdf}
 \caption{Die Werte der Tabellen \ref{tab:12}
  in der Form $D/(L^2+D2)$ in Abhängigkeit von $B$.}
 \label{fig:12}
\end{figure}

Wieder wird mit den aufgetragenen Werten
eine linerare Ausgleichsrechnung durchgeführt.
Aus der Formel \eqref{eqn:??} kann
entnommen werden, dass
der bestimmte Parameter
\begin{align}
m=\frac{1}{\sqrt{8\,U_B}}\sqrt{\frac{e_0}{m_0}}\label{eqn:m}
\end{align}
entspricht. Um nun die spezifische Ladung zu bestimmen,
 wird die Formel \eqref{eqn:m} nach $e_0/m_0$ umgestellt
\begin{align*}
  \frac{e_0}{m_0}=8\,U_B\,m^2\label{eqn:em}
\end{align*}
und die entsprechenden Steigungen $m$ sowie die
Beschleunigungsspannungen $U_B$ in die Formel \eqref{eqn:em}
eingesetzt.
In der Tabelle \ref{tab:em} sind die Parameter der
Ausgleichsrechnung bei den unterschiedlichen $U_B$ sowie die berechneten spezifischen Ladungen aus Formel \eqref{eqn:em}
aufgelistet.


\begin{table}
  \centering
  \caption{Die Ergenisse der Ausgleichsrechnung und die berechneten spezifischen Ladungen.}
  \label{tab:em}
  \begin{tabular}{c c c c}
  \toprule  %\multicolumn{2}{c}{}
$U_B/\si{\volt}$ & $m/\si{\per\tesla\per\meter}$ & $b/\si{\per\meter}$ & $\frac{e_0}{m_0}/\si{\coulomb\per\kilo\gram}$ \\
  \midrule
&  &  &   \\
%\multicolumn{2}{c}{}
\bottomrule
\end{tabular}
\end{table}
\FloatBarrier

Die berechneten Werte aus Tabelle \ref{tab:em}
für die spezifische Ladung des Elektron werden gemittelt
und des ergibt sich ein Wert von:
\begin{align*}
\overline{\frac{e_0}{m_0}}=.
\end{align*}


Um die Intensität des lokalen Erdmagnetfeldes
zu bestimmen, wird der gemessene Strom
\begin{align*}
I=0,16
\end{align*}
bei dem die Horizontalkomponete
des Ergdmagnetfelds $B_\mathrm{hor}$
von dem Spulenfeld $B_\mathrm{Spule} $ kompensiert wird in die
Gleichung \eqref{eqn:b} eigesetzt.
Es gilt somit:
\begin{align*}
|B_\mathrm{hor}|=|B_\mathrm{Spule}|=... .
\end{align*}
Für die Gesamtintensität $B_\mathrm{total}$
wird der gemessene Inklinationswinkel
\begin{align*}
  \phi=
\end{align*}
und B_\mathrm{hor} in die Formel \eqref{eqn:total}
\begin{align}
  B_\mathrm{total}=\frac{B_\mahtrm{hor}}{\cos{phi}}\label{eqn:total}
\end{align}
eingesetzt.
Es ergibt sich eine Gesamtintensität von:
\begin{align*}
  B_\mathrm{total}=..
\end{align*}
