\begin{table}

%puppe Arme Ausgestreckt
%  Tgemessen           Igemssen           Itheorie
5.53\pm0.5  &  0.00059\pm0.00017  &  0.00074\pm0.00024\\
5.44\pm0.5  &  0.00057\pm0.00017  &  0.00074\pm0.00024\\
5.40\pm0.5  &  0.00057\pm0.00016  &  0.00074\pm0.00024\\
5.53\pm0.5  &  0.00059\pm0.00017  &  0.00074\pm0.00024\\
5.55\pm0.5  &  0.00060\pm0.00017  &  0.00074\pm0.00024\\
5.58\pm0.5  &  0.00060\pm0.00017  &  0.00074\pm0.00024\\
5.60\pm0.5  &  0.00061\pm0.00017  &  0.00074\pm0.00024\\
5.43\pm0.5  &  0.00057\pm0.00017  &  0.00074\pm0.00024\\
5.50\pm0.5  &  0.00059\pm0.00017  &  0.00074\pm0.00024\\
5.40\pm0.5  &  0.00057\pm0.00016  &  0.00074\pm0.00024\\


\end{table}

I=D\cdot \farc{T^2}{4\pi^2} % für das gemessene Trägheitsmoment

%für I_D

I=2\cdot I_z+I_D
I=2(I_z+m\cdot a^2)+I_D
D\cdot \farc{T^2}{4\pi^2}=2(I_s+m\cdot a^2)+I_D
D\cdot \farc{T^2}{8\pi^2m}-\frac{I_s}{m}-\frac{I_D}{2m}=a^2
b=-\frac{I_z}{m}-\frac{I_D}{2m}


 zylinder parrallel:
 \begin{equation}

I_z=\frac{mR^2}{2}

 \end{equation}

 zylinder senktrecht
\begin{equation}
  I_{zh}=m\left(\frac{R^2}{4}+frac{h^2}{12}\right)
 \end{equation}
%
\begin{equation}
  I_{Gemessen}=I_D + I_{Körper}
  I_{Körper}= I_{Gemessen} + I_D  
\end{equation}
%Gesamtvolumen
V_z=\pi*R^2h


Vges=2V_{Beine}+2V_{Arme}+V_{Kopf}+V_{Torso}

%dichte
\rho=M_p/Vges
