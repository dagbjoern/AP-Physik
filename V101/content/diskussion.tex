\section{Diskussion}
\label{sec:Diskussion}
Bei den Messwerte von den einzelnen Trägheitsmomentn sind deutliche Abweichungen
zu den Theoriewerten zu erkennen. Desweitern sind alle gemessenen Trägheitsmomente negativ, dies lässt sich durch ein zu größes Eigenträgheitsmoment der Drillachse im Verhältnis
zu den zu messenden Trägheitsmoment begründen. Die weitere Abweichungen entstehen, da bei
der Bestimmung des Trägheitsmomentes der Drillachse die Stange zur befestigung der
Zylinder nicht masselos ist, wie in der Auswertung angenommen. Außerdem wurde die
Periodendauer manuell gemessen, was ebenfalls zu Ungenauigkeiten führt.
Die Näherung der Puppe in der Auswertung ist zu grob um genau Theoriewerte zu liefern.
Und die Drehachse der Puppe in dem Versuch ist nicht wie angenommen genau in der Mitte der Puppe.
All diese Faktoren führen dazu, dass der Satz von Steiner nicht mit dem Versuch als Bestätigt
angesehen werden kann.
