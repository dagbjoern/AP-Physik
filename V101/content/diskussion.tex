\section{Diskussion}
\label{sec:Diskussion}
Bei den Messwerte von den einzelnen Trägheitsmomenten kommt es zu unterschiedlich großen Abweichungen von den Theoriewerten.
Desweitern muss gesagt werden, dass alle gemessenen Trägheitsmomente negativ wären, würde nicht, wie in der Auswertung,
darauf verzichtet werden, das Trägheitsmoment der Drillachse $I_\mathrm{D}$, wie in Formel \eqref{eqn:I} verlangt, abzuziehen.
Weitere Gründe für die unterschiedlichen Abweichungen sind zum Beispiel, dass die
Periodendauer manuell gemessen wurde oder, dass bei der Bestimmung
des Trägheitsmomentes der Drillachse die Stange zur Befestigung der
Zylinder als masselos angenommen wird. Ebenfalls ist die Näherung der Puppe in der Auswertung zu grob, um genau Theoriewerte zu liefern, und
die Drehachse der Puppe in dem Versuch ist nicht wie angenommen genau in der Mitte der Puppe.
Auch eine ausgeleierte Feder könnte der Grund für die Abweichungen sein.
Alles in allem führen diese Faktoren dazu, dass der Satz von Steiner mit dem Versuch nicht verifiziert werden kann.
