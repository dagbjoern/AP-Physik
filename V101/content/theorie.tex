\section{Theorie}
\label{sec:Theorie}

Die Charakteristika von Rotationsbewegungen sind das Drehmoment,
die Winkelbeschleunigung und das Trägheitsmoment.
Das Gesamtträgheitsmoment, mit den Masseelementen $m_{\mathrm{i}}$ und dem Abstand $r_{\mathrm{i}}$
der Massen zur Drehachse, egibt sich zu:
\begin{equation}
  I=\sum_{\mathrm{i}} r_{\mathrm{i}}^2\cdot m_{\mathrm{i}}.
\end{equation}
Dementsprechend gilt für infinitesimale Massen $dm$:
\begin{equation}
  I=\int r^2dm.
\end{equation}
Entspricht die Drehachse nicht einer der Schwerpunktsachsen des Körpers, sondern
ist parallel um den Abstand $a$ versetzt, so kann das Trägheitsmoment mit dem Satzt
von Steiner berechnet werden. Dieser setzt sich zusammen aus dem Trägheitsmoment $I_{\mathrm{s}}$
bezüglich der Drechachse durch den Schwerpunkt, der Gesamtmasse $m$ und dem Abstand $a$
zu:
\begin{equation}
  I=I_{\mathrm{s}}+m\cdot a^2.
\end{equation}
Für das Drehmoment für einen drehbaren Körper gilt:
\begin{equation}
 \vec{M}=\vec{F}\times\vec{r}.
\end{equation}
$\vec{F}$ ist dabei die greifende Kraft im Abstand $\vec{r}$ zur Achse.
Ist das System schwingungsfähig, so wirkt ein rücktreibendes Drehmoment
der Auslenkung um $\varphi$ entgegen. Dies führt zu harmonischen Schwingungen
mit der Schwingdauer:
\begin{equation}
  T=2\pi\sqrt{\frac{I_{\mathrm{{ges}}}}{D}}.
\end{equation}
Für $D$ die Winkelrichtgröße gilt die Beziehung $M=D\cdot\varphi$, hieraus lässt
sich, durch Umstellen, das Trägheitsmoment berechnen.
Es gilt:
\begin{equation}
 I=\frac{T^2 D}{4\pi^2}-I_{\mathrm{D}}.\label{eqn:I}
\end{equation}
$I_{\mathrm{D}}$ ist das Trägheitsmoment der Drillachse und $I_{\mathrm{ges}}$ setzt sich in diesm Versuch zusammen aus dem Trägheitsmoment des Körpers und dem der Drillachse.
Die Winkelrichtgröße bestimmt sich nach:
\begin{equation}
  D=\bigl|\frac{F\cdot r}{\varphi}\bigr|. \label{eqn:D}
\end{equation}\\
Der Betrag darf verwendet werden, da die Winkelrichtgröße immer positiv ist.
Beispiele für Trägheitsmomente verschiedener Körper:
\begin{align}
  I_{\mathrm{Kugel}}&=\frac{2}{5} m (\frac{D}{2})^2.\label{eqn:Kugel}\\
  I_{\mathrm{Zylinder\parallel}}&=\frac{m(\frac{D}{2})^2}{2}\label{eqn:zs}.\\
  I_{\mathrm{Zylinder\bot}}&=m\left(\frac{(\frac{D}{2})^2}{4}+\frac{h^2}{12}\right).\label{eqn:zl}
\end{align}
