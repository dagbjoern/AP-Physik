\section{Theorie}
\label{sec:Theorie}

Ein Stoff kann in drei verschiedenen Phasen, auch Aggregatzustände genannt, vorliegen.
Wie in der Zustandsdiagramm (Abb:\ref{fig:zustand}) zu sehen, können in den abgegrenzten
Bereichen alle zugehörigen $p-$ und $T-$Werte angenommen werden, die Linien
beschreiben die Übergänge der einzelnen Phasen. Auf den Linien existieren zwei
Phasen nebeneinander. In diesem Versuch wird sich mit der Dampfdruckkurve beschaftigt, diese
liegt zwischen dem Tiefpunkt $TP.$ und dem kritischen Punkt $K.P.$.
\begin{figure}
 \centering
 \includegraphics[width=0.7\textwidth]{zustand.png}
 \caption{Qualitatives Zustandsdiagramm von Wasser.}
 \label{fig:zustand}
 \end{figure}\\
 Die Form dieser Kurve ist durch die Verdampfungswärme $L$ bestimmt. Diese ist die benötigte Wärmemenge
 um ein Mol einer Flüssigkeit in Dampf gleicher Temperatur umzuwandeln. Die Größe $L$ ist
 temperaturabhängig, kann aber im rot markierten Bereichen als konstant angesehen werden.
 Beim Umwandlungsvorgang verlassen die Moleküle, die nach der Maxwellwellschen
 Geschwindigkeitsverteilung maximale kinetische Energie besitzen, die Flüssigkeitsoberfläche
 und gehen über in die gasförmige Phase. Um die Energie aufzubringen, die benötigt wird, um
 die Bindungskräfte zu überwinden, muss entweder Energie von außen zugeführt weren oder
 der Flüssigkeit Wärme entzogen werden. Die entzogene Wärme wird bei der Kondensation des
 Gases wieder frei. Nach einiger Zeit stellt sich ein Gleichgewicht ein, es kehrt ebenso
 viel Flüssigkeit bei der Kondensation ins System zurück wie auch verdampft wird.
 In dem Gleichgewichtszustand herrscht der sogenannte Sättigungsdruck.
 Der Sättigungsdruck ist nicht abhängig vom Volumen des Gasraumes, daher lässt sich dieser nicht durch
 die ideale Gasgleichung beschreiben:
 \begin{align}
 \rho V =RT
 \end{align}
 mit R als allgemeine Gaskonstante.
 \subsection{Herleitung der DGL für die Dampfdruckkurve}
Um die Dampfdruckkurve zu ermitteln, wird der reversible Kreisprozess der
Verdampfung und anschließender Kondensation zur Hilfe genommen.
In Abbildung \ref{fig:kreisprozess} ist der Kreisprozess qualitativ dargestellt.
\begin{figure}
 \centering
 \includegraphics[width=0.7\textwidth]{kreisprozess.png}
 \caption{Kreisprozess im pV-Diagramm. }
 \label{fig:kreisprozess}
 \end{figure}\\
Ein Mol einer Flüssigkeit wird um die Temperatur $dT$ erwärmt, es steigt ebenfalls
der Druck um $dp$ und das Volumen auf $V_\mathrm{F}$. Dies entspricht dem Übergang $A\longrightarrow B$.
Beim Übergang $B\longrightarrow C$ geht die Wassermenge, nach Zufuhr der Verdampfungswärme, isobar
und isotherm in Gas über, dabei erweitert sich das Volumen auf $V_\mathrm{D}$.
Im Übergang $C \longrightarrow D$ kühlt sich der Dampf um $dT$ ab und der Druck
fallt ebenfalls zurück auf $p$. Im letzten Übergang $D\longrightarrow A$ wird die
Verdampfungswärme wieder freigesetzt, die Wassermenge kondensiert isobar und isotherm.
Die geleistete Arbeit wird mit der gesamt Wärmeenergie des Prozesses gleichgesetzt,
es ergibt sich folgende Gleichung:
\begin{align}
(C_\mathrm{F}-C_\mathrm{D})dT+dL=(V_\mathrm{D}-v\mathrm{F})dp.
\end{align}\\
Hier beschreiben $C_\mathrm{F}$ und $C_\mathrm{V}$ die Molwärmen im flüssigen und
gasförmigen Zustand und $V_\mathrm{F}$ und $V_\mathrm{D}$ die Volumina von Flüssigkeit und Dampf.
Mit dem zweiten Hauptsatz der Thermodynamik:
\begin{align}
\sum_{i}\frac{Q_i}{T_i}=0
\end{align}
und weiteren Umformungen ergibt sich die Clausius-Clapeyronsche Gleichung:
\begin{align}
(V_\mathrm{D}-V_\mathrm{F})dp=\frac{L}{T}dT\label{eqn:clausius}.
\end{align}
Die Integration der Clausius-Clapeyronsche Gleichung erweist sich als schwierig, weil $V_\mathrm{D}$, $V_\mathrm{F}$ und $L$
komplizierte Funktionen der Temperatur sein können.
Liegt die Temperatur $T$ weit unter der kritischen Temperatur $T_\mathrm{Kr}$, so greifen die Annahmen, dass
$V_\mathrm{F}$ gegen über
$V_\mathrm{D}$ vernächlässigbar ist, für $V_\mathrm{D}$
die idealen Gasgleichung gilt und $L$ Druck- und
Temperaturunabhängig ist.
Mit den Annahmen und Integration gilt nach umformungen:
\begin{align}
  p=&p_\mathrm{0}\exp\left(-\frac{L}{R}\cdot\frac{1}{T}\right)\label{eqn:cceinfach}.
\end{align}
