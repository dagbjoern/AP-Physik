\section{Diskussion}
\label{sec:Diskussion}
Das Ergebnis für die Verdampfungswärme von
der Messung unter einem $\si{\bar}$ besitzt
zwar eine Abweichung von ungefähr
18\% vom Literaturwert, jedoch muss gesagt werden, dass
der Literaturwert für konstanten Druck bei $100\si{\degreeCelsius}$
gilt und somit nur zur groben Überprüfung des gemessenen Wertes dient,
da der Versuch unter isochoren Bedingungen durchgeführt wird.
Desweiteren war die Apperatur nach der Evakuierung nicht komplett Dicht,
da der Druck danach direkt langsam wieder anstieg.\\
\\
Bei der Messung über einem $\si{\bar}$ ergibt sich für die
temperaturabhängige Verdampfungswärme zwei unterschiedliche
Kurven. Aus dem Versuchsskript\cite{sample} lässt sich entnehmen, dass
die Kurve der Verdampfungswärme aus der Abbildung \ref{abb:-} nicht der
geforderten Bedingung entspricht die besagt, dass L in der Nähe der
Temperatur des kritischen Punktes verschwindend klein wird und nicht
wie in dem Graphen \ref{abb:-} stetig steigt.
Diese Bedingung stimmt mit der Kurve aus der Abbildung \ref{abb:+} überein
und somit ist die Gleichung \eqref{eqn:L+}
die gesuchte Abhängigkeit der Verdampfungswärme zur Temperatur.\\
\\
Alles in allem kann gesagt werden, dass aus dem Versuch die Verdampfungswärme
sowohl für unter einem $\si{\bar}$ und als auch für über einem $\si{\bar}$
durch den Versuch bestimmt werden kann.
