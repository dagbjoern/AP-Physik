\section{Theorie}
\label{sec:Theorie}
\subsection{Betrachtungen zur Energieformel}
Bei einer Spektroskopie ergeben sich Energien angeregter Atomzustände, aus diesen Energien lässt sich die Stärke
des Kernfeldes am Ort der äußeren Elektronen bestimmen. Dieses Coulomb-Feld des Kerns wird von den inneren Elektronen
abgeschirmt, dieser Effekt zeichnet sich durch die Abschirmungszahl $\sigma$ aus. Damit herrscht am Ort des angeregten Elektrons
ein Coulomb-Feld, entsprechend dem eines Kerns mit der Ladungszahl:
\begin{align}
  z_\mathrm{eff} = z-\sigma.
\end{align}
Alkali-Metalle haben den Vorteil von nur einem Valenzelektron, damit gilt die Ein-Elektron-Näherung und damit vereinfacht sich die Berechnung der Energie.
Allerdings zeigen sich deutliche Abweichungen, bei der Näherung besteht nur eine Abhängigkeit von der Hauptquantenzahl $n$.
Zur vollständigen Beschreibung bedarf es noch die Bahndrehimpulsquantenzahl $l$ und den Spin $s$. Die beiden Werte setzen sich zusammen
zur Spin-Bahn-Kopplung $j=l+s$, diese Wechselwirkung beeinflusst ebenfalls die Energieeigenwerte.
Unter Berücksichtigung relativistischer Effekte und umständlichen Umformungen ergibt sich eine Formel für die Energie:
\begin{align}
  E_\mathrm{n,j}=-R_\mathrm{\infty}\left\{\frac{(z-\sigma)^2}{n^2}+\alpha^2\frac{(z-\sigma)^4}{n^3}\left(\frac{1}{j+\frac{1}{2}}-\frac{3}{4n}\right)\right\}\label{eqn:sf}
\end{align}
Hierbei ist $\alpha$ die Sommerfeldsche Feinstrukturkonstante und $R_\mathrm{\infty}$ die Rydbergenergie.

\subsection{Auswahlregeln}
Bei Betrachtung der Spektren zeichnen sich Regeln ab:($\Delta\ell=0$) gehorcht dem Übergang $\pm 1$, unwarscheinlich ist der Übergang
$\Delta j=0$. Für $\Delta n$ zeigen sich keine Regeln, lediglich werden die Übergänge unwarscheinlicher mit höherem $\Delta n$.
Energieniveaus mit gleichem $\ell$ aber unterschiedlichem $j$ liegen dichter bei einander, als jenige mit unterschiedlichem $\ell$, zu beobachten
ist die sogenannte Dublett-Struktur, diese charakterisiert sich durch zwei dicht beieinander liegenden Linien.

\subsection{Abschirmungszahlen}
Es ist sinnvoll bei näherer Betrachtung der Gleichung \eqref{eqn:sf} zwei Abschirmungszahlen einzuführen.
Damit ergibt sich für die Energie:
\begin{align}
  E_\mathrm{n,j}=-R_\mathrm{\infty}\left\{\frac{(z-\sigma_\mathrm{1})^2}{n^2}+\alpha^2\frac{(z-\sigma_\mathrm{2})^4}{n^3}\left(\frac{1}{j+\frac{1}{2}}-\frac{3}{4n}\right)\right\}
\end{align}
Dabei ist $\sigma_\mathrm{1}$ die Konstante der vollständigen Abschirmung, d.h. alle Elektronen sind an der Abschirmung beteiligt,
und $\sigma_\mathrm{2}$ ist die Konstante der inneren Abschirmung, daran sind alle Elektronen außer der Valenzelektronen beteiligt.
Um die Konstante $\sigma_\mathrm{2}$ zu bestimmen, ist es notwendig den Abstand $\Delta E_\mathrm{D}$ der zwei linien eines Dubletts auszumessen.
Es gilt:
\begin{align}
  \Delta E_\mathrm{D} = E_\mathrm{n,j}-E_\mathrm{n,j+1}=\frac{R_\mathrm{\infty}\alpha^2}{n^3}(z-\sigma_\mathrm{2})^4\frac{1}{\ell(\ell+1)}.
\end{align}
Da $E=\frac{hc}{\lambda}$ gilt, folgt:
\begin{align}
E_\mathrm{D}=\left(\frac{1}{\lambda}-\frac{1}{\lambda'}\right) \approx hc\frac{\Delta\lambda}{\lambda^2}.
\end{align}
Mit einem Gitterspektralapparat hinreichender Auflösung können $\Delta\lambda$ und $\lambda$ ermittelt werden.
Die emittierte Strahlung wird am Gitter gebeugt und 
