\section{Diskussion}
\label{sec:Diskussion}
Das Experiment zu den Spektren der
Alkali-Atome zeigt, dass die Abschimungszahl
$\sigma_2$ bei allen drei Akalimetallen ungefähr gleich der
Kernladungszahl ist. Durch das Experiment lässt sich
folgern, dass das Lechtelektron kaum einen Beitrag zu Abschirmung
des Kerns leistet. Desweiteren werden die Winkel nur per Augenmaß gemessen,
was zu Ungenauhigkeiten bei der Eichgröße $\varepsilon$ führen kann. Bei der
Bestimmung der Gitterkonstante g über die linearen Regression wurde die
Wellenlänge $438,8\si{\nano\meter}$ nicht berücksichtigt, da
diese von den anderen Werten abweicht. Dies liegt daran, dass
die Spektrallinie bei $\lambda=438,8\si{\nano\meter}$
schwach ausgeprägt ist
und das violette Licht schwer für das menschliche Auge sichtbar ist.
Folglich konnte die Spektrallinie nicht genau erkannt werden. Trotz
dieser Fehlerquellen liefert der Versuch das erwartete Ergebniss.
