\section{Diskussion}
\label{sec:Diskussion}
Das Experiment zur den Spektren der
Alkali-Atome zeigt, dass sich die Abschimungszahl
$\sigma_2$ bei allen drei Akalimetallen ungefähr gleich der
Kernladungszahl ist. Dadurch lässt sich durch das Experiment
folgern, dass das Lechtelektron kaum einen Beitrag zu Abschirmung
des Kerns leistet. Desweiteren werden die Winkel nur per Augenmaß gemessen,
was zu Ungenahigkeiten bei der Eichgröße $\varepsilon$ führen kann. Bei der
Bestimmung der Gitterkonstante g über die Linearenregression wurde die
Wellenlänge $438,8\si{\nano\meter}$ nicht berücksichtigt, da
diese von den anderen Werte abweicht. Dies liegt daran, dass
die Spektrallinie bei $\lambda=438,8\si{\nano\meter}$ schwach ausgeprägt
und das violette Licht schwer für das Meschlicheauge sichtbar ist. Trotz
dieser Fehlerquellen liefert der Versuch das erwartete Ergebniss.    
