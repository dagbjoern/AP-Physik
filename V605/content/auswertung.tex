\newpage
\section{Auswertung}
\label{sec:Auswertung}

\subsection{Eichung des Okularmikrometer}
Um die Gitterkonstante g des verwendeten Transmissionsgitter zu bestimmen,
werden die Beugngswinkel $\phi_\mathrm{i}$ der wichtigsten Spektrallinen
des He-Spektrums , die in der Tabelle \ref{tab:} zu finden sind, verwendet.
\begin{table}
  \centering
  \caption{Beugungswinkel der Spektrallinen des He-Spektrums.}
  \begin{tabular}{c c c}
    \toprule
Wellenlänge  & Beugungswinkel & $ \ $  \\
$\lambda/\si{\nano\meter}$ & $\phi/\si{\radian}$ & $sin(\phi)$\\
    \midrule
    706,5 & 0,77 & 0.696\\
    667,8 & 0,72 & 0.660\\
    587,6 & 0,61 & 0.569\\
    504,8 & 0,52 & 0.500\\
    501,6 & 0,52 & 0.497\\
    492,2 & 0,51 & 0.488\\
    471,3 & 0,49 & 0.468\\
    447,1 & 0,46 & 0.445\\
    438,8 & 0,39 & 0.383\\
    \bottomrule
  \end{tabular}
\end{table}
$sin(\phi)$ wird nun gegen $\lambda$ aufgetragen
wie in Abbildung \ref{fig:plot1} zu sehen
und es wird ein Lineareregression durchgeführt.
\begin{figure}
  \centering
  \includegraphics[width=0.7\textwidth]{a).pdf}
  \caption{$sin(\phi)$ nach der Wellenlänge $\lambda$ aufgetragen.}
  \label{fig:plot1}
\end{figure}
aus der Linearenregression ergeben sich die Werte:
\begin{align*}
b=(1.03\pm0.02)\cdot10^{-6}\si{\micro\meter}
a=(-11\pm10)\cdot10^{-9}
\end{align*}
Aus der Formel für die Hauptmaxima \eqref{eqn:winkel}
\begin{align*}
  sin(\phi)= \mathrm{k}\frac{\lambda}{\mathrm{g}} \,\ \ \ \ \ \mathrm{k} = 0,1,2,\dots
\mathrm{g}\cdot sin(\phi) = k\lambda
\end{align*}
folgt, da es sich um die ersten Hauptmaxima $(k=1)$ handelt, dass
die berechnete Steigung der Ausgleichsgrade
\begin{align*}
b=\mathrm{g}
  \intertext{ist. Somit ergibt sich für die Gitterkonstante}
\mathrm{g}=(1,03\pm0,02)\si{\micro\meter}
\end{align*}
\subsection{Berechung der Eichgröße}
Die Eichgröße lässt sich mit Hilfe der Formel \eqref{eqn:eich}
\begin{equation}
  \varepsilon=\frac{(\lambda_\mathrm{1}-\lambda_\mathrm{2})}{\Delta t\cdot cos(\overline{\phi}_\mathrm{12})}\label{eqn:eich}
\end{equation}
berechnen. Hierbei sind $\lambda_1$ und $\lambda_2$ zwei benachtbarte Spektralinien
und $\overline{\phi}_\mathrm{12}$ der gemittelte Winkel von den 2 gemessenen Winkel der Spektrallinien.
$\Delta \mathrm{t}$ ist der mit dem  Okularmikro-meters
ermittelte Abstand zwischen den beiden Spektrallinien.
In der Tabelle \ref{tab:eich}sind die berechenten Eichgrößen dargestellt.
\begin{table}
  \centering
  \caption{}
  \begin{tabular}{c c c c c}
    \toprule
    $\lambda_1/\si{\nano\meter}$ & $\lambda_2/\si{\nano\meter}$ & $\overline{\phi}_{12}$ & $\Delta t/$ & Eichgröße $\varepsilon / 10^{-11}\si{\meter}$\\
    \midrule
    504,8 & 501,6 & 0,522\pm0,001 & 120 & 3,076\pm0,003\\
    501,6 & 492,2 & 0,515\pm0,005 & 365 & 2,959\pm0,009\\
    \bottomrule
  \end{tabular}
\end{table}
Die verschiedenen berechneten Eichgrößen werden nun gemittelt und es ergibt sich ein Eichgröße von:
\begin{align*}
  \overline\varepsilon=(0,03,0185\pm0,005)\si{\meter}
\end{align*}

\subsection{Bestimmung der Abschirmungszahl}
Um die Abschirmungszahl $\sigma_2$ von Natrium, Kalium und Rubidium
bestimmen zu können, ist es notwendig den Wellenlängenunterschied
$\Delta s$ zwischen den Dublettlinien zu bestimmen. Hierzu wird die
Formel \eqref{eqn:deltalam} verwendet
\begin{align}
\Delta\lambda=cos\overline{\phi}\cdot\Delta s \cdot\overline{\varepsilon}.
\end{align}
Durch $\Delta\lambda$ lässt sich nun die Energiedifferenz $\Delta E_\mathrm{D}$ mit der Formel \eqref{eqn:E}
zwischen den Dublettlinien ausrechnen.
Um nun die Abschimungszahl $\sigma_2$ zu berechnen
wird die Formel \eqref{eqn:popo} nach $\sigma_2$ umgestellt und es ergibt sich:
\begin{equation}
  \sigma_2=z-\sqrt[4]{\frac{\Delta E_\mathrm{D}\cdot{n^3}\cdot l\left(l+1\right)}{\mathrm{R}_\infty \alpha^2}} \label{eqn:sigma}
\end{equation}
In den fogenenden Tabellen sind die nötigen Daten und die
entsprechende Abschirmungszahl aufgelistet.
\subsubsection{Natrium}
Natrium besitzt die Kernladungszahl  $z=11$.
Die gemessenen Na-Dubletts entstehen durch die Aufspaltund des 3P-Niveaus in $3P_\frac{1}{2}$ und $3P_\frac{3}{2}$.
Somit beträgt die Quantenzahl $n=3$ und
die Bahndrehimpulsquantenzahl  $l=1$.

\begin{table}
  \centering
  \caption{Gemessene und errechnete Größen zur Bestimmung der Abschirmungszahl $\sigma_2$ von Natrium. \label{tab:na}}
  \begin{tabular}{c c c c c c c c }
    \toprule
    \multicolumn{2}{c}{Winkel}   & \multicolumn{4}{c}{ \ }    \\
    $\phi_1/\si{\radian}$ & $\phi_2/\si{\radian}$ & $\overline{\phi}/\si{\radian}$ & $\overline{\lambda}/\si{\nano\meter}$ &  $\Delta \mathrm{s}$ & $\Delta\lambda/\si{\nano\meter}$ & $\Delta \mathrm{E}/\mathrm{eV}$ & $\sigma_2$ \\
    \midrule
    0,654 & 0,654 & 0,654\pm0 & 629\pm11 & 27  & 0,646\pm0,001 & $(1,27\pm0,02)\cdot10^{-9}$ & $11\pm2\cdot10^{-10}$ \\
    0,620 & 0,620 & 0,620\pm0 & 600\pm11 & 26  & 0,639\pm0,001 & $(1,32\pm0,02)\cdot10^{-9}$ & $11\pm2\cdot10^{-10}$ \\
    0,597 & 0,597 & 0,597\pm0 & 581\pm11 & 21  & 0,524\pm0,001 & $(1,12\pm0,02)\cdot10^{-9}$ & $11\pm2\cdot10^{-10}$ \\
    \bottomrule
  \end{tabular}
\end{table}

\subsubsection{Kalium}
Kalium besitzt die Kernladungszahl  $z=19$.
Die gemessenen K-Dubletts entstehen durch die Aufspaltund
des 4P-Niveaus in $4P_\frac{1}{2}$ und $4P_\frac{3}{2}$.
Somit beträgt die Quantenzahl $n=4$ und
die Bahndrehimpulsquantenzahl  $l=1$.

\begin{table}
  \centering
  \caption{Gemessene und errechnete Größen zur Bestimmung der Abschirmungszahl $\sigma_2$ von Kalium. \label{tab:k} }
  \begin{tabular}{c c c c c c c c }
    \toprule
    \multicolumn{2}{c}{Winkel}   & \multicolumn{4}{c}{ \ }    \\
    $\phi_1/\si{\radian}$ & $\phi_2/\si{\radian}$ & $\overline{\phi}/\si{\radian}$ & $\overline{\lambda}/\si{\nano\meter}$ &  $\Delta \mathrm{s}$ & $\Delta\lambda/\si{\nano\meter}$ & $\Delta \mathrm{E}/\mathrm{eV}$ & $\sigma_2$ \\
    \midrule
    0,557 & 0,559 & 0,558 & 547\pm10 & 62 & 1,587\pm0,002 & $(3,60\pm0,07)\cdot10^{-9}$ &$19\pm3\cdot10^{-10}$\\
    0,559 & 0,560 & 0,559 & 548\pm10 & 65 & 1,662\pm0,003 & $(3,76\pm0,07)\cdot10^{-9}$ &$19\pm3\cdot10^{-10}$\\
    0,611 & 0,612 & 0,612 & 594\pm11 & 91 & 2,248\pm0,004 & $(4,70\pm0,09)\cdot10^{-9}$ &$19\pm3\cdot10^{-10}$\\
    0,614 & 0,616 & 0,615 & 596\pm11 & 78 & 1,922\pm0,003 & $(4,00\pm0,07)\cdot10^{-9}$ &$19\pm3\cdot10^{-10}$\\
    \bottomrule
  \end{tabular}
\end{table}

\subsubsection{Rubidium}
Rubibium besitzt die Kernladungszahl  $z=37$.
Die gemessenen Na-Dubletts entstehen
durch die Aufspaltund des 5P-Niveaus
in $5P_\frac{1}{2}$ und $5P_\frac{3}{2}$.
Somit beträgt die Quantenzahl $n=5$ und
die Bahndrehimpulsquantenzahl  $l=1$.

\begin{table}
  \centering
  \caption{Gemessene und errechnete Größen zur Bestimmung der Abschirmungszahl $\sigma_2$ von Rubidium. \label{tab:ru}}
  \begin{tabular}{c c c c c c c c }
    \toprule
    \multicolumn{2}{c}{Winkel}   & \multicolumn{4}{c}{ \ }    \\
    $\phi_1/\si{\radian}$ & $\phi_2/\si{\radian}$ & $\overline{\phi}/\si{\radian}$ & $\overline{\lambda}/\si{\nano\meter}$ &  $\Delta \mathrm{s}$ & $\Delta\lambda/\si{\nano\meter}$ & $\Delta \mathrm{E}/\mathrm{eV}$ & $\sigma_2$ \\
    \midrule
    0,614 & 0,672 & 0,667 & 639\pm12 & 409 & 9,70\pm0,04 & $(18,82\pm0,04)10^{-9}$ & $37\pm50\cdot10^{-9}$\\
    \bottomrule
  \end{tabular}
\end{table}
