\newpage
\section{Auswertung}
\label{sec:Auswertung}

\subsection{Eichung des Okularmikrometer}
Um die Gitterkonstante g des verwendeten Transmissionsgitter zu bestimmen,
werden die Beugngswinkel $\phi_\mathrm{i}$ der wichtigsten Spektrallinen
des He-Spektrums , die in der Tabelle \ref{tab:} zu finden sind, verwendet.
\begin{table}
  \centering
  \caption{Beugungswinkel der Spektrallinen des He-Spektrums.}
  \begin{tabular}{c c c}
    \toprule
Wellenlänge  & Beugungswinkel & $ \ $  \\
$\lambda/\si{\nano\meter}$ & $\phi/\si{\radian}$ & $sin(\phi)$\\
    \midrule
    706,5 & 0,77 & 0.696\\
    667,8 & 0,72 & 0.660\\
    587,6 & 0,61 & 0.569\\
    504,8 & 0,52 & 0.500\\
    501,6 & 0,52 & 0.497\\
    492,2 & 0,51 & 0.488\\
    471,3 & 0,49 & 0.468\\
    447,1 & 0,46 & 0.445\\
    438,8 & 0,39 & 0.383\\
    \bottomrule
  \end{tabular}
\end{table}
$sin(\phi)$ wird nun gegen $\lambda$ aufgetragen
wie in Abbildung \ref{fig:plot1} zu sehen
und es wird ein Lineareregression durchgeführt.
\begin{figure}
  \centering
  \includegraphics[width=0.7\textwidth]{a).pdf}
  \caption{$sin(\phi)$ nach der Wellenlänge $\lambda$ aufgetragen.}
  \label{fig:plot1}
\end{figure}
aus der Linearenregression ergeben sich die Werte:
\begin{align*}
b=(1033\pm19)\si{\per\meter}
a=-11\pm10
\end{align*}
Aus der Formel für die Hauptmaxima \eqref{eqn:winkel}
\begin{align*}
  sin(\phi)= \mathrm{k}\frac{\lambda}{\mathrm{g}} \,\ \ \ \ \ \mathrm{k} = 0,1,2,\dots
\end{align*}
folgt, da es sich um die ersten Hauptmaxima $(k=1)$ handelt, dass
die berechnete Steigung der Ausgleichsgrade
\begin{align*}
b=\frac{1}{\mathrm{g}}
  \intertext{ist. Somit ergibt sich für die Gitterkonstante}
\mathrm{g}=967\si{\micro\meter}
\end{align*}
\subsection{Berechung der Eichgröße}
Die Eichgröße lässt sich mit Hilfe der Formel \eqref{eqn:eich}
\begin{equation}
  \varepsilon=\frac{(\lambda_\mathrm{1}-\lambda_\mathrm{2})}{\Delta t\cdot cos(\overline{\phi}_\mathrm{12})}\label{eqn:eich}
\end{equation}
berechnen. Hierbei sind $\lambda_1$ und $\lambda_2$ zwei benachtbarte Spektralinien
und $\overline{\phi}_\mathrm{12}$ der gemittelte Winkel von den 2 gemessenen Winkel der Spektrallinien.
$\Delta \mathrm{t}$ ist der mit dem  Okularmikro-meters
ermittelte Abstand zwischen den beiden Spektrallinien.
In der Tabelle \ref{tab:eich}sind die berechenten Eichgrößen dargestellt.
\begin{table}
  \centering
  \caption{}
  \begin{tabular}{c c c c c}
    \toprule
    $\lambda_1/\si{\nano\meter}$ & $\lambda_2/\si{\nano\meter}$ & $\overline{\phi}_{12}$ & $\Delta t/$ & Eichgröße $\varepsilon / \si{\meter}$\\
    \midrule
    504,8 & 501,6 & 0,522\pm0,001 & 120 & 0,03076\pm0,00003\\
    501,6 & 492,2 & 0,515\pm0,005 & 365 & 0,02959\pm0,00009\\
    \bottomrule
  \end{tabular}
\end{table}
Die verschiedenen berechneten Eichgrößen werden nun gemittelt und es ergibt sich ein Eichgröße von:
\begin{align*}
  \overline\varepsilon=(0,030175\pm0,00005)\si{\meter}
\end{align*}

\subsection{Bestimmung der Abschirmungszahl $\sigma_2$}
Um die Abschirmungszahl $\sigmal_2$ von Natrium, Kalium und Rubidium
bestimmen zu können, ist es notwendig den Wellenlängenunterschied
$\Delta t$ zwischen den Dublettlinien zu bestimmen. Hierzu wird die
Formel \eqref{eqn:deltalam} verwendet
\begin{align}
\Delta\lambda=cos\overline{\phi}\cdot\Delta s \cdot\overline{\varepsilon}
.Durch $\Delta\lambda$
