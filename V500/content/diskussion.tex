\newpage
\section{Diskussion}
Das aus den Messwerten bestimmte Verhältnis
\begin{align*}
\frac{\mathrm{h}}{\mathrm{e_0}}_\mathrm{(Experiment)} &= (2,8 \pm 0,5) \si{\volt\second} \cdot 10^{-15}
\intertext{besitzt eine Abweichung von $32\,\si{\percent}$ von dem theoretischen Wert}
  \frac{\mathrm{h}}{\mathrm{e_0}}_\mathrm{(Theorie)} &= 4,1\,\si{\volt\second} \cdot 10^{-15}.
\end{align*}
Ursachen für diese Abweichung von dem Theoriewert sind erschwerten
Messbedingungen wie zum Beispiel wechselnde Lichtverhältnisse die dazuführen das
der Photostrom leichte Schwankungen aufweist.
Dies wird in der Abbildung \ref{fig:1}\subref{fig:v1}
sichtbar, wo die Messwerte deutlich um den Fitt streuen.
\\
\\
Des Weitern kann der Verlauf der Messwerte aus
der Abbildung \ref{fig:speck}
gedeutet werden. Zu einem kann vermutet werden, dass die Messwerte
für den Photostrom
bei hoher Beschleunignungsspannung gegen einen Sättigungswert streben.
Dies steht im Wiederspruch zu dem Ohmschen Gesetzt.
Die Ursache liegt an der Abhängigkeit des Phototstroms von der
Intensität des einfallenden Lichtes. Zwar erreichen
durch die hoher Beschleunigungsspannung
alle ausgelösten Elektronen die Annode,
da aber die Anzahl der Elektronen durch
die Intensität beschränkt ist,
ist der Storm ebenfalls beschränkt. Um einen Sättigungsstrom bei
endlichen Beschleunigungsspannugen messen zu können
sollte folglich die Intensität verringert werden.
\\
\\
Weiter fallt auf, dass der Photostrom bei der Gegenspannung $U_g$
nicht abrupt auf 0 abfällt.
Dies liegt an der Energieverteilung für die Elektronen im Metall, folglich
besitzen manche Elektronen bei $U_g$ noch genug Energie um die Anode zu erreichen.
\\
\\
Bei hohen Gegenspannungen kann ein entgegengerichteter Strom beobachtet werden.
Ursache dafür ist das Photokathoden Material, das schon bei
Raumtemperatur verdampft und sich teilweise an die Anode anlagert.
Somit tritt nun bei der Anode der Photoeffekt auf.
Da das Material an der Anode nun gering ist, ist die
Anzahl der Elektronen beschränkt und es
wird schnell ein Sättigungsstrom erreicht.
