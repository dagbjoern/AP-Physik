\section{Auswertung}
\label{sec:Auswertung}
Die Tabelle \ref{tab:1} enthält die Messwerte
der Bremsspannung $U$ bei einem bestimmten Photostrom $I$
für verschiedene Wellenlängen.


\begin{table}
  \centering
  \caption{Die Messwerten von Photostrom $I_\mathrm{ph}$ Bremsspannung $U$
   für unterschiedlichen Wellenlängen.}
  \label{tab:1}
  \begin{tabular}{c | c c c c c}
  \toprule  %\multicolumn{2}{c}{}
  Photostrom  & $\lambda=578\,\si{\nano\meter}$ & $\lambda=546\,\si{\nano\meter}$ & $\lambda=492\,\si{\nano\meter}$ & $\lambda=435\,\si{\nano\meter}$ & $\lambda=405\,\si{\nano\meter}$ \\
$I/\si{\pico\ampere}$  & $U_\mathrm{gelb}/\si{\volt}$ & $U_\mathrm{grün}/\si{\volt}$ & $U_\mathrm{blaugrün}/\si{\volt}$ & $ U_\mathrm{violett_1}/\si{\volt}$ & $ U_\mathrm{violett_2}/\si{\volt}$ \\
  \midrule
  0   &   0,45  &  0,542  &  0,82   &  0,980  &   1,054 \\
  2   &   0,413 &  0,527  &  0,628  &  0,970  &   1,040 \\
  4   &   0,394 &  0,517  &  0,507  &  0,962  &   1,025 \\
  6   &   0,371 &  0,507  &  0,387  &  0,953  &   1,012 \\
  8   &   0,353 &  0,496  &  0,304  &  0,937  &   0,994 \\
  10  &   0,344 &  0,489  &  0,239  &  0,933  &   0,984 \\
  12  &   0,330 &  0,479  &  0,180  &  0,935  &   0,975 \\
  14  &   0,314 &  0,469  &  0,103  &  0,922  &   0,962 \\
  16  &   0,301 &  0,463  &  0,010  &  0,916  &   0,951 \\
  18  &   0,291 &  0,455  &  -      &  0,902  &   0,937 \\
  20  &   0,284 &  0,447  &  -      &  0,903  &   0,952 \\
%\multicolumn{2}{c}{}
\bottomrule
\end{tabular}
\end{table}
\FloatBarrier



Die Werte aus der Tabelle \ref{tab:1} werden nun in der
Form $\sqrt{I}$ in Abhängigkeit von U aufgetragen,
wie in Abbildung \ref{fig:1} zu sehen.


\begin{figure}
\centering
\begin{subfigure}{0.48\textwidth}
\centering
\includegraphics[height=5.70cm]{plotV500aGelb.pdf}
\caption{Gelbes Licht $\lambda=577\,\si{\nano\meter}$.}
\label{fig:ge}
\end{subfigure}
\begin{subfigure}{0.48\textwidth}
\centering
\includegraphics[height=5.70cm]{plotV500aGrun.pdf}
\caption{Grünes Licht $\lambda=546\,\si{\nano\meter}$.}
\label{fig:gr}
\end{subfigure}

\begin{subfigure}{0.48\textwidth}
\centering
\includegraphics[height=5.70cm]{plotV500agrunblau.pdf}
\caption{Grün-blaues Licht  $\lambda=492\,\si{\nano\meter}$.}
\label{fig:gb}
\end{subfigure}
\begin{subfigure}{0.48\textwidth}
\centering
\includegraphics[height=5.70cm]{plotV500aBlau.pdf}
\caption{Violettes Licht $\lambda=435\,\si{\nano\meter}$.}
\label{fig:v1}
\end{subfigure}

\begin{subfigure}{0.48\textwidth}
\centering
\includegraphics[height=5.70cm]{plotV500aviolett.pdf}
\caption{Violettes Licht $\lambda=405\,\si{\nano\meter}$.}
\label{fig:v2}
\end{subfigure}

\caption{Die Abbildungen \subref{fig:ge}-\subref{fig:v2} enthalten die
Messwerte, für das jeweilige Licht aus der Tabelle \ref{tab:1}. Diese sind in der Form
$\sqrt{I}$ gegen $U$ aufgetragen.}
\label{fig:1}
\end{figure}




Für die Messwerte für die unterschiedlichen Wellenlängen werden
nun mit Hilfe eines lineraren Fitt eine Ausgleichsgerade bestimmt.
Um nun $U_g$ für die unterschiedlichen Wellenlängen zu bestimmen,
werden die Nullstellen der verschiedenen Ausgleichsgeraden mit der Formel \eqref{eqn:Ug}
\begin{align}
  U_g=-\frac{b}{m}\label{eqn:Ug}
\end{align}
bestimmt.
Die berechneten Werte aus dem linearen Fit und für $U_g$ aus der Formel
\eqref{eqn:Ug} sind in der Tabelle \ref{tab:Ug} aufgelistet.
Des Weiteren wird die Frequenz $\nu$ für die Wellenlängen $\lambda$ über die Formel \eqref{eqn:nu}
\begin{align}
 \nu=\frac{c}{\lambda}\label{eqn:nu}
\end{align}
berechenet. Die Frequenzen sind ebenfalls in der Tabelle \ref{tab:Ug} zu finden.



\begin{table}
  \centering
  \caption{Die berechneten Werte aus dem linearen Fit und für $U_g$.}
  \label{tab:2}
  \begin{tabular}{c c c c c}
  \toprule  %\multicolumn{2}{c}{}
  $\lambda\,\si{\nano\meter}$ & $\nu/\si{\tera\mathrm{h}ertz}$& $m/\si{\per\micro\volt}$  & $b/10^{-6}$ & $U_g/\si{\volt}$      \\
  \midrule
  578 & 520 & -25,2\pm1,0 & 11,7\pm0,4 & 0,464\pm0.024\\
  546 & 549 & -42,8\pm3,1 & 23,9\pm1,5 & 0,558\pm0.053\\
  492 & 609 &  -4,8\pm0,2 &  4,3\pm0,1 & 0,884\pm0.049\\
  435 & 689 & -49,6\pm4,5 & 49,4\pm4,2 & 0,996\pm0.124\\
  405 & 740 & -33,6\pm2.8 & 36,2\pm2,8 & 1,076\pm0.123\\
%\multicolumn{2}{c}{}
\bottomrule
\end{tabular}
\end{table}
\FloatBarrier


In dem Diagramm \ref{fig:e/h} wird die Gegenspannung $U_g$ gegen die Lichtfrequenz $\nu$
aufgetragen.

\begin{figure}
 \centering
 \includegraphics[width=0.7\textwidth]{plotV500b.pdf}
 \caption{Die Messwerte der Tabelle \ref{tab:2} in der Form
 $U_g$ gegen $\nu$ aufgetragen.}
 \label{fig:e/h}
\end{figure}

Wieder wird eine linearer Fit durchgeführt, der
ebenfalls in Abbildung \ref{fig:e/h} enthalten ist.
Aus der umgestellten Gleichung \eqref{eqn:??}
\begin{align}
U_g=\frac{\mathrm{h}}{\mathrm{e_0}}\cdot\nu - \frac{A_k}{\mathrm{e_0}}
\end{align}
kann entnommen werden, dass
es sich bei der ermittelten Steigung
\begin{align*}
m&=(2,8\pm0,5)\si{\volt\second}\cdot10^{-15}
\intertext{um das gesuchte Verhältnis $\mathrm{h/e_0}$ handelt.
Des Weiteren ist der bestimmte y-Achsenabschitt}
b&=-(0,95\pm0,28)\si{\volt}
\intertext{die negative Austrittsarbeit in $\si{\electronvolt}$, also folgt:}
\frac{\mathrm{h}}{\mathrm{e_0}}&=(2,8\pm0,5)\si{\volt\second}\cdot10^{-15}\\
A_k&=(0,95\pm0,28)\si{\electronvolt}.
\end{align*}

Die folgende Tabelle \ref{tab:speck} enthält für die Wellenlänge $\lambda=578$
die Messwerte für den Photostrom $I$ bei einer Spannung $U$ zwischen $-20\si{\volt}$ und
$20\si{\volt}$. Die negative Spannung entspricht dabei einer Gegenspannung und
die positive einer Beschleunigungsspannung.
\begin{center}
\begin{longtable}{c c}
  \caption{Die Messwerte des Photostroms $I$ bei $\lambda=578$ und bei einer Spannung $U$ zwischen $-20\si{\volt}$ und
  $20\si{\volt}$.}
  \label{tab:speck}\\
  \toprule
  $U/\si{\volt}$ & $I/\si{\pico\ampere}$      \\
  \midrule
  \endfirsthead
  \toprule
  $U/\si{\volt}$ & $I/\si{\pico\ampere}$      \\
  \midrule
  \endhead
  \bottomrule
  \endfoot
  -19  &  -17\\
  -18  &  -17\\
  -17  &  -16\\
  -16  &  -16\\
  -15  &  -15\\
  -14  &  -12\\
  -13  &  -12\\
  -12  &  -12\\
  -11  &  -11\\
  -10  & -12\\
  -9   & -11\\
  -8   & -10\\
  -7   & -10\\
  -6   & -9\\
  -5   & -9\\
  -4   & -8\\
  -3   & -7\\
  -2   & -7\\
  -1.8 & -6\\
  -1.6 &  -6\\
  -1.4 &  -6\\
  -1.2 &  -6\\
  -1.0 &  -5\\
  -0.8 &  -4\\
  -0.6 &  -3\\
  -0.4 &   1\\
  -0.3 &   8\\
  -0.2 &   20\\
  -0.1 &   31\\
  0    &   46\\
  0.1  &   57\\
  0.2  &   65\\
  0.3  &   70\\
  0.4  &   80\\
  0.6  &   90\\
  0.8  &   96\\
  1.0  &  120\\
  1.2  &  140\\
  1.4  &  150\\
  1.6 &  160\\
  1.8 &  180\\
  2   &  190\\

  3   &  280\\
  4   &  330\\
  5   &  380\\
  6   &  400\\
  7   &  400\\
  8   &  460\\
  9   &  480\\
  10  &  500\\
  11  &  490\\
  12  &  540\\
  13  &  550\\
  14  &  540\\
  15  &  570\\
  16  &  660\\
  17  &  620\\
  18  &  580\\
  19  &  620\\
% multicolumn{2}{c}{}
\end{longtable}
\end{center}

Der Photostrom aus der Tabelle\ref{tab:speck} wird nun in der Abbildung \ref{fig:speck} gegen
die Spannung $U$ aufgetragen.


\begin{figure}
 \centering
 \includegraphics[width=0.7\textwidth]{plotV500c.pdf}
 \caption{Die Messwerte der Tabelle \ref{tab:speck} in der Form
 $I$ gegen $U$ aufgetragen.}
 \label{fig:speck}
\end{figure}

In der Abbildung \ref{fig:speckzoom} ist der
Bereich um  $0\,\si{volt}$ aus der Abbildung \ref{fig:speck}
noch einmal vergrößert dargestellt.

\begin{figure}
 \centering
 \includegraphics[width=0.7\textwidth]{plotV500c0.pdf}
 \caption{Der Bereich um $0\,\si{\volt}$ aus \ref{fig:speck} vergrößert.}
 \label{fig:speckzoom}
\end{figure}
