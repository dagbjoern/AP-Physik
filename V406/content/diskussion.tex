\newpage
\section{Diskussion}
\label{sec:Diskussion}
Die berechenten Spaltbreiten aus der Auswertung können nun mit den Herstellerangaben
verglichen werden. Dies ist in den Tabelle \ref{tab:abweich1} und \ref{tab:abweich2} dargestellt.
\begin{table}
  \centering
  \caption{Abweichung der Messwerte von den Herstellerangaben für die einzel Spaltbreite b.}
  \label{tab:abweich1}
  \begin{tabular}{c c| c c| c c }
Spalt & Herstellerangaben & \multicolumn{2}{c}{Mikroskopmessung} & \multicolumn{2}{c}{Über Begungsfiguren} \\
      & $b/\si{\milli\meter}$ & $b/\si{\milli\meter}$ & $a/\si{\percent}$ & $b/\si{\milli\meter}$ & $a/\si{\percent}$ \\
    \midrule
1 & 0,075 & 0,083 & 11 & 0,054\pm0,001 & 28\\
2 & 0,15  & 0,125 & 17 & 0,119\pm0,003 & 21\\
3 & 0,4   & 0,375 & 6  & 0,072\pm0,002 & 82\\
    \bottomrule
  \end{tabular}
\end{table}

\begin{table}
  \centering
  \caption{Abweichung der Messwerte von den Herstellerangaben für den Doppelspalt.}
  \label{tab:abweich2}
  \begin{tabular}{c c |c c c c| c c c c}
 \multicolumn{2}{c}{Herstellerangaben} & \multicolumn{4}{c}{Mikroskopmessung} &\multicolumn{4}{c}{Über Begungsfiguren} \\
$b/\si{\milli\meter}$& $s/\si{\milli\meter}$ & $b/\si{\milli\meter}$ & $a/\si{\percent}$& $s/\si{\milli\meter}$& $a/\si{\percent}$ &$b/\si{\milli\meter}$ & $a/\si{\percent}$ & $s/\si{\milli\meter}$ & $a/\si{\percent}$  \\
    \midrule
  0,1 & 0,4 & 0,1875 & 88 & 0,4 & 0 & 0,068\pm0,005 & 32 & 0,446\pm0,004  & 12\\
    \bottomrule
  \end{tabular}
\end{table}
Durch die Tabellen wird deutlich, dass die Messung der Spaltbreite $b$ über ein Mikroskop
bei allen Messungen genauer ist, als die über die Beugungsbilder.
Dies liegt zu einem an Ablesefehler beim aufnehemen des
Stromes $I$ und den häufigen wechselnden Messskalen.
Wechselde Lichtverhältnisse bei der Messung könnten ebenfalls zur Messungenauhigkeiten führen.
Ebenfalls weichen mache Messwerte aus den Abblidungnen \ref{fig:spalt1}-\ref{fig:dopp} deulich von dem Fitts ab.
Bei der Messung des Doppelspaltes sind die Messintervalle zu größ gewählt worden, da
da der charakatristische Verlauf nicht aus den Messwerten zu erkennen ist.
Alles im allen kann also gesagt werden, dass die Vermessung des Beugungsbilder zur Bestimmung des Spaltbreite sich
nicht eigenet, da zu viele Fehlerquellen bei der Messung existieren.
