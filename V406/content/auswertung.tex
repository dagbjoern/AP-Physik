\section{Auswertung}
\label{sec:Auswertung}
\subsection{Messung mit Mikrosop}
Zunächt wird mit einem Mikroskop die Spaltbreite der verwendenten
Spalte bestimmt. Die gemessenen Werte sind in der Tabelle \ref{tab:cool}
\begin{table}
  \centering
  \caption{Breite der Spalten für die Messung mit dem Mikroskop.}
  \label{tab:cool}
  \begin{tabular}{c c c c c}
    \toprule
  Spalt 1 & Spalt 2  & Spalt 3 &  \multicolumn{2}{c}{Doppelspalt}\\
  Spaltbreite  &  Spaltbreite  & Spaltbreite  & Spaltbreite  & Spaltabstand \\
    \midrule
   0,083\,\si{\milli\meter}   &  0,125\si{\milli\meter}& 0,375\si{\milli\meter}& 0,1875\si{\milli\meter}& 0,4\si{\milli\meter}\\
\bottomrule
\end{tabular}
\end{table}
\subsection{Messung über Beugungsfiguren}
\subsection{Einzelspalt}
\label{sec:einzel}

Zu Beginn der Messung der Spaltbreite über die Beugungsfiguren wird der Dunkelstrom vermessen, dieser beträgt:
\begin{align*}
  I_\mathrm{dunkel}= 0,225\si{\nano\ampere}
\end{align*}
und wird von dem gemessenen Storm $I$ abgezogen.
Desweiteren wird der Abstand $L$ vom Spalt zum Schirm benötigt:
\begin{align*}
  L=0,9\si{\meter}.
\end{align*}
Es wird ein Helium-Neon-Laser verwendet, dieser besitzt eine Wellenlänge $\lambda$ von:
\begin{align*}
  \lambda=633\si{\nano\meter}.
\end{align*}


In der Tabelle \ref{tab:spalt1} sind die gemessen Werte
des ersten Spaltes aufgetragen.
\begin{table}
  \centering
  \caption{Messwerte des ersten Einzelspalt.}
  \label{tab:spalt1}
  \begin{tabular}{c c c c}
Abstand $d$/$\si{\milli\meter}$ & Strom $I-I_\mathrm{dunkel}/\si{\nano\ampere}$ & Abstand $d$/$\si{\milli\meter}$ & Strom $I-I_\mathrm{dunkel}/\si{\nano\ampere}$\\
    \midrule
     0 & 255,8 & -1   & 219,8\\
     1 & 242,3 & -2   & 189,8\\
     2 & 225,3 & -3   & 159,8\\
     3 & 199,8 & -4   & 119,8\\
     4 & 162,3 & -5   & 79,8 \\
     5 & 124,8 & -6   & 59,8 \\
     6 & 76,0  & -7   & 39,8 \\
     7 & 52,3  & -8   & 19,8 \\
     8 & 29,9  & -9   & 9,8  \\
     9 & 13,8  & -10  & 3,8  \\
    10 & 4,8   & -11  & 2,7  \\
    11 & 2,8   & -12  & 2,8  \\
    12 & 3,4   & -13  & 3,8  \\
    13 & 5,6   & -14  & 4,8  \\
    14 & 7,6   & -15  & 5,0  \\
    15 & 8,8   & -16  & 4,7  \\
    16 & 8,6   & -17  & 4,0  \\
    17 & 7,2   & -18  & 2,9  \\
    18 & 5,4   & -19  & 2,0  \\
    19 & 3,6   & -20  & 1,3  \\
    20 & 2,2   & -21  & 0,8  \\
    21 & 1,4   & -22  & 0,6  \\
    22 & 1,0   & -23  & 0,5  \\
    23 & 1,2   & -24  & 0,4  \\
    24 & 1,4   & -25  & 0,4  \\
    25 & 1,5   & &           \\
    \bottomrule
    \end{tabular}
\end{table}
\FloatBarrier
Die Messwerte aus Tabelle \ref{tab:spalt1} werden nun
in der Form $I$ in Abhängigkeit von $\phi$
aufgetragen.
Wobei $\phi$ sich durch
\begin{align*}
  \phi\approx\frac{d}{L}
\end{align*}
nähren lässt.

\begin{figure}
  \centering
  \includegraphics[width=0.7\textwidth]{spalt1.pdf}
  \caption{ Der Strom $I$ in Abhängigkeit von dem Winkel $\phi$. des 1. Spaltes.}
  \label{fig:spalt1}
\end{figure}
\FloatBarrier
Nun wird versucht durch die Formel \eqref{eqn:einzel} an die Messwerte zu fitten.
Dadurch ergeben sich folgende Parameter.
\begin{align*}
  A_0&=9,0\pm0,1\\
  \text{Die Spaltbreite:} \ b&=(0,054\pm0,001)\si{\milli\meter}\\
\end{align*}

Dieses Vorgehen wird nun für zwei weitere Einzelspalte wiederholt.
Die Tabelle \ref{tab:spalt2} enthält die Messwerte des zweiten Spaltes
und die Tabelle \ref{tab:spalt3} die des dritten Spaltes.




\begin{table}
  \centering
  \caption{Messwerte des zweiten Einzelspalt.}
  \label{tab:spalt2}
  \begin{tabular}{c c c c}
Abstand $d$/$\si{\milli\meter}$ & Strom $I-I_\mathrm{dunkel}/\si{\micro\ampere}$&Abstand $d$/$\si{\milli\meter}$ & Strom $I-I_\mathrm{dunkel}/\si{\micro\ampere}$\\
  \midrule
  0  & 2124,8 & -1  & 2499,8\\
  1  & 1799,8 & -2  & 1504,8\\
  2  & 1299,8 & -3  & 509,8 \\
  3  & 509,8  & -4  & 149,8 \\
  4  & 149,8  & -5  & 15,8  \\
  5  &  18,8  & -6  & 43,8  \\
  6  & 38,8   & -7  & 80,8  \\
  7  & 61,8   & -8  & 61,8  \\
  8  & 39,8   & -9  & 21,8  \\
  9  & 11,8   & -10 & 6,8   \\
  10 & 5,8    & -11 & 17,8\\
  11 & 15,8   & -12 & 27,8\\
  12 & 19,8   & -13 & 21,8\\
  13 & 11,8   & -14 & 9,8     \\
  14 & 4,8    & -15 & 7,8     \\
  15 & 5,6    & -16 & 8,6     \\
  16 & 8,0    & -17 & 9,8   \\
  17 & 6,6    & -18 & 7,3   \\
  18 & 4,3    & -19 & 3,8   \\
  19 & 2,6    & -20 & 2,8       \\
  20 & 4,1    & -21 & 3,6       \\
  21 & 4,5    & -22 & 3,6       \\
  22 & 2,9    & -23 & 2,3     \\
  23 & 1,8    & -24 & 1,6     \\
  24 & 2,2    & -25 & 1,8 \\
  25 & 2,8    & &\\
  \bottomrule
  \end{tabular}
\end{table}
\FloatBarrier
\begin{table}
  \centering
  \caption{Messwerte des dritten Einzelspalt.}
  \label{tab:spalt3}
  \begin{tabular}{c c c c}
Abstand $d$/$\si{\milli\meter}$ & Strom $I-I_\mathrm{dunkel}/\si{\micro\ampere}$&Abstand $d$/$\si{\milli\meter}$ & Strom $I-I_\mathrm{dunkel}/\si{\micro\ampere}$\\
  \midrule
  0  & 1509,8 & -1 & 439,8\\
  1  & 539,8  & -2 & 459,8\\
  2  & 359,8  & -3 & 209,8\\
  3  & 259,8  & -4 & 279,8\\
  4  & 219,8  & -5 &  79,8\\
  5  & 73,8   & -6 &  69,8\\
  6  & 67,8   & -7 &  59,8\\
  7  & 53,8   & -8 &  41,8\\
  8  & 31,8   & -9 &  40,8\\
  9  & 50,8   & -10 & 30,8\\
  19 & 22,8   & -11 & 27,8\\
  11 & 21,0   & -12 & 15,6\\
  12 & 20,3   & -13 & 10,3\\
  13 & 13,5   & -14 & 19,3\\
  14 & 14,8   & -15 & 11,8\\
  15 & 9,5    & -16 &  8,8\\
  16 & 7,6    & -17 &  9,2\\
  17 & 7,6    & -18 &  7,2\\
  18 & 5,4    & -19 &  8,6\\
  19 & 6,4    & -20 &  7,0\\
  20 & 8,6    & -21 &  4,2\\
  21 & 4,0    & -22 &  5,0\\
  22 & 6,0    & -23 &  4,2\\
  23 & 3,6    & -24 &  4,0\\
  24 & 3,2    & -25 &  4,2\\
  25 & 3,1    &     &     \\
  \bottomrule
  \end{tabular}
\end{table}
\FloatBarrier

In den Abblidung \ref{fig:spalt2} und \ref{fig:spalt3} sind
die Messwerte des entsprechenden Spaltes wie zuvor beim Spalt 1 beschrieben
gegen einander aufgetragen.


\begin{figure}
  \centering
  \includegraphics[width=0.7\textwidth]{spalt1.pdf}
  \caption{ Der Strom $I$ in Abhängigkeit von dem Winkel $\phi$ von dem 2. Spalt.}
 \label{fig:spalt2}
\end{figure}
\FloatBarrier

\begin{figure}
  \centering
  \includegraphics[width=0.7\textwidth]{spalt3.pdf}
  \caption{ Der Strom $I$ in Abhängigkeit von dem Winkel $\phi$ von dem 3. Spalt.}
  \label{fig:spalt3}
\end{figure}
\FloatBarrier
Ebenfall wird wieder die Funktion \eqref{eqn:einzel}
an die Messwerte gefittet und es ergeben sich die folgenden Paramter.
\begin{align*}
  \text{Für den 2. Spalt :}\\
  A_0&=13,3\pm0,3\\
\text{Die Spaltbreite:} \ b&=(0,119\pm0,003)\si{\milli\meter}\\
\\
\text{Für den 3. Spalt:}\\
  A_0&=9,8\pm0,3\\
  \text{Die Spaltbreite}\ b&=(0,07\pm0,002)\si{\milli\meter}\\
\end{align*}

\subsection{Doppel-Spalt}
Hierbei wird der Laser an einem Doppel-Spalt gebeugt. Dies Messwerte
sind in der Tabelle \ref{tab:dopp} zu finden
nd werden in der Abllidung \ref{fig:dopp} wie in dem Kapitel \ref{sec:einzel} gegen einander aufgetragen.
\begin{table}
  \centering
  \caption{Messwerte des doppel Spaltes.}
  \label{tab:dopp}
  \begin{tabular}{c c c c}
Abstand $d$/$\si{\milli\meter}$ & Strom $I-I_\mathrm{dunkel}/\si{\micro\ampere}$ & Abstand $d$/$\si{\milli\meter}$ & Strom $I-I_\mathrm{dunkel}/\si{\micro\ampere}$\\
    \midrule
    0,0 & 1374,8 & 24,5 & 1,1  \\
    0,5 & 949,8  & 25,0 & 0,9  \\
    1,0 & 1099,8 & -0,5 & 489,8\\
    1,5 & 1149,8 & -1,0 & 469,8\\
    2,0 & 749,8  & -1,5 & 569,8\\
    2,5 & 1074,8 & -2,0 & 349,8\\
    3,0 & 599,8  & -2,5 & 489,8\\
    3,5 & 649,8  & -3,0 & 319,8\\
    4,0 & 599,8  & -3,5 & 299,8\\
    4,5 & 324,8  & -4,0 & 299,8\\
    5,0 & 419,8  & -4,5 & 159,8\\
    5,5 & 199,8  & -5,0 & 189,8\\
    6,0 & 169,8  & -5,5 & 109,8\\
    6,5 & 119,8  & -6,0 & 79,8 \\
    7,0 & 47,8   & -6,5 & 59,8 \\
    7,5 & 43,8   & -7,0 & 19,8 \\
    8,0 & 13,8   & -7,5 & 21,8 \\
    8,5 & 9,6    & -8,0 & 11,8 \\
    9,0 & 7,6    & -8,5 & 5,6  \\
    9,5 & 19,0   & -9,0 & 4,8  \\
    10,0 & 21,8  & -9,5 & 7,9  \\
    10,5 & 25,8  & -10,0 & 9,8 \\
    11,0 & 43,8  & -10,5 & 9,8 \\
    11,5 & 30,8  & -11,0 & 17,8 \\
    12,0 & 41,8  & -11,5 & 14,8 \\
    12,5 & 40,8  & -12,0 & 15,8 \\
    13,0 & 26,8  & -12,5 & 19,8 \\
    13,5 & 35,8  & -13,0 & 9,8 \\
    14,0 & 19,0  & -13,5 & 13,8 \\
    14,5 & 17,3  & -14,0 & 9,8  \\
    15,0 & 11,8  & -14,5 & 5,8  \\
    15,5 & 3,1   & -15,0 & 6,8  \\
    16,0 & 3,0   & -15,5 & 2,8  \\
    16,5 & 2,8   & -16,0 & 3,0  \\
    17,0 & 3,2   & -16,5 & 2,0  \\
    17,5 & 3,0   & -17,0 & 2,3  \\
    18,0 & 4,0   & -17,5 & 2,8  \\
    18,5 & 3,8   & -18,0 & 2,4  \\
    19,0 & 3,8   & -18,5 & 4,4  \\
    19,5 & 4,47  & -19,0 & 3,0  \\
    20,0 & 3,2   & -19,5 & 4,4  \\
    20,5 & 3,8   & -20,0 & 4,2  \\
    21,0 & 3,0   & -20,5 & 2,6  \\
    21,5 & 2,5   & -21,0 & 3,8  \\
    \bottomrule
    \end{tabular}
\end{table}
\FloatBarrier
\begin{figure}
  \centering
  \includegraphics[width=0.7\textwidth]{doppelspalt.pdf}
  \caption{ Der Strom $I$ in Abhängigkeit von dem Winkel $\phi$. des Doppelspaltes.}
  \label{fig:dopp}
\end{figure}
\FloatBarrier
Ebenfalls wird nun Versuch eine Funktion an die gemessenen Werte zu fitten.
Für den Doppelspalt wird die Funktion \eqref{eqn:dopp} verwendet.
Es ergebenden sich die folgenden
Parameter:
\begin{align*}
  A_0&=0,00116\pm0,00005\\
  \intertext{Für den Abstand der beiden Spalte:}
    s&=(0,458\pm0,004)\si{\milli\meter}\\
  \intertext{und die Spaltbreite:}
    b&=(0,068\pm0,005)\si{\milli\meter}\\
\end{align*}
