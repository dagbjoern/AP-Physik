\section{Auswertung}
\label{sec:Auswertung}
\subsection{Messung mit Mikrosop}
Zunächt wird mit einem Mikroskop die Spaltbreite der verwendenten
Spalte bestimmt. Die gemessenen Werte sind in der Tabelle \ref{tab:cool}
\begin{table}
  \centering
  \caption{Breite der Spalten für die Messung mit dem Mikroskop.}
  \label{tab:cool}
  \begin{tabular}{c c c c c}
    \toprule
  Spalt 1 & Spalt 2  & Spalt 3 &  \multicolumn{2}{c}{Doppelspalt}\\
  Spaltbreite  &  Spaltbreite  & Spaltbreite  & Spaltbreite  & Spaltabstand \\
    \midrule
   0,083\,\si{\milli\meter}   &  0,125\si{\milli\meter}& 0,375\si{\milli\meter}& 0,1875\si{\milli\meter}& 0,4\si{\milli\meter}\\
\bottomrule
\end{tabular}
\end{table}

\subsection{Messung über Beugungsfiguren}
\subsection{Einzelspalt}
\label{sec:einzel}
Zu Beginn der Messung der Spaltbreite über die Beugungsfiguren wird der Dunkelstrom vermessen, dieser beträgt:
\begin{align*}
  I_\mathrm{dunkel}= 0,225\si{\nano\ampere}
\end{align*}
und wird von dem gemessenen Storm $I$ abgezogen.
Desweiteren wird der Abstand $L$ vom Spalt zum Schirm benötigt:
\begin{align*}
  L=0,9\si{\meter}.
\end{align*}
Es wird ein Helium-Neon-Laser verwendet, dieser besitzt eine Wellenlänge $\lambda$ von:
\begin{align*}
  \lambda=633\si{\nano\meter}.
\end{align*}

In der Tabelle \ref{tab:spalt} sind die gemessen Werte
für die ersten drei Spalte aufgetragen.
\newpage
\begin{center}
\begin{longtable}{c c c c }
  \caption{Die Messwerte für die ersten drei Spalte .}
  \label{tab:spalt}\\
  \toprule
      & Spalt 1 & Spalt 2 & Spalt 3\\
  Abstand $d/\si{\milli\meter}$ & Strom $I/\si{\nano\ampere}$ & Strom $I/\si{\nano\ampere}$   & Strom $I/\si{\nano\ampere}$     \\
  \midrule
  \endfirsthead
  \toprule
  & Spalt 1 & Spalt 2 & Spalt 3\\
Abstand $d/\si{\milli\meter}$ & Strom $I/\si{\nano\ampere}$ & Strom $I/\si{\nano\ampere}$   & Strom $I/\si{\nano\ampere}$     \\
  \midrule
  \endhead
  \bottomrule
  \endfoot
  0,0 & 254,8  & 2124,8 & 1499,8\\
  1,0 & 242,3  & 1799,8 &  539,8\\
  2,0 & 225,31 & 1199,8 &  359,8\\
  3,0 & 199,8  &  509,8 &  259,8\\
  4,0 & 162,3  &  149,8 &  219,8\\
  5,0 & 124,8  &   17,8 &   73,8\\
  6,0 & 76,0   &   35,8 &   67,8\\
  7,0 & 52,3   &   60,8 &   53,8\\
  8,0 & 29,8   &   39,8 &   31,8\\
  9,0 & 13,8   &   11,8 &   50,8\\
  10,0 & 4,8   &   05,8 &   22,8\\
  11,0 & 2,8   &   15,8 &   21,1\\
  12,0 & 3,4   &   19,8 &   20,3\\
  13,0 & 5,6   &   11,8 &   13,5\\
  14,0 & 7,6   &    4,8 &   14,8\\
  15,0 & 8,8   &    5,6 &    9,5\\
  16,0 & 8,6   &    8,0 &    7,6\\
  17,0 & 7,2   &    6,6 &    7,6\\
  18,0 & 5,4   &    4,3 &    5,5\\
  19,0 & 3,6   &    2,6 &    6,4\\
  20,0 & 2,2   &    4,1 &    8,6\\
  21,0 & 1,4   &    4,5 &    4,0\\
  22,0 & 1,0   &    2,9 &    6,0\\
  23,0 & 1,2   &    1,8 &    3,6\\
  24,0 & 1,4   &    2,2 &    3,2\\
  25,0 & 1,5   &    2,8 &    3,1\\
  -1,0 & 219,8 & 2499,8 &  439,8\\
  -2,0 & 189,8 & 1504,8 &  459,8\\
  -3,0 & 159,8 &  509,8 &  209,8\\
  -4,0 & 119,8 &  149,8 &  279,8\\
  -5,0 & 79,8  &   15,8 &   79,8\\
  -6,0 & 59,8  &   43,8 &   69,8\\
  -7,0 & 39,8  &   80,8 &   59,8\\
  -8,0 & 19,8  &   61,8 &   41,8\\
  -9,0 & 9,8   &   21,8 &   40,8\\
  -10,0 & 3,8  &    6,8 &   30,8\\
  -11,0 & 2,8  &   17,8 &   27,8\\
  -12,0 & 2,8  &   27,8 &   15,6\\
  -13,0 & 3,8  &   21,8 &   10,3\\
  -14,0 & 4,8  &   9,8 &    19,3\\
  -15,0 & 5,0  &   7,8 &    11,8\\
  -16,0 & 4,7  &   8,6 &     8,8\\
  -17,0 & 4,0  &   9,8 &     9,2\\
  -18,0 & 2,9  &   7,3 &     7,2\\
  -19,0 & 2,0  &   3,8 &     8,6\\
  -20,0 & 1,3  &   2,8 &     7,0\\
  -21,0 & 0,8  &   3,6 &     4,2\\
  -22,0 & 0,6  &   3,6 &     5,0\\
  -23,0 & 0,5  &   2,3 &     4,2\\
  -24,0 & 0,4  &   1,6 &     4,0\\
  -25,0 & 0,4  &   1,8 &     4,2\\
  \end{longtable}
  \end{center}












%
%
%
% \begin{table}
%   \centering
%   \caption{Messwerte des ersten Einzelspalt.}
%   \label{tab:spalt1}
%   \begin{tabular}{c c c c}
% Abstand $d$/$\si{\milli\meter}$ & Strom $I-I_\mathrm{dunkel}/\si{\nano\ampere}$ & Abstand $d$/$\si{\milli\meter}$ & Strom $I-I_\mathrm{dunkel}/\si{\nano\ampere}$\\
%     \midrule
%      0 & 255,8 & -1   & 219,8\\
%      1 & 242,3 & -2   & 189,8\\
%      2 & 225,3 & -3   & 159,8\\
%      3 & 199,8 & -4   & 119,8\\
%      4 & 162,3 & -5   & 79,8 \\
%      5 & 124,8 & -6   & 59,8 \\
%      6 & 76,0  & -7   & 39,8 \\
%      7 & 52,3  & -8   & 19,8 \\
%      8 & 29,9  & -9   & 9,8  \\
%      9 & 13,8  & -10  & 3,8  \\
%     10 & 4,8   & -11  & 2,7  \\
%     11 & 2,8   & -12  & 2,8  \\
%     12 & 3,4   & -13  & 3,8  \\
%     13 & 5,6   & -14  & 4,8  \\
%     14 & 7,6   & -15  & 5,0  \\
%     15 & 8,8   & -16  & 4,7  \\
%     16 & 8,6   & -17  & 4,0  \\
%     17 & 7,2   & -18  & 2,9  \\
%     18 & 5,4   & -19  & 2,0  \\
%     19 & 3,6   & -20  & 1,3  \\
%     20 & 2,2   & -21  & 0,8  \\
%     21 & 1,4   & -22  & 0,6  \\
%     22 & 1,0   & -23  & 0,5  \\
%     23 & 1,2   & -24  & 0,4  \\
%     24 & 1,4   & -25  & 0,4  \\
%     25 & 1,5   & &           \\
%     \bottomrule
%     \end{tabular}
% \end{table}
% \FloatBarrier
Die Messwerte aus Tabelle \ref{tab:spalt} werden nun für jeden Spalt
in der Form $I$ in Abhängigkeit von $\phi$
aufgetragen.
Wobei $\phi$ sich durch
\begin{align*}
  \phi\approx\frac{d}{L}
\end{align*}
nähren lässt.

In den Abbildung \ref{fig:spalt1}-\ref{fig:spalt3} sind
die Messwerte des entsprechenden Spaltes wie zuvor beschrieben
gegen einander aufgetragen.
%
% \begin{table}
%   \centering
%   \caption{Messwerte des zweiten Einzelspalt.}
%   \label{tab:spalt2}
%   \begin{tabular}{c c c c}
% Abstand $d$/$\si{\milli\meter}$ & Strom $I-I_\mathrm{dunkel}/\si{\micro\ampere}$&Abstand $d$/$\si{\milli\meter}$ & Strom $I-I_\mathrm{dunkel}/\si{\micro\ampere}$\\
%   \midrule
%   0  & 2124,8 & -1  & 2499,8\\
%   1  & 1799,8 & -2  & 1504,8\\
%   2  & 1299,8 & -3  & 509,8 \\
%   3  & 509,8  & -4  & 149,8 \\
%   4  & 149,8  & -5  & 15,8  \\
%   5  &  18,8  & -6  & 43,8  \\
%   6  & 38,8   & -7  & 80,8  \\
%   7  & 61,8   & -8  & 61,8  \\
%   8  & 39,8   & -9  & 21,8  \\
%   9  & 11,8   & -10 & 6,8   \\
%   10 & 5,8    & -11 & 17,8\\
%   11 & 15,8   & -12 & 27,8\\
%   12 & 19,8   & -13 & 21,8\\
%   13 & 11,8   & -14 & 9,8     \\
%   14 & 4,8    & -15 & 7,8     \\
%   15 & 5,6    & -16 & 8,6     \\
%   16 & 8,0    & -17 & 9,8   \\
%   17 & 6,6    & -18 & 7,3   \\
%   18 & 4,3    & -19 & 3,8   \\
%   19 & 2,6    & -20 & 2,8       \\
%   20 & 4,1    & -21 & 3,6       \\
%   21 & 4,5    & -22 & 3,6       \\
%   22 & 2,9    & -23 & 2,3     \\
%   23 & 1,8    & -24 & 1,6     \\
%   24 & 2,2    & -25 & 1,8 \\
%   25 & 2,8    & &\\
%   \bottomrule
%   \end{tabular}
% \end{table}
% \FloatBarrier
% \begin{table}
%   \centering
%   \caption{Messwerte des dritten Einzelspalt.}
%   \label{tab:spalt3}
%   \begin{tabular}{c c c c}
% Abstand $d$/$\si{\milli\meter}$ & Strom $I-I_\mathrm{dunkel}/\si{\micro\ampere}$&Abstand $d$/$\si{\milli\meter}$ & Strom $I-I_\mathrm{dunkel}/\si{\micro\ampere}$\\
%   \midrule
%   0  & 1509,8 & -1 & 439,8\\
%   1  & 539,8  & -2 & 459,8\\
%   2  & 359,8  & -3 & 209,8\\
%   3  & 259,8  & -4 & 279,8\\
%   4  & 219,8  & -5 &  79,8\\
%   5  & 73,8   & -6 &  69,8\\
%   6  & 67,8   & -7 &  59,8\\
%   7  & 53,8   & -8 &  41,8\\
%   8  & 31,8   & -9 &  40,8\\
%   9  & 50,8   & -10 & 30,8\\
%   19 & 22,8   & -11 & 27,8\\
%   11 & 21,0   & -12 & 15,6\\
%   12 & 20,3   & -13 & 10,3\\
%   13 & 13,5   & -14 & 19,3\\
%   14 & 14,8   & -15 & 11,8\\
%   15 & 9,5    & -16 &  8,8\\
%   16 & 7,6    & -17 &  9,2\\
%   17 & 7,6    & -18 &  7,2\\
%   18 & 5,4    & -19 &  8,6\\
%   19 & 6,4    & -20 &  7,0\\
%   20 & 8,6    & -21 &  4,2\\
%   21 & 4,0    & -22 &  5,0\\
%   22 & 6,0    & -23 &  4,2\\
%   23 & 3,6    & -24 &  4,0\\
%   24 & 3,2    & -25 &  4,2\\
%   25 & 3,1    &     &     \\
%   \bottomrule
%   \end{tabular}
% \end{table}
% \FloatBarrier

\begin{figure}
  \centering
  \includegraphics[width=0.7\textwidth]{spalt1.pdf}
  \caption{ Der Strom $I$ in Abhängigkeit von dem Winkel $\phi$. des 1. Spaltes.}
  \label{fig:spalt1}
\end{figure}
\FloatBarrier

\begin{figure}
  \centering
  \includegraphics[width=0.7\textwidth]{spalt1.pdf}
  \caption{ Der Strom $I$ in Abhängigkeit von dem Winkel $\phi$ von dem 2. Spalt.}
  \label{fig:spalt2}
\end{figure}
\FloatBarrier

\begin{figure}
  \centering
  \includegraphics[width=0.7\textwidth]{spalt3.pdf}
  \caption{ Der Strom $I$ in Abhängigkeit von dem Winkel $\phi$ von dem 3. Spalt.}
  \label{fig:spalt3}
\end{figure}
\FloatBarrier

Nun wird Versucht durch die Formel \eqref{eqn:einzel} an die Messwerte zu fitten.
Dadurch ergeben sich folgende Parameter, wobei b die Spaltbreite ist.
\begin{align*}
  \intertext{Für den ersten Spalt:}
  A_0&=9,0\pm0,1\\
 b&=(0,054\pm0,001)\si{\milli\meter}\\
\intertext{Für den 2. Spalt :}\\
  A_0&=13,3\pm0,3\\
 b&=(0,119\pm0,003)\si{\milli\meter}\\
\intertext{Für den 3. Spalt:}\\
  A_0&=9,8\pm0,3\\
 b&=(0,07\pm0,002)\si{\milli\meter}\\
\end{align*}

\subsubsection{Doppel-Spalt}
Hierbei wird der Laser an einem Doppel-Spalt gebeugt. Dies Messwerte
sind in der Tabelle \ref{tab:dopp} zu finden
nd werden in der Abllidung \ref{fig:dopp} wie in dem Kapitel \ref{sec:einzel} gegen einander aufgetragen.
\begin{table}
  \centering
  \caption{Messwerte des doppel Spaltes.}
  \label{tab:dopp}
  \begin{tabular}{c c c c}
Abstand $d$/$\si{\milli\meter}$ & Strom $I-I_\mathrm{dunkel}/\si{\micro\ampere}$ & Abstand $d$/$\si{\milli\meter}$ & Strom $I-I_\mathrm{dunkel}/\si{\micro\ampere}$\\
    \midrule
    0,0 & 1374,8 & 24,5 & 1,1  \\
    0,5 & 949,8  & 25,0 & 0,9  \\
    1,0 & 1099,8 & -0,5 & 489,8\\
    1,5 & 1149,8 & -1,0 & 469,8\\
    2,0 & 749,8  & -1,5 & 569,8\\
    2,5 & 1074,8 & -2,0 & 349,8\\
    3,0 & 599,8  & -2,5 & 489,8\\
    3,5 & 649,8  & -3,0 & 319,8\\
    4,0 & 599,8  & -3,5 & 299,8\\
    4,5 & 324,8  & -4,0 & 299,8\\
    5,0 & 419,8  & -4,5 & 159,8\\
    5,5 & 199,8  & -5,0 & 189,8\\
    6,0 & 169,8  & -5,5 & 109,8\\
    6,5 & 119,8  & -6,0 & 79,8 \\
    7,0 & 47,8   & -6,5 & 59,8 \\
    7,5 & 43,8   & -7,0 & 19,8 \\
    8,0 & 13,8   & -7,5 & 21,8 \\
    8,5 & 9,6    & -8,0 & 11,8 \\
    9,0 & 7,6    & -8,5 & 5,6  \\
    9,5 & 19,0   & -9,0 & 4,8  \\
    10,0 & 21,8  & -9,5 & 7,9  \\
    10,5 & 25,8  & -10,0 & 9,8 \\
    11,0 & 43,8  & -10,5 & 9,8 \\
    11,5 & 30,8  & -11,0 & 17,8 \\
    12,0 & 41,8  & -11,5 & 14,8 \\
    12,5 & 40,8  & -12,0 & 15,8 \\
    13,0 & 26,8  & -12,5 & 19,8 \\
    13,5 & 35,8  & -13,0 & 9,8 \\
    14,0 & 19,0  & -13,5 & 13,8 \\
    14,5 & 17,3  & -14,0 & 9,8  \\
    15,0 & 11,8  & -14,5 & 5,8  \\
    15,5 & 3,1   & -15,0 & 6,8  \\
    16,0 & 3,0   & -15,5 & 2,8  \\
    16,5 & 2,8   & -16,0 & 3,0  \\
    17,0 & 3,2   & -16,5 & 2,0  \\
    17,5 & 3,0   & -17,0 & 2,3  \\
    18,0 & 4,0   & -17,5 & 2,8  \\
    18,5 & 3,8   & -18,0 & 2,4  \\
    19,0 & 3,8   & -18,5 & 4,4  \\
    19,5 & 4,47  & -19,0 & 3,0  \\
    20,0 & 3,2   & -19,5 & 4,4  \\
    20,5 & 3,8   & -20,0 & 4,2  \\
    21,0 & 3,0   & -20,5 & 2,6  \\
    21,5 & 2,5   & -21,0 & 3,8  \\
    \bottomrule
    \end{tabular}
\end{table}
\FloatBarrier
\begin{figure}
  \centering
  \includegraphics[width=0.7\textwidth]{doppelspalt.pdf}
  \caption{ Der Strom $I$ in Abhängigkeit von dem Winkel $\phi$. des Doppelspaltes.}
  \label{fig:dopp}
\end{figure}
\FloatBarrier
Ebenfalls wird nun Versuch eine Funktion an die gemessenen Werte zu fitten.
Für den Doppelspalt wird die Funktion \eqref{eqn:dopp} verwendet.
Es ergebenden sich die folgenden
Parameter:
\begin{align*}
  A_0&=0,00116\pm0,00005\\
  \intertext{Für den Abstand der beiden Spalte:} \  s&=(0,458\pm0,004)\si{\milli\meter}\\
  \intertext{und die Spaltbreite:} \  b&=(0,068\pm0,005)\si{\milli\meter}\\
\end{align*}
