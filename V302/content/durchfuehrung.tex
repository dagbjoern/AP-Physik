\section{Durchführung}
\label{sec:Durchführung}
Alle Brücken, außer der TT-Brücke, werden mit Wechselstrom der Frequenz $700\si{\hertz}$ betrieben.
Als Nullindikator dient ein Osziloskop, mit einem vorgeschalteten Tiefpass werden Störspannungen rausgefiltert.
\subsection{Messung zweier unbekannter Widerstände}
Die Wheatstone-Brücke wird nach Abbidung \ref{fig:wheat} aufgebaut. Ziel ist die Bestimmung zweier unbekannter Widerstände (Wert 11 und Wert 14).
Als Abstimmvorrichtung dient ein Zehngang-Präzisionspotentiometer mit $1\si{\kilo\ohm}$ Gesamtwiderstand, damit kann die Brückenspannung auf Null geregelt werden.
Vermessen werden die Widerstände $R_\mathrm{3}$ und $R_\mathrm{4}$ mit drei verschiedenen $R_\mathrm{2}$.
\subsection{Messung zweier unbekannter Kapazitäten und einer RC-Kombination}
Die Kapazitätsmessbrücke wird nach Abbildung \ref{fig:kapazitaet} aufgebaut.
Ziel ist die Bestimmung zweier unbekannter Kapazitäten (Wert 1 und Wert 3) und die Kapazität eines Kondensators mit Verlustwiderstand (Wert 8).
Vermessen werden die Widerstände $R_\mathrm{3}$ und $R_\mathrm{4}$ mit drei verschiedenen $C_\mathrm{2}$.
Zur Bestimmung von Wert 8 wird zusätzlich ein regulierbares Stellglied $R_\mathrm{2}$ eingebaut und vermessen.
\subsection{Messung einer Induktivität mit Verlustwiderstand}
Die Induktivitätsmessbrücke wird nach Abbildung \ref{fig:induktivitaet} aufgebaut.
Ziel ist die Bestimmung einer Induktivität mit Verlustwiederstand (Wert 17).
Vermessen werden $R_\mathrm{3}$, $R_\mathrm{4}$ und $R_\mathrm{2}$ mit drei verschiedenen $L_\mathrm{2}$.
$R_\mathrm{2}$ dient wieder als regulierbares Stellglied.\\
Die unbekannte Induktivität kann auch mittels Maxwell-Brücke bestimmt werden.
Diese wird nach Abbildung \ref{fig:maxwell} aufgebaut.
Vermessen werden $R_\mathrm{3}$, $R_\mathrm{4}$ mit drei verschiedenen $R_\mathrm{2}$.
$C_\mathrm{4}$ ist eine verlustarme Kapazität und wird während der Messung nict variiert.
\subsection{Bestimmung der Frequenzabhängigkeit der Brückenspannung}
Für die Bestimmung der Frequenzabhängigkeit wird eine TT-Brücke wie in Abbildung \ref{fig:tt}
aufgebaut. Vermessen werden die Brückenspannung im Bereich zwischen $20\si{\hertz}$ und $30000\si{\hertz}$ und die Speisespannung.
