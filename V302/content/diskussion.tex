\section{Diskussion}
\label{sec:Diskussion}
Da bei dem Versuch keine Referenzangaben für die
einzelnen gemessenen Wiederstände, Kapazitäten und Induktivitäten
gegenben sind, kann nur auf die Abweichung der gemessenen
Werte,
die durch die Variation der einzelnen Bauteile entsteht,
eingegangen werden.
Bei der Messung, die mit der Wheatstoneschen Brücke durchgeführt
wird, liegen die gemessenen Wiederstände nah beieinander.
Daruch kann gesagt werden, dass sich diese Brücke zum
Messen eines Widerstandes eignet.
Dies gilt auch für die Kapazitätsmessbrücke und für
die Induktivitätsmessbrücke.
Jedoch gibt es bei der Maxwell-Brücke, wo der Widerstand
und die Induktivität der selben
RL-Kombination, wie bei der Induktivitätsmessbrücke,
bestimmt wird, deutliche Abweichungen von den zuvor
bestimmten Werten bei der Induktivitätsmessbrücke.
Folglich können durch die Messung mit Hilfe der Maxwell-Brücke
die Formeln aus der Theorie \ref{sec:Max} nicht bestätigt werden.
Die Messwerte, die mit der TT-Brück bestimmt wurden,
liegen im Bereich des Tiefpunktes nahe an der Theoriekurve und
das aus der Messung bestimmte $\nu_0$ weicht nur
um $3,7\,\si{\percent}$ von dem theoretischen Wert ab.
Nur bei höherern Frequenzen kommt es zu größeren Abweichungen von der
Theoriekurve.
Da dies nicht in die Berechung des Klirrfaktors
eingeht, führt dies nicht zu größeren Ungenauigkeiten.
Jedoch kann keine Aussage über die Genauigkeit
des Klirrfaktors gemacht werden, da ebenfalls kein Referenzwert
vorhanden ist.
