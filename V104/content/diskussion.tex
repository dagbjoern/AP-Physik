\newpage
\section{Diskussion}
\label{sec:Diskussion}
Die Laufzeit des Wagens kann recht genau bestimmt werden, die größere Fehlerquelle
ist die Messung der Strecke zwischen den Lichtschranken, somit lässt sich die Geschwindigkeit
nur mit kleinen Abweichungen bestimmen.
Bei der Messung der Wellenlänge sind die ungenauen Lissajous-Figuren eine Fehlerquelle,
da diese sich nicht auf die Form einer Graden bringen lassen, was zu Abweichung bei der Wellenlänge
führt. Im Gegensatz dazu lassen sich die Frequenzen recht genau bestimmen, auch bei wiederholter
Messung ergibt sich der gleiche Wert, was zu einem $\sigma=0,0$ führt.
Desweiteren lässt sich bei den Messwerten aus den Graphen gut erkennen, dass eigendlich schon bei den geringen
Geschwindigkeiten ein sichtbarer Unterschied zwischen Formel \ref{eqn:frequenzempf} und \ref{eqn:frequenzsender} vorliegt.
Dieses ist erkennbar, da die Regressionsgrade der gemessenen Frequenzen, bei dem sich der Sender bewegt,
bei höheren Geschwindigkeiten leicht von der Geraden, die mit den Werten
aus der Wellenlängenmessung und der Formel \ref{eqn:frequenzempf} für einen bewegten Empfänger berechnet wurde, nach oben abweicht.
Folglich handelt es sich hierbei um eine sehr genau Messung, da meist die Abschätzung gilt, dass die Formeln bei v$\ll$c ungefähr gleich
sind, wie auch in der Auswertung angenommen.
