\section{Theorie}
\label{sec:Theorie}
Sind Wellen an ein Medium gebunden,
so muss zwischen einer Bewegung des Senders und des Empfängers differenziert werden.
Ohne Bewegung sendet der Sender eine Welle der Frequenz $\nu$ und der Empfänger
empfagt eine Frequenz $\nu$.
Bewegt sich der Empfänger mit der Geschwindigkeit $v$ auf den ruhenden Sender zu,
so überstreicht in der Zeit $\Delta t$
\begin{align}
  \Delta z=\frac{\Delta t\cdot v}{\lambda_0}
\end{align}
Wellenzüge.
Im Vergleich zu einem ruhenden Empfänger, an dem $\Delta n= \Delta t\nu$ Schwingungen
vorbeilaufen, erfasst der bewegte Empfänger
\begin{align}
  \Delta n + \Delta z = \Delta t \left(\nu_0 +\frac{v}{\lambda_0}\right)
\end{align}
Schwingunegen, in gleicher Zeit werden also mehr Schwingungen registriert.
Unter der Betrachtung der Frequenz als die Anzahl der Schwingungen pro
Zeiteinheit und der Ausbreitungsgeschwindigkeit
\begin{align}
c=\nu_0\cdot\lambda_0 \label{eqn:c},
\end{align}
gilt für die Frequenz eines bewegten Empfängers:
\begin{align}
  \nu_E=\nu_0\left(1+\frac{v}{c}\right)\label{eqn:frequenzempf}.
\end{align}
Die Frequenzänderung beträgt:
\begin{align}
  \Delta\nu=\nu_0\frac{v}{c}\label{eqn:aenderungempf}.
\end{align}
Bewegt sich der Empfänger vom Sender fort, so sinkt die Frequenz um \eqref{eqn:aenderungempf}
unter $\nu_0$.
Im anderen Fall bewegt sich der Sender und der Empfänger ruht.
Mit der Wellenlänge als Abstand zweier Punkte mit gleicher Phase, erscheint
die Wellenlänge als verkürzt um:
\begin{align}
  \Delta\lambda=\frac{v}{\nu_0}.
\end{align}
Unter Berücksichtigung der Ausbreitungsgeschwindigkeit $c=\nu_0\cdot\lambda_0$
folgt für die geänderte Frequenz:
\begin{align}
  \nu_Q=\nu_0\frac{1}{1-\frac{v}{c}}\label{eqn:frequenzsender}.
\end{align}
Wird die Formel \eqref{eqn:frequenzsender} in einer Reihe nach Potenzen von $\frac{v}{c}$
entwickelt, wie in Formel :
\begin{equation}
  \nu_Q=\nu_0\left(1+\frac{v}{c}+\left(\frac{v}{c} \right)^2+\left(\frac{v}{c}\right)^3...\right)= \nu_E +\nu_0\left(\left(\frac{v}{c}\right)^2+...\right).
\end{equation}
Hieran lässt sich erkennen, dass für $\lvert v \rvert \ll c$ der Unterschied zwischen
$\nu_Q$ und $\nu_E$ beliebig klein wird.
Auch bei elektromagnetischen Wellen tritt der Doppler-Effekt
auf, allerdings wird nicht zwischen bewegter Quelle und bewegtem Empfänger gfeachtet.
