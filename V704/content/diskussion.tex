\newpage
\section{Diskussion}
\label{sec:Diskussion}
Die aus der Messung bestimmten Werte für den Absoptionskoeffizienten $\mu$
für Kupfer und Blei,
\begin{align*}
\mu_{Cu}=(50,4\pm1,2)\si{\per\meter},\\
\mu_{Pb}=(112,0\pm4,0)\si{\per\meter}
\end{align*}
können nun mit dem entsprechenden berechnenten Compton-Absorptionskoeffizienten,
\begin{align*}
  \mu_\mathrm{Com_{cu}}={63,02\si{\per\meter}},\\
  \mu_\mathrm{Com_{pb}}={6,94\si{\per\meter}},
\end{align*}
verglichen werden.
Dabei fällt auf das der Absorptionskoeffizienten von Kupfer
in der nähe des berechneten Compton-Absorptionskoeffizienten liegt.
Folglich trägt auch diese nun dazu bei .
Im Gegensatz zu Kupfer ist der Absorptionskoeffizienten von Blei deutlich höher
als $\mu_\mathrm{Com_{pb}}$ somit kann davon ausgegangen werden, dass der
Photoeffekt ebenfalls einen hohen Beitrag zur Absorption liefert.
Deweitern lässt sich zu der Messung der $\gamma$-Absorption anmerken, dass
die aus dem Fitt berechneten Werte für $N_0$,
\begin{align*}
  N_{0_{Pb}}=(166,8\pm3,2)\si{\per\second},\\
  N_{0_{Cu}}=(155,6\pm1,7)\si{\per\second},
\end{align*}
nahe bei einander liegen und somit im Rahmen der Messung liegen. Die Geringe Abweichung lässt
sich durch die Variation bei dem Zerfall erklären.
Die Messung der $\beta$-Absorption liefert ebenfalls annehmbare Messwerte, da
die den Messwerten sich deutlich in die zwei Bereiche einteilen lässen
und sogar die bestimmte Gerade im ersten Bereich durch alle messpunkte verrläuft.
