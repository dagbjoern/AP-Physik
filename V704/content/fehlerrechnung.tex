\newpage
\section{Fehlerrechnung}
Die Mittelwerte bestimmen sich in der Auswertung nach:
\begin{align}
  \bar{x}=\frac{1}{n} \sum_{i=1}^n x_i\,.
\end{align}
Für die Standardabweichung ergibt sich:
\begin{align}
 s_i=\sqrt{\frac{1}{n-1}\sum_{j=1}^n (v_j-\bar{v_i})^2}
\end{align}
mit $v_j$ mit $j=1,..,n$ als Wert mit zufällig behafteten Fehlern.\\
Diese werden mit Hilfe von
Numpy 1.9.2, einer Erweiterung von Python 3.2.0, berechnet.
Die Fehlerfortpflanzung wird mit der Gauß´schen Fehlerfortpflanzung berechnet
 \eqref{eqn:gaus}.
\begin{equation}
\Delta f=\sqrt{\sum_{j=1}^n \left(\frac{\partial f}{\partial x_j}\Delta x_j \right)^{2} }\label{eqn:gaus}.
\end{equation}
Diese wird von der Erweiterung Uncertainties 2.4.6.1 von Python 3.2.0 übernommen.\\
\\
Abweichungen von den Theoriewerten werden mit der Formel
\begin{align}
  a=\frac{|a_\mathrm{gemessen}-a_\mathrm{theorie}|}{a_\mathrm{theorie}} \label{eqn:abweich}
\end{align}
berechnet.
\newpage
