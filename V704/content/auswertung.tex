\section{Auswertung}
\label{sec:Auswertung}
%\subsection{Die $\gamma$-Absorption}
Der hier verwendete $\gamma$-Strahler ist $^{137}$Cs.
Bei der Nullmessung über $800\si{\second}$ wurde ein Werte von
\begin{align*}
Z_u=\pm
\intertext{gemessen. Dies entspricht einer Impulsrate von:}
N_u=\pm\si{\second}
\end{align*}
Der Fehler dieser und den folgenden Messungen entspricht $\sqrt{Z}$
In den Tabelle sind
die Messwerte für $\gamma$-Strahlung bei den Absorptionsmaterialien Kupfer \ref{tab:yCu}
und Blei \ref{tab:yPb} aufgelistet.
\begin{table}
  \centering
  \caption{Messwerte bei einer Kupferplatte.}
  \label{tab:yCu}
  \begin{tabular}{c c c c}
Zeit $s$ in $\si{\second}$& Impulse $Z$  & Dicke $D$ in \si{\milli\meter} & Impulsrate $N-N_u$ in $\si{\per\second}$\\
       \midrule
       40 & 6081\pm78 & 0,48 &152,0\pm1,9 \\
       40 & 5948\pm77 & 1,92 &148.7\pm1,9 \\
       40 & 4364\pm66 & 6,92 &109,1\pm1,6 \\
       40 & 5291\pm73 & 2,88 &132,3\pm1,8 \\
       40 & 3692\pm61 & 10   & 92,3\pm1,5 \\
       40 & 2370\pm49 & 20   & 59,3\pm1,2 \\
       40 & 1904\pm44 & 25   & 47,6\pm1,1 \\
       40 & 1411\pm38 & 30   & 35,3\pm0,9 \\
       40 & 1058\pm33 & 35   & 26,5\pm0,8 \\
       40 & 850\pm29  & 40   & 21,3\pm0,7 \\
       40 & 770\pm28  & 45   & 19,3\pm0,7 \\
       40 & 546\pm23  & 50   & 13,7\pm0,6 \\
       40 & 321\pm18  & 60   & 8,0\pm0,4 \\
      \bottomrule
    \end{tabular}
\end{table}
\begin{table}
  \centering
  \caption{Messwerte bei einer Bleiplatte.}
  \label{tab:yPb}
  \begin{tabular}{c c c c}
Zeit $s$ in $\si{\second}$& Impulse $Z$  & Dicke $D$ in \si{\milli\meter} & Impulsrate $N-N_u$ in $\si{\per\second}$ \\
       \midrule
       40  & 5583\pm75 & 1,4  & 139,6\pm1,9\\
       40  & 4965\pm70 & 2,8  & 124,1\pm1,8\\
       40  & 4456\pm67 & 4,2  & 111,4\pm1,7\\
       40  & 2059\pm45 & 10   & 51,5\pm1,1\\
       40  & 1509\pm39 & 14,2 & 37,7\pm1,0\\
       40  & 643\pm25  & 20   & 16,1\pm0,6\\
       40  & 451\pm21  & 24,2 & 11,3\pm0,5\\
       40  & 255\pm16  & 30   & 6,4\pm0,4 \\
       60  & 290\pm17  & 34,2 & 4,8\pm0,3\\
       100 & 318\pm18  & 40   & 3,2\pm0,2\\
       200 & 455\pm21  & 50   & 2,3\pm0,1\\
       300 & 351\pm19  & 60   & 1,2\pm0,1\\
      \bottomrule
    \end{tabular}
\end{table}
Aus diesen Tabellen werden nun
die Impulsrate $N-N_u$ in Abbhängigkeit von der Plattendicke $D$
aufgetragen. Die Abbildung \ref{fig:cu} enthält die
Messwerte für die Kupferplatten und Abbildung \ref{fig:pb}
die für die Bleiplatten.

\begin{figure}
  \centering
  \includegraphics[width=0.7\textwidth]{a)cu.pdf}
  \caption{Impulsrate $N$ in Abhängigkeit der Dicke $D$ der Kupferplatten.}
  \label{fig:cu}
\end{figure}

\begin{figure}
  \centering
  \includegraphics[width=0.7\textwidth]{a)pb.pdf}
  \caption{Impulsrate $N$ in Abhängigkeit der Dicke $D$ der Bleiplatten.}
  \label{fig:pb}
\end{figure}

Durch diese Messwerte wird nun versuch eine e-funktion in der Form
\begin{align*}
  N=A\cdot e^{C\cdot D}
\end{align*}
zu fitten.
Die Koeffizienten $A$ und $C$ entsprechen $N_0$ und $mu$ aus dem Absorptionsgesetz \eqref{eqn:abs}.
Somit ergibt sich für


\begin{align*}
N_L=2,6867811%\cdot10^{25}\si{\per\meter\tothe{3}}\cite{constants}
\end{align*}

%werte Für Kufer
%\begin{align*}
%z_{Cu}=29\\
%V_{M_{Cu}}=7,11 \cdot 10^{-6}\si{\meter\tothe{3}\per\mol}  \cite{chemie.de}
%\end{align*}
%werte Für Blei
%\begin{align*}
%z_{Pb}=82\\
%V_{M_{Pb}}=18,26 \cdot 10^{-6}x\si{\meter\tothe{3}\per\mol}  \cite{chemie.de}
%\end{align*}

%\subsection{ $\beta$-Apsorption}
